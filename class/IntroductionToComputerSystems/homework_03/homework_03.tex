\documentclass[12pt]{article}

\author{陶文华}

\usepackage{../../homeworks_preamble}
\title{计算机系统概论-作业3}

\begin{document}
    \maketitle
    \section{猴子吃桃问题}
    \begin{enumerate}
        \item \texttt{\%edi}
        \item \texttt{.L3}
        \item \texttt{\%rax}
        \item \texttt{.L2}
        \item \texttt{ret}
        \item \texttt{\$8}
    \end{enumerate}
    \section{栈变量位置}
    \begin{figure}[htbp]
        \centering
        \begin{tabular}{|c|c|}
            \hline
            变量 & 十进制形式的 offset 或者说明\\\hline
            \texttt{a} & \texttt{-24(\%rbp)}\\\hline
            \texttt{a[2]} & \texttt{-16(\%rbp)}\\\hline
            \texttt{x} & \texttt{\%edi},作为传递参数不进栈\\\hline
            \texttt{buf} & \texttt{-32(\%rbp)}\\\hline
            \texttt{buf[3]} & \texttt{-29(\%rbp)}\\\hline
            \texttt{\%rbx} 的保存值 & \texttt{-8(\%rbp)}\\\hline
        \end{tabular}
    \end{figure}
    \section{非常规过程调用}
    \begin{enumerate}
        \item 返回地址是 \texttt{GET} 调用处的下一条地址,返回值为 1.
        \item
        \begin{enumerate}
            \item 存放 \texttt{GET} 的第一个参数.
            \item 存放 \texttt{GET} 的返回地址.
            \item 将 \texttt{GET} 的第一个参数放入 \texttt{\%ecx} 从而下一步将其存放到 \texttt{72(\%eax)}.
            \item \texttt{60}.
            \item 将存储的返回地址入栈,以便 \texttt{ret} 返回时能够返回到存储的返回地址.
        \end{enumerate}
    \end{enumerate}
    \section{类函数调用}
    \begin{enumerate}
        \item 作为隐藏的第一个参数在 \texttt{\%rdi} 中传递,处于 \texttt{-32(\%rbp)} 处.
        \item 使用隐藏的第一个参数 \texttt{this} 指针进行寻址.
    \end{enumerate}
    \section{补全 C 代码}
    \begin{enumerate}
        \item \texttt{f[1]}
        \item \texttt{b.i->f[3]}
        \item \texttt{h->b.j} 或 \texttt{i->e}
        \item \texttt{i->g->d->a[1]}
    \end{enumerate}
    \section{内存布局分析}
    \begin{enumerate}
        \item \texttt{short}
        \item \texttt{uint64\_t}
        \item \texttt{int32\_t}
        \item \texttt{double}
        \item \texttt{5}
    \end{enumerate}
    内存布局:

    \texttt{|a,2 bytes|pad,6 bytes|b, 8 bytes|\\
        |c,4 bytes|pad, 4 bytes|d, 8 bytes|
        \ldots (repeat 4 times)}
\end{document}
