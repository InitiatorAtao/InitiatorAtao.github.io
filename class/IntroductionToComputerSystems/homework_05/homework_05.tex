\documentclass[12pt]{article}

\author{陶文华}

\usepackage{../../homeworks_preamble}
\title{计算机系统概论-第五次作业}

\begin{document}
    \maketitle
    \section{\texttt{fork()}}   
    \subsection{程序会输出多少行?(空行不计算在内)}
    10 行,分别为 2 行 \texttt{counter = 1} 和 8 行 \texttt{counter = 2}.
    \subsection{程序的全部输出中,第一行和最后一行分别会是什么?}
    第一行为 \texttt{counter = 1},最后一行为 \texttt{counter = 2}.
    \subsection{根据系统对进程的调度情况,程序一共有多少种可能的输出结果?(注:如果同一时间有若干进程在同时运行,他们运行的先后顺序的不同可能导致输出结果不同)}
    第一行必定为 \texttt{counter = 1},剩余一行 \texttt{counter = 1} 可能在 2-6 行出现,故总共可能有 5 种可能输出结果.
    \section{文件读写}
    \subsection{当执行 ./program1 后,文件 file1.txt 的内容是什么?}
    \texttt{ddccbbaa}
    \subsection{ 从以下几点解释一下为什么是这个结果(言之有理即可,简短作答):}
    \begin{itemize}
        \item  \texttt{./program2} 做了什么?

            从当前的标准输入读取两个字节到 \texttt{buf2} 再将其写入当前标准输出.
        \item \texttt{./program1} 在 \texttt{fork()} 之前做了什么,相应的 \texttt{buf} 的内容如何变化?

            从 \texttt{fd\_x} 读取两个字节到 \texttt{buf},再从 \texttt{fd\_y} 读取四个字符到 \texttt{buf+2}. (都是从 \texttt{file1.txt} 的起始位置开始读,故 \texttt{buf} 写入 \texttt{aaaabb})

        \item \texttt{./program1} 调用 \texttt{fork()} 了以后,子进程在干什么?

            子进程将标准输出重定向到 \texttt{fd\_x},将标准输入重定向到 \texttt{fd\_y} 并转到 \texttt{program2} 运行,也就是从 \texttt{fd\_y} 读取两个字符 (\texttt{cc}) 写入 \texttt{fd\_x} (替换了文件中的 \texttt{bb}).
        \item 子进程返回后,\texttt{./program1} 又做了什么,相应的 \texttt{buf} 的内容如何变化?

            从 \texttt{fd\_y} 再向 \texttt{buf+6} 读入两个字节 (\texttt{dd}),此时 \texttt{buf=aaaabbdd}.

            从 \texttt{buf+6} 向 \texttt{fd\_z} 写入两个字节 (\texttt{dd},替换了文件中的 \texttt{aa}).

            从 \texttt{buf+4} 向 \texttt{fd\_x} 写入两个字节 (\texttt{bb},替换了文件中的 \texttt{cc}).

            从 \texttt{buf+2} 向 \texttt{fd\_x} 写入两个字节 (\texttt{aa},替换了文件中的 \texttt{dd})
    \end{itemize}
    综上,文件内容被替换为 \texttt{ddccbbaa}
    \section{信号量}
    \texttt{1 0 0}

    \texttt{P(board) V(physics) V(chemistry)}

    \texttt{physics board}

    \texttt{chemistry board}
    \section{线程与子线程}
    \subsection{}\texttt{a=1}
    \subsection{}\texttt{a=1}

    \texttt{a=1}
    \subsection{}\texttt{a=2}

    \texttt{a=1}

    \texttt{a=1}

    以上三行的输出顺序任意
    \section{信号处理}
    ACDE
\end{document}
