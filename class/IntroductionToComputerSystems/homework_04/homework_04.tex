\documentclass[12pt]{article}

\author{陶文华}

\usepackage{../../homeworks_preamble}
\title{计算机系统概论-第四次作业}

\begin{document}
    \maketitle
    \section{虚拟内存映射}
    \subsection{静态参数的设置}
    \begin{enumerate}
        \item 19 9 10
        \item 9 23
    \end{enumerate}
    \subsection{地址翻译}
    \begin{enumerate}
        \item VPN 是 \texttt{0x6F56DF}
        \item 物理地址是 \texttt{0x466EF}
        \item C
    \end{enumerate}
    \subsection{快表}
    \begin{enumerate}
        \item 3 20
        \item \texttt{0xDEADB} 7
        \item E
        \item B
    \end{enumerate}
    \subsection{懒惰文件映射}
    2 3 2 1024
    \subsection{硬件查询页表过程,及走向多级页表}
    3 1 A 11MB (不考虑内存对齐)
    \section{链接和重定位}
    \subsection{请列出 \texttt{test\_call} 中的五个函数调用,在链接期间需要全局重定位参与的调用有哪些,并简短解释原因.}
    \begin{itemize}
        \item \texttt{test1},使用 \texttt{extern} 关键字显示引用外部函数指针
        \item \texttt{test4},使用 \texttt{extern} 关键字显式引用外部函数
    \end{itemize}
    \subsection{请列出 \texttt{foo.o} 文件中 \texttt{.symtab} 中必须以全局符号存储的符号名,并简短解释原因}
    \begin{itemize}
        \item \texttt{test1},使用 \texttt{extern} 关键字显示引用外部函数指针
        \item \texttt{test3},非静态全局函数参与全局链接
        \item \texttt{test4},使用 \texttt{extern} 关键字显式引用外部函数
        \item \texttt{test\_call},非静态全局函数参与全局链接
    \end{itemize}
    \subsection{最终可执行文件链接外部变量,函数所处的对象文件时,可以使用静态链接也可以使用动态链接.在运行时谁更有性能优势?请简要解释原因}
    静态链接更有性能优势,因为其启动时不需要加载链接库,运行时不需要进行函数查找与符号解析,可以直接由编译期确定的符号地址进行引用.

    相比于静态链接,动态链接程序在启动时需要加载链接库,运行时需要根据符号计算链接库的引用地址,可能造成一定的性能损耗.
\end{document}
