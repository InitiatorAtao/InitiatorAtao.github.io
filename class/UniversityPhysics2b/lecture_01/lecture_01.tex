\documentclass[12pt]{article}

\author{陶文华}

\usepackage{../../lectures_preamble}

\begin{document}
    \section{静电场}
    \begin{definition}
        (电荷) $Q$ 或 $q$,单位为库仑 $C$.最小单元是电子电量 $e\approx 1.602\times 10^{-19} \ \mathrm{C}$.
    \end{definition}
    \begin{theorem}
        (电荷) 电荷守恒,且在不同参考系下不变 (电荷的相对论不变性)
    \end{theorem}
    \begin{theorem}
        (库伦定律) $\bm{F}_{21}=k \frac{q_1q_2}{r_{21}^2}\bm{e}_{r21}$.
    \end{theorem}
    \begin{definition}
        (真空介电常量) $\epsilon_0=\frac{1}{4\pi k}=8.85\times 10^{-12} \ \mathrm{C^2 / (N\cdot m^2)}$
    \end{definition}
    由真空介电常量书写库仑定律 $\bm{F}_{21}=\frac{q_1q_2}{4\pi \epsilon_0 r_{21}^2}\bm{e}_{r 21}$.
    \begin{definition}
        (电场强度) $\bm{E}=\frac{\bm{F}}{q}=\frac{q}{4\pi\epsilon_0 r^2}\bm{e}_{r}$
    \end{definition}
    \begin{definition}
        (电通量) $\mathrm{d}\Phi_{e}=\bm{E}\cdot \mathrm{d}\bm{S}=E\mathrm{d}S\cos \theta$
    \end{definition}
    \begin{theorem}
        (高斯定律) $\oint_{S}^{}\bm{E}\cdot \mathrm{d}\bm{S}=\frac{1}{\epsilon_0}\sum_{}^{}q_{\texttt{in}}$
    \end{theorem}
    研究随时间变化的电场时,库仑定律不再有效,但高斯定律仍有效.
    \section{电势}
    \begin{theorem}
        (静电场环路定理) $\oint_{}^{}\bm{E}\cdot \mathrm{d}\bm{r}=0$
    \end{theorem}
    \begin{definition}
        (电势差和电势) $\phi_1-\phi_2=\int_{P_1}^{P_2}\bm{E}\cdot \mathrm{d}\bm{r}$ 选定电势零点 $P_0$ 后, $\phi=\int_{P}^{P_0}\bm{E}\cdot\mathrm{d}\bm{r}$
    \end{definition}
    当电荷只分布在有限区域时,电势零点通常选在无限远处.
    \begin{theorem}
        (电势梯度) $\bm{E}=-\grad \phi$
    \end{theorem}
    \begin{definition}
        (静电势能) $W=q_0\phi$,为电荷在外电场中的势能.
    \end{definition}
    标准单位为焦耳 $ \ \mathrm{J}$,或者电子伏 $1\ \mathrm{eV}=1.60\times 10^{-19} \ \mathrm{J}$ 表示一个电子经过 $1 \ \mathrm{V}$ 电势差获得的动能.
    \begin{definition}
        (电荷系的静电能) 将各电荷从现有位置分散到无限远时相互静电力做功, $W=\frac{1}{2}\sum_{}^{}q_{i}\phi_{i}=\frac{1}{2}\int_{q}^{}\phi\mathrm{d}q$.
    \end{definition}
    \begin{definition}
        (电场能量密度) $w_{e}=\frac{\mathrm{d}W}{\mathrm{d}V}=\frac{\epsilon_0E^2}{2}$,这里的 $W$ 指的是电荷系的静电能.
    \end{definition}
    \section{静电场中的导体}
    \begin{definition}
        (导体的静电平衡) 导体内部和表面都没有电荷定向移动的状态.
    \end{definition}
    静电平衡时导体表面紧邻处的电场强度必定和导体表面垂直,导体为等势体,其表面是等势面,内部净电荷为零,电荷只能分布在表面,且表面电荷密度与紧邻处电场强度大小成正比.
    \begin{theorem}
        (静电平衡导体表面的电场强度) $\sigma=\epsilon_0E$.
    \end{theorem}
    导体静电平衡时,表面曲率越大的地方面电荷密度越大.
    \begin{theorem}
        (唯一性定理) 有若干个导体存在时,如果对于任意导体都已知其上总电量或电势 (称为边界条件) ,则空间的电场分布和导体表面的电荷分布唯一确定.
    \end{theorem}
    \section{静电场中的电介质}
    \begin{definition}
        (相对介电常量) 又称相对电容率 $\epsilon_{r}$,加入电介质后 $E=\frac{E_0}{\epsilon_r}$.介电常量即 $\epsilon=\epsilon_0\epsilon_{r}$.
    \end{definition}
    \begin{definition}
        (固有电矩) 内部电荷分布不对称的分子 (极性分子) 的电矩.
    \end{definition}
    \begin{definition}
        (感生电矩) 内部电荷分布对称的分子 (非极性分子) 在外电场中产生的电矩.
    \end{definition}
    感生电矩的方向总与外加电场的方向相同.
    \begin{definition}
        (面束缚/极化电荷) 在外电场影响下出现在电介质表面的电荷.
    \end{definition}
    \begin{definition}
        (电极化强度) $\bm{P}=\frac{\sum_{}^{}\bm{p}_{i}}{\Delta V}$. 其中 $\bm{p}_{i}$ 表示分子电矩, $\Delta V$ 为宏观极小微观极大微元.
    \end{definition}
    对于固定的固有电矩,令 $n$ 为分子密度,则有 $\bm{P}=n \bm{p}$.
    \begin{theorem}
        (电场与电极化强度之间的关系) 当 $\bm{E}$ 不太强时,各向同性电介质中有 $\bm{P}=\epsilon_0(\epsilon_{r}-1)\bm{E}$.\mn{也记 $\chi=\epsilon_{r}-1$ 为电极化率,有 $\bm{P}=\epsilon_0\chi \bm{E}$}
    \end{theorem}
    \begin{definition}
        (介电强度,击穿场强) 一种电介质材料所能承受的不被击穿的最大电场强度,单位为 $\ \mathrm{kV / mm}$.
    \end{definition}
    \begin{definition}
        (电位移) $\bm{D}=\epsilon_0 \bm{E}+\bm{P}=\epsilon_0\epsilon_r \bm{E}$.
    \end{definition}
    \begin{theorem}
        ($D$ 的高斯定理) $\oint_{S}^{}\bm{D}\cdot \mathrm{d}\bm{S}=\sum_{}^{}q_{0\texttt{in}}$.
    \end{theorem}
    \begin{theorem}
        (静电场的边界条件) 电介质分界面上无自由电荷存在时,分界面两侧电场强度的切向分量相等,电位移矢量的法向分量相等.
    \end{theorem}
    \begin{theorem}
        ($D$ 线的折射定律) $D$ 线通过电介质分界面上的入射角 $\theta_1$ 和折射角 $\theta_2$ 满足 $\frac{\tan \theta_1}{\tan\theta_2}=\frac{\epsilon_{r 1}}{\epsilon_{r 2}}$.
    \end{theorem}
    \begin{definition}
        (电容器的电容) $C=\frac{Q}{U}$.单位为法拉 $\ \mathrm{F}$.
    \end{definition}
    常用单位为 $1 \ \mathrm{\mu F}=10^{-6} \ \mathrm{F}$, $1 \ \mathrm{p F}=\times 10^{-12} \ \mathrm{F}$.
    \begin{theorem}
        (电容器的串并联) 并联电容器的电容求和,串联电容器电容倒数求和.
    \end{theorem}
    \begin{definition}
        (电容器的能量) 能量密度 $w_{e}=\frac{1}{2}\epsilon E^2=\frac{1}{2}DE=\frac{1}{2}\bm{D}\cdot \bm{E}$, 能量 $W=\int_{}^{}w_{e}\mathrm{d}V=\int_{}^{}\frac{\epsilon E}{2}\mathrm{d}V$.或直接使用放电做功得 $W=\frac{1}{2}CU^2=\frac{1}{2}QU$.
    \end{definition}
    \section{恒定电流}
    \begin{definition}
        (电流) $I=\frac{\Delta q}{\Delta t}$,单位为安培 $\ \mathrm{A}$.
    \end{definition}
    \begin{definition}
        (电流密度) $\mathrm{d}I=\bm{J}\cdot \mathrm{d}\bm{S}$,称 $\bm{J}$ 为小面积 $\mathrm{d}\bm{S}$ 处的电流密度.
    \end{definition}
    导体中 $\bm{J}=ne \overline{\bm{v}}$,其中 $n$ 为载流子密度, $\overline{\bm{v}}$ 为平均速度.对于有限面积, $I=\int_{S}^{}\mathrm{d}I=\int_{S}^{}\bm{J}\cdot \mathrm{d}\bm{S}$.
    \begin{theorem}
        (电流的连续性方程) $\oint_{S}^{}\bm{J}\cdot \mathrm{d}\bm{S}=-\frac{\mathrm{d}q_{\texttt{in}}}{\mathrm{d}t}$.
    \end{theorem}
    \begin{definition}
        (恒定电流) 导体内各处的电流密度都不随时间变化的电流.
    \end{definition}
    \begin{theorem}
        (恒定电流) $\oint_{S}^{}\bm{J}\cdot \mathrm{d}\bm{S}=0$.
    \end{theorem}
    \begin{theorem}
        (恒定电场) 恒定电流下导体内电荷分布不随时间改变,故电场不随时间改变,称恒定电场.
    \end{theorem}
    恒定电场的性质与静电场基本相同.
    \begin{theorem}
        (欧姆定律) $U=IR$.
    \end{theorem}
    \begin{theorem}
        (电阻定律) $R=\rho \frac{l}{S}$.
    \end{theorem}
    有时也使用电导率 $\sigma=\frac{1}{\rho}$,此时 $R=\frac{1}{\sigma S}$.
    \begin{theorem}
        (欧姆定律的微分形式) $\bm{J}=\sigma \bm{E}$.
    \end{theorem}
    注意欧姆定律及其微分形式仅对于一般的金属或电解液成立.
    \begin{definition}
        (电源的电动势) $E=\frac{A_{\texttt{ne}}}{q}$,其中 $A_{\texttt{ne}}$ 表示非静电力做功.
    \end{definition}
    有时将非静电力作用看作等效"非静电场"作用,有 $E=\int_{}^{}\bm{E}_{\texttt{ne}}\cdot \mathrm{d}\bm{r}$.
    \begin{theorem}
        (电容器的充电与放电) 充电时 $q=CE\left( 1-e^{-\frac{t}{RC}} \right) $, $i=\frac{\mathrm{d}q}{\mathrm{d}t}=\frac{E}{R}e^{-\frac{t}{RC}}$,放电时 $q=Qe^{-\frac{t}{RC}}$, $i=\frac{Q}{RC}e^{-\frac{t}{RC}}$.
    \end{theorem}
    \begin{definition}
        (时间常量) $\tau=RC$,为变量变化倍率 $\frac{1}{e}$ 时的时间.
    \end{definition}
    回路线度远小于 $c\tau$ 时称为似稳电场,也可以应用基尔霍夫方程.
    \begin{definition}
        (电子的平均自由飞行时间) $\tau=\frac{\sum_{}^{}t_{i}}{n}$.
    \end{definition}
    \begin{theorem}
        (金属的电导率的微观解释) $\sigma=\frac{ne^2\tau}{m}$.
    \end{theorem}
    \begin{definition}
        (热功率密度,焦耳定律的微分形式) $p=\sigma E^2$.
    \end{definition}
    \section{磁场和它的源}
    \begin{definition}
        (磁感应强度) $\bm{F}=q\bm{v}\times \bm{B}$. $\bm{B}$ 称磁感应强度,单位特斯拉 $\ \mathrm{T}$ 或非标准的高斯 $1 \ \mathrm{G}=10^{-4} \ \mathrm{T}$.
    \end{definition}
    \begin{definition}
        (磁通量) $\Phi=\int_{S}^{}\bm{B}\cdot \mathrm{d}\bm{S}$.单位韦伯 $1\ \mathrm{Wb}=1 \ \mathrm{T\cdot m^2}$.
    \end{definition}
    \begin{definition}
        (真空磁导率) $\mu_0=\frac{1}{\epsilon_0c^2}=4\pi \times 10^{-7} \ \mathrm{N / A^2}$.
    \end{definition}
    \begin{theorem}
        (磁通连续定理) $\oint_{S}^{}\bm{B}\cdot \mathrm{d}\bm{S}=0$.
    \end{theorem}
    \begin{theorem}
        (毕奥-萨伐尔定律) $\mathrm{d}B=\frac{\mu_0}{4\pi}\frac{I\mathrm{d}\bm{l}\times \bm{e}_{r}}{r^2}$.
    \end{theorem}
    \begin{theorem}
        (匀速运动点电荷的磁场) $\bm{B}=\frac{\mu_0}{4\pi}\frac{q \bm{v}\times \bm{e}_{r}}{r^2}$.
    \end{theorem}
    在运动速度较小时,近似的使用 $E=\frac{q}{4\pi\epsilon_0r^2}\bm{e}_{r}$ 可得 $B=\frac{1}{c^2}\bm{v}\times \bm{E}$.
    \begin{theorem}
        (安培环路定理) $\oint_{C}^{}\bm{B}\cdot \mathrm{d}\bm{r}=\mu_0\sum_{}^{}I_{\texttt{in}}$.
    \end{theorem}
    电流 $I_{\texttt{in}}$ 指的是闭合恒定电流 (无限长直电流可视为在无穷远处闭合).用于求磁场分布时根据对称性等选取 $\bm{B}$ 大小恒定的路径,使得其能以标量形式从积分号内提取出来.
    \begin{theorem}
        (变化电场产生磁场) $\oint_{C}^{}\bm{B}\cdot \mathrm{d}\bm{r}=\mu_0\epsilon_0 \frac{\mathrm{d}}{\mathrm{d}t}\int_{S}^{}\bm{E}\cdot \mathrm{d}\bm{S}$,其中 $S$ 是任意曲面使得 $\partial S=C$.
    \end{theorem}
    \begin{theorem}
        (推广/普遍安培环路定理) $\oint_{C}^{}\bm{B}\cdot \mathrm{d}\bm{r}=\mu_0\left( I_{\texttt{c,in}}+\epsilon_0 \frac{\mathrm{d}}{\mathrm{d}t}\int_{S}^{}\bm{E}\cdot \mathrm{d}\bm{S} \right) =\mu_0\int_{S}^{}\left( \bm{J}_{c}+\epsilon_0 \frac{\partial \bm{E}}{\partial t} \right) \cdot \mathrm{d}\bm{S}$.
    \end{theorem}
    \begin{definition}
        (位移电流) $I_{d}=\epsilon_0 \frac{\mathrm{d}\Phi_{e}}{\mathrm{d}t}=\epsilon_0 \frac{\mathrm{d}}{\mathrm{d}t}\int_{S}^{}\bm{E}\cdot \mathrm{d}S$.
    \end{definition}
    \begin{definition}
        (位移电流密度) $\bm{J}_{d}=\epsilon_0 \frac{\partial \bm{E}}{\partial t}$.
    \end{definition}
    \begin{definition}
        (全电流) $I=I_{c}+I_{d}=\int_{S}^{}\bm{J}_{c}\cdot\mathrm{d}\bm{S}+\int_{S}^{}\bm{J}_{d}\cdot \mathrm{d}\bm{S}=\int_{S}^{}\left( J_{c}+\epsilon_0 \frac{\partial \bm{E}}{\partial t} \right) \cdot \mathrm{d}\bm{S}$.
    \end{definition}
    \begin{theorem}
        (全电流的连续性) $\oint_{S}^{}I\cdot \mathrm{d}\bm{S}=0$.
    \end{theorem}
    \section{磁力}
    \begin{theorem}
        (磁场中的匀速圆周运动) 半径 $R=\frac{mv}{qB}=\frac{p}{qB}$,回旋周期 $T=\frac{2\pi m}{qB}$.
    \end{theorem}
    \begin{definition}
        (螺距) 含有平行磁场和垂直磁场分速度的带电粒子运行的螺旋轨迹之间的间隔 $h=v_{\parallel}T=\frac{2\pi m}{qB}v_{\parallel}$.
    \end{definition}
    \begin{definition}
        (磁镜) 逐渐增强的磁场,能使粒子发生"反射".
    \end{definition}
    \begin{definition}
        (磁瓶) 两个相对的磁镜构成的能约束带电粒子的磁场分布.
    \end{definition}
    \begin{definition}
        (霍尔效应) 外磁场中的通电金属窄条边缘出现的电压 $U_{\texttt{H}}=\frac{IB}{nqb}$,其中 $n$ 为载流子密度, $b$ 为磁场穿透方向的厚度.方向由载流子电性以及左手定则确定.
    \end{definition}
    \begin{definition}
        (安培力) $\mathrm{d}\bm{F}=I\mathrm{d}\bm{l}\times \bm{B}$.
    \end{definition}
    \begin{theorem}
        (载流线圈在均匀磁场中受的磁力矩) $\bm{M}=\bm{m}\times \bm{B}$,其中 $\bm{m}$ 为磁矩.
    \end{theorem}
    \begin{definition}
        (磁矩的势能) $W_{m}=-\bm{m}\cdot \bm{B}$.
    \end{definition}
    \begin{theorem}
        (平行载流导线间的相互作用力) $F=\frac{\mu_0I_1I_2}{2\pi d}$,其中 $d$ 为导线距离, $F$ 为单位长度线段所受作用力,同向相吸,反向相斥.
    \end{theorem}
    \section{磁场中的磁介质}
    \begin{definition}
        (相对磁导率) $B=\mu_{r}B_0$,$\mu_{r}$ 称相对磁导率.
    \end{definition}
    \begin{definition}
        (抗磁质,顺磁质,铁磁质) 抗磁质 $\mu_{r}$ 略小于 1,顺磁质 $\mu_{r}$ 略大于 1,铁磁质 $\mu_{r}$ 远大于 1 且随 $B_0$ 的大小发生变换.
    \end{definition}
    \begin{definition}
        (磁偶极矩) 简称磁矩 $\bm{m}=IS\bm{e}_{n}$. $I$ 为圆电流,磁矩方向与其满足右手螺旋关系.
    \end{definition}
    \begin{definition}
        (玻尔磁子) $m_{B}=\frac{e}{m_{e}}s=\frac{e}{2m_{e}}\hbar =9.27\times 10^{-24} \ \mathrm{J / T}$.
    \end{definition}
    \begin{definition}
        (固有磁矩) 某些物质正常情况下磁矩矢量和具有的一定值,这些物质称作顺磁质.
    \end{definition}
    \begin{definition}
        (感生磁矩) 外磁场作用下电子轨道发生变化产生的附加磁矩.
    \end{definition}
    感生磁矩的方向与外加磁场方向相反.
    \begin{definition}
        (束缚/磁化电流) 磁介质在外磁场中磁矩变化在磁介质表面产生的电流.
    \end{definition}
    \begin{definition}
        (磁化) 磁介质表面上出现束缚电流的现象.
    \end{definition}
    \begin{definition}
        (磁化强度) $\bm{M}=\frac{\sum_{}^{}\bm{m}_{i}}{\Delta V}$,单位为安每米 $\ \mathrm{A / m}$.
    \end{definition}
    \begin{theorem}
        (磁化强度与外磁场的关系) $\bm{M}=\frac{\mu_{r}-1}{\mu_0\mu_{r}}\bm{B}$.
    \end{theorem}
    \begin{theorem}
        (面束缚电流密度) $j'=\bm{M}\times \bm{e}_{n}$,其中 $\bm{e}_{n}$ 是磁介质表面的外正法线方向的单位矢量.
    \end{theorem}
    \begin{theorem}
        (总束缚电流) $I'=\oint_{L}^{}\mathrm{d}I'=\oint_{L}^{}\bm{M}\cdot \mathrm{d}\bm{r}$ 代表 $L$ 包围的总束缚电流.
    \end{theorem}
    总束缚电流表示的是与所求环路铰链的电流.
    \begin{definition}
        (磁导率) $\mu=\mu_0\mu_{r}$.
    \end{definition}
    \begin{definition}
        (磁场强度) $\bm{H}=\frac{\bm{B}}{\mu_0}-\bm{M}=\frac{\bm{B}}{\mu_0\mu_{r}}=\frac{\bm{B}}{\mu}$.辅助物理量.单位为安每米 $\ \mathrm{A / m}$.
    \end{definition}
    磁场强度不受磁介质影响,可以直接由环路定理求解,而 $\bm{B}$ 由磁导率从 $\bm{H}$ 间接求解,在铁磁质等磁导率发生变化的情况下不能直接求解.
    \begin{theorem}
        ($\bm{H}$ 的环路定理) $\oint_{L}^{}\left( \frac{\bm{B}}{\mu_0}-\bm{M} \right) \cdot \mathrm{d}\bm{r}=\oint_{L}^{}\bm{H}\cdot \mathrm{d}\bm{r}=\sum_{}^{}I_{0\texttt{in}}$.
    \end{theorem}
    在无磁介质的情况下 $\bm{M}=0$ 还原为 $\bm{B}$ 的环路定理.
    \begin{definition}
        (磁化曲线) 关于磁介质的 $H-B$ 或 $H-M$ 曲线,从未磁化开始的曲线叫起始磁化曲线.
    \end{definition}
    \begin{definition}
    (铁磁质的性质) $B$ 随 $H$ 增大逐渐饱和,称磁饱和状态,在饱和后减小 $H$, $B$ 减小得比增加时慢,称磁滞效应, $H$ 恢复到零时仍保留的磁化状态称剩磁 $B_{r}$,将剩磁完全消失的反向磁场强度称矫顽力 $H_{c}$.正反向磁化曲线形成的闭合曲线称磁滞回线.矫顽力小的材料\mn{如纯铁,硅钢,铁镍坡莫合金}磁滞回线较瘦,称软磁材料.矫顽力大的材料\mn{如碳钢,钨钢,铝镍钴合金}磁滞回线较胖,称硬磁材料.高温使铁磁质变为顺磁质的温度称居里点.铁磁体内磁矩相同的小区域称磁畴\mn{线度约 $10 ^{-4} \ \mathrm{m}$}.铁磁质反复磁化发热引起的能量损失称磁滞损耗或"铁损"\mn{因此交流电磁装置中常利用软磁材料做铁芯}.
    \end{definition}
    \begin{theorem}
        (磁场的边界条件) 在磁介质的边界上无自由电流存在时,分界面两侧磁场强度的切向分量相等,磁感应强度的法向分量相等.
    \end{theorem}
    \begin{theorem}
        ($B$ 线的折射定律) $B$ 线通过磁介质分界面上的入射角 $\theta_1$ 和折射角 $\theta_2$ 满足 $\frac{\tan\theta_1}{\tan\theta_2}=\frac{\mu_{r 1}}{\mu _{r 2}}$.
    \end{theorem}
    \begin{definition}
        (永磁体) 仍保留一定磁化状态的铁磁体.
    \end{definition}
    \begin{definition}
        (漏磁通) 铁芯外部相对很弱的磁场.
    \end{definition}
    \begin{definition}
        (磁路) 由铁芯或一定的间隙构成的磁感线集中的通路.
    \end{definition}
    \begin{definition}
        (磁阻) $\frac{l}{\mu_0\mu_{r}S}$.
    \end{definition}
    在绕线气隙铁环磁导公式 $\Phi\left( \frac{l}{\mu_0\mu_{r}S}+\frac{\delta}{\mu_0S} \right) =NI$ 中, $\Phi$ 对应电流, $NI$ 称磁动势.磁通,磁阻和磁动势形式上服从串并联规律.
    \section{电磁感应}
    \begin{theorem}
        (法拉第电磁感应定律) $E=-\frac{\mathrm{d}\Phi}{\mathrm{d}t}$,方向遵循楞次定律.
    \end{theorem}
    \begin{definition}
        (全磁通) $\Psi=\sum_{}^{}\Phi$ 为穿过各匝线圈的磁通量总和,各匝线圈磁通量相等时 $\Psi=N\Phi$ 称磁链.
    \end{definition}
    \begin{definition}
        (动生电动势) 导体在恒定磁场中运动产生的感应电动势 $E=\int_{}^{}\left( \bm{v}\times \bm{B} \right) \mathrm{d}\bm{l}=Blv$,遵循右手定则.
    \end{definition}
    \begin{definition}
        (感生电场) 磁场变化产生的电场 $\bm{E}_{i}$.
    \end{definition}
    \begin{definition}
        (感生电动势) 磁场发生变化导致导体回路中产生的电动势 $E=\oint_{L}^{}\bm{E}_{i}\cdot \mathrm{d}\bm{l}=-\frac{\mathrm{d}\Phi}{\mathrm{d}t}=-\frac{\mathrm{d}}{\mathrm{d}t}\int_{S}^{}\bm{B}\cdot \mathrm{d}\bm{S}=-\int_{S}^{}\frac{\partial \bm{B}}{\partial t}\cdot \mathrm{d}\bm{S}$.
    \end{definition}
    \begin{definition}
        (互感) 闭合导体回路中电流随时间变化时引起周围磁场变化在附近的导体回路中产生感生电动势,称互感电动势.
    \end{definition}
    \begin{definition}
        (互感系数) $E_{12}=-M \frac{\mathrm{d}i_2}{\mathrm{d}t},E_{21}=-M \frac{\mathrm{d}i_1}{\mathrm{d}t}$,其中 $M$ 称互感系数,单位为亨利 $1\ \mathrm{H}=1 \ \mathrm{\Omega \cdot s}$.
    \end{definition}
    \begin{definition}
        (自感) 电流回路的电流发生变化时自身产生感生电动势,称自感电动势,方向总是阻碍本身电流变化.
    \end{definition}
    \begin{definition}
        (自感系数) $E=-\frac{\mathrm{d}\Phi}{\mathrm{d}t}=-L \frac{\mathrm{d}i}{\mathrm{d}t}$,其中 $L$ 称自感系数,单位也为亨利.
    \end{definition}
    \begin{theorem}
        (电感的通电与断电) 通电时 $E=L \frac{\mathrm{d}i}{\mathrm{d}t}+iR,i=\frac{E}{R}\left( 1-e^{-\frac{R}{L}t} \right) $,断电时 $L \frac{\mathrm{d}i}{\mathrm{d}t}+iR=0,i=\frac{E}{R}e^{-\frac{R}{L}t}$.
    \end{theorem}
    \begin{definition}
        (时间常数) $\tau=\frac{L}{R}$ 为变量变化倍率 $\frac{1}{e}$ 的时间.
    \end{definition}
    \begin{definition}
        (磁能) $W_{m}=\frac{1}{2}\int_{}^{}\bm{H}\cdot \bm{B}\mathrm{d}V$,积分应遍及整个磁场分布的空间, $w_{m}=\frac{1}{2}\bm{B}\cdot \bm{H}$ 称磁场能量密度.\mn{由于铁磁质的磁滞,磁能公式对铁磁质不适用}
    \end{definition}
    \section{麦克斯韦方程组和电磁辐射}
    \begin{theorem}
        (麦克斯韦方程组) 
        \begin{align}
            \oint_{S}^{}\bm{D}\cdot \mathrm{d}\bm{S}=&\frac{q}{\epsilon_0}=\frac{1}{\epsilon_0}\int_{}^{}\rho\mathrm{d}V\nonumber\\
            \oint_{S}^{}\bm{B}\cdot \mathrm{d}\bm{S}=&0\nonumber\\
            \oint_{L}^{}\bm{E}\cdot\mathrm{d}\bm{r}=&-\frac{\mathrm{d}\Phi}{\mathrm{d}t}=-\int_{S}^{}\frac{\partial \bm{B}}{\partial t}\cdot\mathrm{d}\bm{S}\nonumber\\
            \oint_{L}^{}\bm{H}\cdot\mathrm{d}\bm{r}=&\mu_0I+\frac{1}{c^2}\frac{\mathrm{d}\Phi_{e}}{\mathrm{d}t}=\mu_0\int_{S}^{}\left( \bm{J}+\epsilon_0 \frac{\partial \bm{D}}{\partial t} \right) \cdot\mathrm{d}\bm{S}\nonumber
        \end{align}
        
        微分形式:
        \begin{align}
            \nabla\cdot \bm{D}=&\rho\nonumber\\
            \nabla\cdot \bm{B}=&0\nonumber\\
            \nabla\times \bm{E}=&-\frac{\partial \bm{B}}{\partial t}\nonumber\\
            \nabla\times \bm{H}=&\bm{J}+\frac{\partial \bm{D}}{\partial t}\nonumber
        \end{align}
        对于各向同性的线性介质:
        \begin{align}
            D=&\epsilon_0\epsilon_{r}E\nonumber\\
            B=&\mu_0\mu_{r}\bm{H}\nonumber\\
            J=&\sigma \bm{E}\nonumber
        \end{align}
    \end{theorem}
    \begin{theorem}
        (电磁波) $\bm{B}=\frac{\bm{c}\times \bm{E}}{c^2}$.
    \end{theorem}
    \begin{definition}
        (能流密度) $\bm{S}=\frac{1}{\mu_0}\bm{E}\times \bm{B}$,又称坡印亭矢量,其时间平均值称电磁波的强度.
    \end{definition}
    对于简谐电磁波,可由振幅计算强度 $I=\frac{1}{2}c\epsilon_0 E^2_{m}=c\epsilon_0E_{rms}^2$.
    \begin{theorem}
        (震荡电偶极子的辐射功率) $P=\frac{p_0^2\omega^{4}}{12\pi\epsilon_0c^3}$,其中 $p_0=ql$ 为振幅, $\omega$ 为频率.
    \end{theorem}
    \begin{definition}
        (电磁波的动量密度) $p=\frac{w}{c}$,其中 $w$ 为单位体积电磁波具有的能量,也可写成 $\bm{p}=\frac{w}{c^2}\bm{c}$ 或 $p=\frac{\epsilon_0 E^2}{c}$.
    \end{definition}
    对绝对黑面的辐射压强 $p_{r}=cp=\epsilon_0E^2=w$,对于全反射面压强加倍.
    \section{常用二级结论}
    \begin{itemize}
        \item 电偶极子中垂线上的电场强度: $\bm{E}=\frac{-q \bm{l}}{4\pi\epsilon_0r^3}=\frac{-\bm{p}}{4\pi\epsilon_0r^3}$, 其中 $\bm{p}=q\bm{l}$ 称电矩,等于单极电量绝对值乘上负电荷指向正电荷的矢量 $\bm{l}$.\mn{直接使用点电荷电场强度叠加}
        \item 线电荷密度为 $\lambda$ 的带电直线中垂线上的电场强度 $E=\frac{\lambda L}{4\pi\epsilon_0 x(x^2+\frac{L^2}{4})^{\frac{1}{2}}}$. 当无限长或距离 $x$ 足够近时 $E\approx \frac{\lambda}{2\pi\epsilon_0 x}$,当足够远离时退化为点电荷.\mn{使用点电荷电场强度积分,无限长时使用高斯定理}
        \item 均匀带电细圆环轴线上的电场 $E=\frac{qx}{4\pi\epsilon_0(R^2+x^2)^{\frac{3}{2}}}$.\mn{点电荷电场强度积分}
        \item 面电荷密度为 $\sigma$ 的均匀带电圆面轴线上的电场 $E=\frac{\sigma}{2\epsilon_0}\left[1-\frac{x}{\sqrt{R^2+x^2}}\right]$,在 $R$ 无穷大时即 $\frac{\sigma}{2\epsilon_0}$.\mn{从圆环半径积分,无穷大时使用高斯定理}
        \item 电偶极子在均匀电场中受力矩 $\bm{M}=\bm{p}\times \bm{E}$,其中 $\bm{p}=q \bm{l}$ 为电矩.\mn{按照 $F=qE$ 计算受力}
        \item 均匀带电球面内部场强处处为零,外部场强与处在球心的等量点电荷相同.\mn{高斯定理}
        \item 体电荷密度为 $\rho$ 的均匀带电球体内部场强为 $\bm{E}=\frac{\rho}{3\epsilon_0}\bm{r}$ 或 $\bm{E}=\frac{q}{4\pi\epsilon_0 R^3}\bm{r}$,外部同点电荷.\mn{高斯定理}
        \item 静止点电荷电势 $\phi=\frac{q}{4\pi \epsilon_0r}$.\mn{从点电荷电场强度积分}
        \item 均匀带电球面内部电势 $\phi=\frac{q}{4\pi\epsilon_0R}$,外部电势同点电荷.\mn{分段积分}
        \item 无限长均匀带电直线,选取无穷远为电势零点会导致电势无穷大,选 $r_0$ 处为电势零点,则电势为 $\phi=\frac{\lambda}{2\pi\epsilon_0}\ln \frac{r_0}{r}$.或者 $\phi=-\frac{\lambda}{2\pi\epsilon_0}\ln r+C$.\mn{由高斯定理导出电场强度后积分}
        \item 电偶极子远处的电势 $\phi=\frac{\bm{p}\cdot \bm{r}}{4\pi\epsilon_0r^3}$ 或 $\frac{p\cos \theta}{4\pi\epsilon_0r^2}$,其中 $\bm{p}$ 为电矩, $\bm{r}$ 为从电偶极子中心到目标点的矢量.\mn{由单个点电荷的电势叠加}
        \item 电偶极子远处的电场 $\bm{E}=\frac{1}{4\pi\epsilon_0}\left[-\frac{\bm{p}}{r^3}+\frac{3\bm{p}\cdot \bm{r}}{r^{5}}\bm{r}\right]$.\mn{对电势求梯度}
        \item 均匀带电细圆环的电势 $\phi=\frac{q}{4\pi\epsilon_0(R^2+x^2)^{\frac{1}{2}}}$.在中心处 $\phi=\frac{q}{4\pi\epsilon_0R}$.\mn{点电荷电势积分}
        \item 均匀带电球体内部的电势 $\phi=\frac{q}{8\pi\epsilon_0R^3}(3R^2-r^2)$.\mn{从电场强度分段积分}
        \item 电偶极子在均匀外电场中的电势能 $W=-\bm{p}\cdot \bm{E}$.\mn{直接由 $W=q\phi$ 转化为矢量形式}
        \item 电介质中的带电金属球外 $\bm{D}=\frac{q}{4\pi r^2}\bm{e}_{r}$, $\bm{E}=\frac{\bm{D}}{\epsilon_0\epsilon_{r}}=\frac{q}{4\pi\epsilon_0\epsilon_{r}r^2}\bm{e}_{r}$. 表面束缚电荷 $q'=(\frac{1}{\epsilon_{r}}-1)q$.\mn{$\bm{D},\bm{E}$由定义可知,表面束缚电荷由叠加定理计算}
        \item 电介质中的平板 $\bm{D}=\sigma$.\mn{由定义即得}
        \item 圆柱形电容器的电容 $C=\frac{2\pi \epsilon_0\epsilon_{r}L}{\ln (\frac{R_2}{R_1})}$,其中 $L$ 为圆柱长度, $R_2$ 为外径, $R_1$ 为内径.\mn{由高斯定理导出电场强度后积分得到 $Q-U$ 关系式}
        \item 球形电容器的电容 $C=\frac{4\pi \epsilon_0\epsilon_{r}R_1R_2}{R_2-R_1}$.\mn{同上}
        \item 球形电容器的能量 $W=\frac{Q^2}{8\pi \epsilon_0\epsilon_{r}}(\frac{1}{R_1}-\frac{1}{R_2})$.\mn{$W=\frac{Q^2}{2C}$}
        \item 中间充满电阻材料的同轴金属圆筒间的电阻 $R=\frac{\rho}{2\pi l}\ln \frac{R_2}{R_1}$, 其中 $R_2$ 为外径, $R_1$ 为内径.\mn{由电阻公式以半径积分}
        \item 直线电流的磁感应强度 $B=\frac{\mu_0I}{4\pi r}\left( \cos \theta_1 -\cos \theta_2 \right) $,方向由右手螺旋定则确定.其中 $r$ 为到直线的距离, $\theta_1,\theta_2$ 分别是电流起点和终点到目标点的矢量与电流矢量之间的夹角.对于无限长直电流, $B=\frac{\mu_0I}{2\pi r}$.\mn{由毕奥-萨伐尔定理积分,无限长使用安培环路定理}
        \item 圆形导线轴线上的磁感应强度 $\bm{B}=\frac{\mu_0 \bm{m}}{2\pi r^3}=\frac{\mu_0 \bm{m}}{2\pi \left( R^2+x^2 \right) ^{\frac{3}{2}}}$,在中心处有 $B=\frac{\mu_0I}{2R}$,方向由右手螺旋定则.其中磁矩 $\bm{m}=IS \bm{e}_{n}$.\mn{电流微元积分}
        \item 磁矩为 $\bm{m}$ 的小线圈在较远距离 $\bm{r}$ 处产生的磁场为 $\bm{B}=\frac{\mu_0}{4\pi}\left( -\frac{\bm{m}}{r^3}+\frac{3 \bm{m}\cdot \bm{r}}{r^{5}}\bm{r} \right) $.
        \item 螺线管轴线上的磁感应强度 $B=\frac{\mu_0nI}{2}\left( \cos\theta_2-\cos\theta_1 \right) $,其中 $\theta_1,\theta_2$ 为到螺线管开口边缘的矢量与轴线的夹角,正负号由右手螺旋定则确定.对于无限长直螺线管内轴线, $B=\mu_0nI$,对于其一段开口中心, $B=\frac{1}{2}\mu_0nI$.\mn{由圆形线圈的磁感应强度积分}
        \item 无限长圆柱面的磁感应强度 $B=\frac{\mu_0I}{2\pi r}$ 与直线电流相同,内部磁场为零.\mn{安培环路定理}
        \item 螺绕环内部的磁感应强度 $B=\frac{\mu_0NI}{2\pi r}=\mu_0nI$,其中 $N$ 为匝数, $n$ 为单位长度的匝数,在环管半径比环半径小得多时可取 $r=R$ 为环半径.管外的磁场为零.\mn{安培环路定理}
        \item 在无限大均匀平面电流两侧磁场大小相等,方向相反且平行于平面,垂直于电流方向,遵循右手螺旋定则, $B=\frac{1}{2}\mu_0j$,其中 $j$ 为面电流密度.\mn{对称性,安培环路定理}
        \item 充电的圆平行板电容器间的磁场环绕分布,方向遵循右手螺旋定则,大小在圆内为 $B=\frac{\mu_0 rI_{c}}{2\pi R^2}$,圆外等于线电流 $B=\frac{\mu_0I_{c}}{2\pi r}$.\mn{由平板电场求位移电流,全电流安培环路定理}
        \item 均匀磁场中载流导线受力等于连接其两端的直导线段受力 $F=I \bm{l}\times \bm{B}$.\mn{线积分,记结论}
        \item 充满磁介质的无限长直螺线管内的磁场强度 $H=nI$,磁感应强度 $B=\mu \bm{H}=\mu_0\mu_{r}nI$,磁介质表面的束缚电流密度 $j'=\bm{M}\times \bm{e}_{n}=M=\left( \mu_{r}-1 \right) \bm{H}=\left( \mu_{r}-1 \right) nI$.\mn{$H$ 的环路定理以及定义,束缚电流的定义}
        \item 磁介质中的圆柱形长直载流导体周围 $H=\frac{I}{2\pi r},B=\frac{\mu_0\mu_{r}}{2\pi r}I$,内表面束缚电流密度 $j'=\frac{\mu_{r}-1}{2\pi R}I$,总束缚电流 $I'=j'\cdot 2\pi R=\left( \mu_{r}-1 \right) I$.\mn{$H$ 的环路定理,$H$ 与 $B$的互相转化,束缚电流的定义}
        \item 绕线气隙铁环中有 $\frac{Bl}{\mu_0\mu_{r}}+\frac{B\delta}{\mu_0}=NI$,其中 $l$ 为铁环长度, $\delta$ 为气隙长度, $N$ 为绕线匝数.\mn{$H$ 的环路定理与 $B$ 的互相转化}
        \item 圆形磁场变化时感生电场 $E_{i}=-\frac{r}{2}\frac{\mathrm{d}\overline{B}}{\mathrm{d}t}$,其中 $\overline{B}$ 为围绕面积上的平均磁感应强度.\mn{由定义即得,系数由面积除周长取负}
        \item 长直螺线管内圆环与其的互感系数为 $M=\pi r^2 \mu_0n$, $r$ 为圆环半径.\mn{由内部 $B=\mu_0nI$ 乘上面积得到}
        \item 充满磁介质的螺绕环的自感为 $L=\mu_0\mu_{r}n^2V$,其中 $n$ 为单位长度上的匝数, $V$ 为螺绕环内部体积 $2\pi RS$ (即 $L=2\pi\mu_0\mu_{r}Rn^2S$).\mn{由全磁通除以电流得到}
        \item 两同轴薄壁金属管构成的电流回路的自感为 $L=\frac{\mu_0}{2\pi}\ln \frac{R_2}{R_1}$.\mn{按半径求出全磁通,除以电流}
        \item 电感的磁能 $W=\frac{1}{2}LI^2$.\mn{背}
        \item 互感线圈的额外磁能 $W=MI_1I_2$.\mn{背}
        \item 充电平行板电容器边缘坡印亭矢量 $\bm{S}$ 指向电容器内部,且按坡印亭矢量计算进入电容器内部的总能量等于电容器内静电能量增加率.\mn{由定义推导}
    \end{itemize}
    \section{例题题型}
    \begin{itemize}
        \item 求电场强度,由电荷微元积分,由电势求梯度,高斯定理
        \item 求电势,由电场强度积分
        \item 求静电势能,由电势和电荷分布求积分
        \item 有导体时求电势,电荷,电场分布,由高斯定理,面电荷密度与电场强度的关系联立求解方程.注意利用导体内部电场为零的条件.必要时使用镜像法,唯一性定理结合边界条件求解.
        \item 求解有磁介质存在时的磁场,根据自由电流的分布和 $\bm{H}$ 的环路定理求出 $\bm{H}$ 的分布,再利用 $\bm{H}=\frac{\bm{B}}{\mu}$ 求出 $\bm{B}$ 的分布.
    \end{itemize}
\end{document}
