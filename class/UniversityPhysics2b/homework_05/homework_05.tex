\documentclass[12pt]{article}

\author{陶文华}

\usepackage{../../homeworks_preamble}
\title{大学物理-第五次作业}

\begin{document}
    \maketitle
    \section{22.2} 汞弧灯发出的光通过一滤光片后照射双缝干涉装置。已知缝间距 $d=0.60 \ \mathrm{mm}$,观察屏与双缝相距 $D=2.5 \ \mathrm{m}$,并测得相邻明纹间距离 $\Delta x=2.27 \ \mathrm{mm}$.试计算入射光的波长.并指出属于什么颜色.

    解:由 $\Delta x=\frac{D}{d}\lambda$ 解得入射光波长 $\lambda=544.8 \ \mathrm{nm}$,属于绿色.
    \section{22.3} 劳埃德镜干涉装置如图 22.33 所示,光源波长 $\lambda=7.2\times 10^{-7} \ \mathrm{m}$,试求镜的右边缘到第一条明纹的距离.

    \begin{figure}[htbp]
        \centering
        \includegraphics[width=0.7\linewidth]{./figures/homework_05_figure_01.png}
    \end{figure}

    解:该劳埃德镜相当于相距 $d=4 \ \mathrm{mm}$ ,屏距 $D=0.5 \ \mathrm{m}$ 的反向平行光源形成的双缝干涉,右边缘到第一条明纹的距离即 $\frac{1}{2}\Delta x=\frac{1}{2}\frac{D}{d}\lambda=4.5\times 10^{-5} \ \mathrm{m}$.
    \section{22.14} 制造半导体元件时,常常要精确测定硅片上二氧化硅薄膜的厚度,这时可把二氧化硅薄膜的一部分腐蚀掉,使其形成劈尖,利用等厚条纹测出其厚度.已知 $\mathrm{Si}$ 的折射率为 $3.42$ , $\mathrm{SiO_2}$ 的折射率为 $1.5$ ,入射光波长为 $589.3 \ \mathrm{nm}$ ,观察到 $7$ 条暗纹 (如图 22.36 所示).问 $\mathrm{SiO_2}$ 薄膜的厚度 $h$ 是多少?

    \begin{figure}[htbp]
        \centering
        \includegraphics[width=0.7\linewidth]{./figures/homework_05_figure_02.png}
    \end{figure}
    
    解: $7$ 条暗纹且即标记处处为第 $7$ 条明纹,对应的薄膜厚度有 $2nh+\frac{\lambda}{2}=7\lambda$,解得 $h=\frac{13\lambda}{4n}\approx 1.28 \ \mathrm{mm}$.
    \section{22.19} 在折射率 $n_1=1.52$ 的镜头表面涂有一层折射率 $n_2=1.38$ 的 $\mathrm{MgF_2}$ 增透膜,如果此膜适用于波长 $\lambda=550 \ \mathrm{nm}$ 的光,膜的厚度应是多少?

    解: 最小厚度为 $h=\frac{\lambda}{4n_2}\approx 99.6 \ \mathrm{nm}$.所有的可用厚度为 $\left( 2k+1 \right) \frac{\lambda}{4n_2}=\left( 99.6+199.3k \right) \ \mathrm{nm},k=0,1,2,\ldots$.
    \section{22.22} 用迈克耳孙干涉仪可以测量光的波长,某次测得可动反射镜移动距离 $\Delta L=0.3220 \ \mathrm{mm}$ 时,等倾条纹在中心处缩进 $1204$ 条条纹.试求所用光的波长.

    解:由等倾条纹随薄膜厚度的变化关系可知 $\Delta L=k \frac{\lambda}{2n}$,由此解得 $\lambda=\frac{2n\Delta L}{k}=534.9 \ \mathrm{nm}$.
    \section{22.23} 一种干涉仪可以用来测定气体在各种温度和压力下的折射率,其光路如图 22.37 所示.图中 $S$ 为光源, $L$ 为凸透镜, $G_1,G_2$ 为两块完全相同的玻璃板,彼此平行放置 $T_1,T_2$ 为两个等长度的玻璃管,长度均为 $d$.测量时,先将两管抽空,然后将待测气体徐徐充入一管中,在 $E$ 处观察干涉条纹的变化,即可测得该气体的折射率.某次测量时,将待测气体充入 $T_2$ 管中,从开始进气到到达标准状态的过程中,在 $E$ 处看到共移过 $98$ 条干涉条纹.若光源波长 $\lambda=589.3 \ \mathrm{nm}$,$d=20 \ \mathrm{cm}$,试求该气体在标准状态下的折射率.

    \begin{figure}[htbp]
        \centering
        \includegraphics[width=0.7\linewidth]{./figures/homework_05_figure_03.png}
    \end{figure}

    解:与迈克耳孙干涉仪同理有 $\Delta L=\left( n-1 \right) d=k \lambda$,于是解得 $n=\frac{k\lambda}{d}+1\approx 1.00029$.
\end{document}
