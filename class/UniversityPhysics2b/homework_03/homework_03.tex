\documentclass[12pt]{article}

\author{陶文华}

\usepackage{../../homeworks_preamble}

\title{大学物理-第三次作业}

\begin{document}
    \maketitle
    \begin{itemize}
        \item 16.4 一铜棒的横截面积为 $20mm\times 80mm$ ,长为 $2m$ ,两端的电势差为 $50mV$ .已知铜的电阻率为$\rho=1.75\times 10^{-8}\Omega \cdot m$ . 铜内自由电子的数密度为 $8.5 \times 10^{28} /m^{3}$ .求:
            \begin{enumerate}
                \item 棒的电阻
                    $R=\rho \frac{L}{S}=\rho \frac{L}{ab}=2.1875\times 10^{-5}\Omega$.
                \item 通过棒的电流 $I=\frac{U}{R}\approx 2.28\times 10^{3}A$.
                \item 棒内的电流密度 $J=\frac{I}{ab}\approx 1.43\times 10^{6} A / m^2$.
                \item 棒内的电场强度 $E=\frac{U}{L}=2.5\times 10^{-2}V / m$.
                \item 棒所消耗的功率 $P=UI\approx 1.14\times 10^{-2}W$.
                \item 棒内电子的漂移速度 $v=\frac{J}{\sigma e}\approx 1.05\times 10^{-4}m /s$.
            \end{enumerate}
        \item 16.8 
            \begin{figure}[htbp]
                \includegraphics[width=0.5\linewidth]{./figures/homework_03_figure_01.png}
            \end{figure}

            如图 16.20 所示,电缆的芯线是半径为 $r_1=0.5cm$ 的铜线,在铜线外面包一层同轴的绝缘层,绝缘层的外半径为 $r_2=2 cm$ ,电阻率 $\rho=1\times 10^{12}\Omega \cdot m$ .在绝缘层外面又用铅层保护起来.
            \begin{enumerate}
                \item 求长 $L=1000m$ 的这种电缆沿径向的电阻

                    由电阻的定义可知所求总电阻 $R=\int_{}^{} \mathrm{d} R=\int_{r_1}^{r_2}\rho \frac{\mathrm{d}r}{2\pi r L}=\frac{\rho}{2\pi L}\ln \frac{r_2}{r_1}\approx 2.2\times 10^{8}\ \mathrm{\Omega}$.
                \item 当芯线与铅层的电势差为 $100\ \mathrm{V}$ 时,在这电缆中沿径向的电流多大?

                    由欧姆定律 $I=\frac{U}{R}\approx 4.5\times 10^{-7}\ \mathrm{A}$.
            \end{enumerate}
        \item 17.6
            \begin{figure}[htbp]
                \includegraphics[width=0.5\linewidth]{./figures/homework_03_figure_02.png}
            \end{figure}

            如图 17.41 所示,求半圆形电流 $I$ 在半圆的轴线上离圆心距离 $x$ 处的 $\bm{B}$ .

        \item 17.18 将 $\bm{B}$ 分解为轴向和径向,在半圆上选取一电流源 $I\mathrm{d}l$,令其到 $P$ 的向量为 $r$, $r$ 与 $OP$ 的夹角为 $\theta$,其中轴向分量 $\mathrm{d}B_1=\mathrm{d}B\sin \theta=-\frac{\mu_0IR}{4\pi r^3}\mathrm{d}l$,径向分量 $\mathrm{d}B_2=\mathrm{d}B\cos \theta=-\frac{\mu_0Ix}{4\pi r^3}\mathrm{d}l$. 轴向分量积分可得 $B_1=-\frac{\mu_0IR^2}{4r^3}=-\frac{\mu_0IR^2}{4(R^2+x^2)^{3 / 2}}$, 径向分量再次分解为垂直直径的部分 $\mathrm{d}B_3=\mathrm{d}B_2\sin \phi$ 和 平行直径的部分 $\mathrm{d}B_4=\mathrm{d}B_2\cos \phi$,其中 $\phi$ 是 $O$ 与 $\mathrm{d}l$ 的连线与直径形成的夹角,由对称性可得 $\mathrm{d}B_4$ 在积分中相互抵消,径向分量的和磁场垂直于半圆直径,大小为 $\int_{0}^{\pi}\mathrm{d}B_2 R\sin \phi \mathrm{d}\phi=-\frac{\mu_0IRx}{2\pi r^3}=-\frac{\mu_0IRx}{2\pi(R^2+x^2)^{3 / 2}}$,负号表示其方向垂直于直径与 $OP$ 所在平面向下. 综上所述,令 $\bm{i}$ 为 $OP$ 方向的向量,  $\bm{j}$ 为垂直于直径与 $OP$ 所在平面向上的向量,则所求 $\bm{B}=-\frac{\mu_0IR^2}{4(R^2+x^2)^{3 / 2}}\bm{i}-\frac{\mu_0IRx}{2\pi(R^2+x^2)^{3 / 2}}\bm{j}$.
        \item 18.4 估算地求磁场对电视机显像管中电子束的影响.假设加速电势差为 $2.0\times 10^{4}\mathrm{V}$,如电子枪到屏的距离为 $0.2\mathrm{m}$,试计算电子束在大小为 $0.5\times 10^{-4}T$ 的横向地磁场作用下约偏转多少?假定没有其他偏转磁场,这偏转是否显著?

            在加速电势差下加速的电子速度 $v$ 满足 $\frac{1}{2}mv^2=eU$.在从电子枪发射到屏幕的过程中,在地磁场的作用下做匀速圆周运动,半径满足 $R=\frac{mv}{eB}=\frac{1}{B}\sqrt{\frac{2mU}{e}}\approx 9.54\mathrm{m}$,在飞行距离 $L=0.2\mathrm{m}$ 下偏转距离为 $R-\sqrt{R^2-L^2}\approx 2\mathrm{mm}$.在没有其他偏转磁场下,这偏转不算显著.
        \item 18.12 
            \begin{figure}[htbp]
                \includegraphics[width=0.5\linewidth]{./figures/homework_03_figure_03.png}
            \end{figure}
            
            如图 18.24 所示,一块半导体样品的体积为 $a \times b \times c$,沿 $x$ 方向有电流 $I$,在 $z$ 轴方向加有均匀磁场 $\bm{B}$.这时实验得出的数据$a=0.10\mathrm{cm},b=0.35\ \mathrm{cm},c=1.0\ \mathrm{cm},I=1.0\mathrm{mA},B=3000\ \mathrm{G}$,片两侧的电势差 $U_{AA^{'}}= 6.55\ \mathrm{mV}$
            \begin{enumerate}
                \item 这半导体是正电荷导电(P 型)还是负电荷导电(N 型)?

                    根据左手定则,载流子应集中在半导体的 $A^{'}$ 侧,又由电势差 $U_{AA^{'}}$ 为正, $A^{'}$ 侧电势较低,故载流子应为负电荷,半导体为 N 型.
                \item 求载流子浓度.

                    由 $U_{\mathrm{H}}=\frac{IB}{nqa}$ 可知 $n=\frac{IB}{U_{\mathrm{H}}qa}\approx 2.86\times 10^{20}\ \mathrm{/ m^3}$.
            \end{enumerate}
        \item 18.18 一正方形线圈由外皮绝缘的细导线绕成,共绕有 200 匝,每边长为 $150 \ \mathrm{mm}$,放在 $B = 4.0 \ \mathrm{T}$ 的外磁场中,当导线中通有 $I=8.0 \ \mathrm{A}$ 的电流时,求:
            \begin{enumerate}
                \item 线圈磁矩 $\bm{m}$ 的大小.

                    由磁矩的定义知 $m=nIS=nIa^2=36 \ \mathrm{A\cdot m^2}$.
                \item 作用在线圈上的力矩的最大值.

                    由 $\bm{M}=\bm{m}\times \bm{B}$ 可知当 $\bm{m}$ 与 $\bm{B}$ 垂直时力矩取得最大值 $mB=144 \ \mathrm{N\cdot m}$.
            \end{enumerate}
        \item 18.23 一无限长薄壁金属筒,沿轴线方向有均匀电流流通,面电流密度为 $j\ \mathrm{(A / m)}$.求单位面积筒壁受的磁力的大小和方向.

            由无限长圆柱面的对称性,其表面的磁场沿切向,大小为 $B=\frac{\mu_0I}{2\pi R}=\mu_0j$.电流方向垂直于磁场方向,对于单位面积的筒壁,其受磁力大小为 $F=\frac{1}{2}\mu_0j^2$,由右手螺旋定则以及左手定则可知磁力方向沿法向指向筒内.
        \item 18.26 
            \begin{figure}[htbp]
                \includegraphics[width=0.5\linewidth]{./figures/homework_03_figure_04.png}
            \end{figure}
            
            如图 18.31 所示,一半径为 $R$ 的无限长半圆柱面导体,其上电流与其轴线上一无限长直导线的电流等值反向,电流 $I$ 在半圆柱面上均匀分布.
            \begin{enumerate}
                \item 试求轴线上导线单位长度所受的力.

                    在半圆柱面上取条形电流微元 $jR\mathrm{d}\theta$,由对称性沿半圆直径方向的力互相抵消,合力垂直半圆直径所在平面,大小为 $F=\int_{}^{}\mathrm{d}F\sin\theta=\frac{\mu_0jI}{2\pi}\int_{0}^{\pi}\sin\theta\mathrm{d}\theta=\frac{\mu_0jI}{\pi}=\frac{\mu_0I^2}{\pi^2R}$.
                \item 若将另一无限长直导线(通有大小,方向与半圆柱面相同的电流 $I$) 代替圆柱面,产生同样的作用力,该导线应放在何处?
                    解 $\frac{\mu_0I^2}{2\pi r}=\frac{\mu_0I^2}{\pi^2R}$ 得到 $r=\frac{\pi R}{2}$.
            \end{enumerate}
    \end{itemize}
\end{document}
