\documentclass[12pt]{article}

\author{陶文华}

\usepackage{../../homeworks_preamble}
\title{大学物理-第六次作业}

\begin{document}
    \maketitle
    \section{23.1} 有一单缝,缝宽 $a=0.10 \ \mathrm{mm}$,在缝后放一焦距为 $50 \ \mathrm{cm}$ 的会聚透镜,用波长 $\lambda = 546.1 \ \mathrm{ nm}$ 的平行光垂直照射单缝,试求位于透镜焦平面处屏上中央明纹的宽度.

    解:由单缝衍射性质,中央明纹宽度 $d=2f\frac{\lambda}{a}=5.46 \ \mathrm{mm}$.
    \section{23.6} 有一种利用太阳能的设想是在 $3.5\times 10^{4} \ \mathrm{km}$ 的高空放置一块大的太阳能电池板,把它收集到的太阳能用微波形式传回地球.设所用微波波长为 $10 \ \mathrm{cm}$ ,而发射微波的抛物天线的直径为 $1.5 \ \mathrm{ km}$.此天线发射的微波的中央波束的角宽度是多少?在地球表面它所覆盖的面积的直径多大?

    解:由圆孔衍射规律和瑞利判据可得最小分辨角即中央波束角宽度的一半,故中央波束 $\theta=2\delta \theta=2\times 1.22 \frac{\lambda}{D}\approx 1.63\times 10^{-4} \ \mathrm{ rad}$. 与此对应的面积直径 $d=D+h\theta\approx 7.19 \ \mathrm{km}$.
    \section{23.8} 据说间谍卫星上的照相机能清楚识别地面上汽车的牌照号码.
    \begin{enumerate}
        \item 如果需要识别的牌照上的字划间的距离为 $5 \ \mathrm{ cm}$,在 $160 \ \mathrm{ km}$ 高空的卫星上的照相机的角分辨率应多大?

            解:由角分辨率的定义知其等于 $\delta\theta=\frac{d}{l}=3.125\times 10^{-7} \ \mathrm{rad}$.
        \item 此照相机的孔径需要多大?光的波长按 $500 \ \mathrm{nm}$ 计.

            解:由最小分辨角的定义可知 $\delta\theta=1.22 \frac{\lambda}{D}$,代入数据解得 $D=1.952 \ \mathrm{m}$.
    \end{enumerate}
    \section{23.12} 一双缝,缝间距 $d=0.10 \ \mathrm{mm}$,缝宽 $a=0.02 \ \mathrm{mm}$,用波长 $\lambda=480 \ \mathrm{nm}$ 的平行单色光垂直入射该双缝,双缝后放一焦距为 $50 \ \mathrm{cm}$ 的透镜,试求:
    \begin{enumerate}
        \item 透镜焦平面处屏上干涉条纹的间距.

            解:由光栅远场衍射的性质,干涉条纹间距即 $D_1=f \frac{\lambda}{d}= 2.4 \ \mathrm{mm}$.
        \item 单缝衍射中央亮纹的宽度.

            解:由单缝远场衍射的性质,中央亮纹宽度 $D_2=2f \frac{\lambda}{a}=2.4 \ \mathrm{cm}$.
        \item 单缝衍射的中央包线内有多少条干涉的主极大.

            解:由光栅衍射的性质,主极大数量即 $\frac{D_2}{D_1}-1=9$.
    \end{enumerate}
    \section{23.17} 北京天文台的米波综合孔径射电望远镜由没置在东西方向上的一列共 28 个抛物面组成.这些天线用等长的电缆连到同一个接收器上(这样各电缆对各天线接收的电磁波信号不会产生附加的相差),接收由空间射电源发射的 $232 \ \mathrm{MHz}$ 的电磁波.工作时各天线的作用等效于间距为 $6 \ \mathrm{m}$,总数为 192 个天线的一维天线阵列.接收器接收到的从正天顶上的一颗射电源发来的电磁波将产生极大强度还是极小强度?在正天顶东方多大角度的射电源发来的电磁波将产生第一级极小强度?又在正天顶东方多大角度的射电源发来的电磁波将产生下一级极大强度?

    解:正天顶对应光栅衍射的主极大,故产生极大强度.

    由光栅衍射的性质,第一级极小的角距离 $\delta\theta_1=\frac{\lambda}{Nd}=\frac{c}{Nfd}\approx 1.12\times 10^{-3} \ \mathrm{rad}\approx 3.85'$.

    第二级极大的角距离 $\delta\theta_2=\frac{\lambda}{d}=\frac{c}{Nd}\approx 0.216 \ \mathrm{rad}\approx 12.3^{\circ}$
    \section{23.18} 若 X 射线衍射中入射角 $\phi=45^{\circ}$,入射的 X 射线包含有从 $0.095\sim{}0.130 \ \mathrm{nm}$ 这一波带中的各种波长.已知晶格常数 $d=0.275 \ \mathrm{nm}$,问是否会有干涉加强的衍射 X 射线产生?如果有,这种射线的波长如何?

    解:干涉加强的条件为 $2d\sin\phi=k\lambda$,代入数据解得 $\lambda\approx \frac{0.389}{k} \ \mathrm{nm},k=1,2,3,\ldots$,计算可得 $k=3,4$ 时符合要求,故有干涉加强的衍射 X 射线产生,波长分别为 $0.130 \ \mathrm{nm},0.097 \ \mathrm{nm}$.
    \section{23.19} 1927 年戴维孙和革末用电子束射到镍晶体上的衍射(散射)实验证实了电子的波动性.实验中电子束垂直入射到晶面上.他们在 $\phi=50^{\circ}$ 的方向测得了衍射电子流的极大强度.已知晶面上原子间距为 $0.215 \ \mathrm{nm}$,求与入射电子束相应的电子波波长.

    解:由晶体散射极大强度条件 $2d\sin \phi=k\lambda$ 代入数据可得 $\lambda\approx \frac{0.329}{k} \ \mathrm{nm},k=1,2,3,\ldots$,代入电子波条件可知 $k=2$, $\lambda\approx 0.165 \ \mathrm{ nm}$.
    \section{24.1} 自然光通过两个偏振化方向间成 $60^{\circ}$ 的偏振片,透射光强为 $I_1$.今在这两个偏振片之间再插入另一偏振片,它的偏振化方向与前两个偏振片均成 $30^{\circ}$,则透射光强为多少?

    解:由马吕斯定律,设第一个偏振片的出射光强为 $I_0$,有 $I_1=I_0\cos ^2\alpha=\frac{1}{4}I_0$,三个偏振片时的透射光强 $I_2=I_0\cos ^{4}\beta=\frac{9}{16}I_0=\frac{9}{4}I_1=2.25I_1$.
    \section{24.2} 自然光入射到两个互相重叠的偏振片上.如果透射光强为 (1) 透射光最大强度的三分之一,或 (2) 入射光强度的三分之一.则这两个偏振片的偏振化方向间的夹角是多少?

    解:
    \begin{enumerate}
        \item 由马吕斯定律,透射光最大强度即两偏振片方向平行,此时透射光强等于第一个偏振片的透射光强,而不平行时有 $\cos ^2\alpha=\frac{1}{3}$,求解得到 $\alpha\approx 54^{\circ}44'$.
        \item 自然光相当于各向同性的线偏振光叠加,由马吕斯定律可知其通过第一个偏振片后的强度 $I=\frac{I_0}{\pi}\int_{0}^{\pi}\cos ^2\alpha=\frac{1}{2}I_0$,由此可知再通过第二个偏振片造成的光强减小比例为 $\frac{2}{3}=\cos ^2\beta$,解得 $\beta\approx 35^{\circ}16'$.
    \end{enumerate}
    \section{24.4} 在图 24.39 所示的各种情况中,以非偏振光和偏振光入射于两种介质的分界面,图中 $i_{b}$ 为起偏振角, $i\ne i_{b}$,试画出折射光线和反射光线并用点和短线表示出它们的偏振状态.

    解:如图所示:
    \begin{figure}[htbp]
        \centering
        \inkfig[0.7\columnwidth]{homework_06_figure_01}
    \end{figure}
    \section{24.5} 水的折射率为 $1.33$,玻璃的折射率为 $1.50$,当光由水中射向玻璃而反射时,起偏振角为多少?当光由玻璃中射向水而反射时,起偏振角又为多少?这两个起偏振角的数值间是什么关系?

    解:由起偏振角的定义,在水射向玻璃反射时起偏振角为 $\arctan n_{21}\approx 48^{\circ}26'$,同理玻璃射向水反射 $i_{b}=\arctan n_{12}\approx 41^{\circ}34'$,这两个角互余,这从它们的正切互为倒数也可以看出.
    \section{24.8} 已知从一池静水的表面反射出来的太阳光是线偏振光,此时,太阳在地平线上多大仰角处?

    解:此时入射角为起偏振角 $\arctan n_{21}\approx 53^{\circ}44'$,故太阳仰角是其余角 $36^{\circ}16'$.
    \section{24.9} 用方解石切割成一个正三角形棱镜.光轴垂直于棱镜的正三角形截面,如图所示.自然光以入射角入射时, $e$ 光在棱镜内的折射线与棱镜底边平行,求入射角并画出 $o$ 光的传播方向和光矢量振动方向.

    解:由方解石 $n_{o}=1.6584,n_{e}=1.4864$,由图可知 $e$ 光的折射角为 $30^{\circ}$,由此可得入射角 $i=\arcsin\left( n_{e}\sin 30^{\circ} \right) \approx 48^{\circ}$.同理可以计算出 $o$ 光的折射角,绘图如下:
    \begin{figure}[htbp]
        \centering
        \inkfig[0.5\columnwidth]{homework_06_figure_02}
    \end{figure}
    \section{24.10} 棱镜 $ABCD$ 由两个 $45^{\circ}$ 的方解石棱镜组成 (如图所示),棱镜 $ABD$ 的光轴平行于 $AB$,棱镜 $BCD$ 的光轴垂直于图面.当自然光垂直于 $AB$ 入射时,试在图中画出$o$ 光和 $e$ 光的传播方向及光矢量振动方向.

    解:如图所示:
    \begin{figure}[htbp]
        \centering
        \inkfig[0.7\columnwidth]{homework_06_figure_03}
    \end{figure}
    \section{24.11} 在图所示的装置中, $P_1,P_2$ 为两个正交偏振片. $C$ 为四分之一波片,其光轴与的偏振化方向间夹角为 $60^{\circ}$,光强为 $I_{i}$ 的单色自然光垂直入射于 $P_1$
    \begin{enumerate}
        \item 试说明 $1,2,3$ 各区光的偏振状态并在图上大致画出

            解:通过 $P_1$ 后为平行于其起偏方向的线偏振光,通过四分之一波片后为椭圆偏振光,再通过 $P_2$ 后为平行于其起偏方向的线偏振光.大体如图所示:
            \begin{figure}[htbp]
                \centering
                \includegraphics[width=0.7\linewidth]{./figures/homework_06_figure_04.pdf}
            \end{figure}
        \item 计算各区光强

            由之前的推论可知,经过 $P_1$ 后的光强 $I_1=\frac{1}{2}I_{i}$,经过波片后光强不变 $I_2=I_1=\frac{1}{2}I_{i}$,经过 $P_2$ 后对椭圆偏振光分解为两个方向的线偏振光,使用马吕斯定律可知光强 $I_3=\frac{3I_{i}}{16}$.
    \end{enumerate}
    \section{24.13} 假设石英的主折射率和与波长无关.某块石英晶片,对 $800 \ \mathrm{nm}$ 波长的光是四分之一波片.当波长为 $400 \ \mathrm{nm}$ 的线偏振光入射到该晶片上,且其光矢量振动方向与晶片光轴成 $45^{\circ}$ 时,透射光的偏振状态是怎样的?

    解:波长减半,四分之一波片变为二分之一波片,入射光被分为两个相位差为 $\pi$ 的相互垂直线偏振光,故出射光仍然是线偏振光,但振动方向与原线偏振光垂直.
    \section{24.15} 石英对波长为 $396.8 \ \mathrm{nm}$ 的光的右旋圆偏振光的折射率为 $n_{R}=1.55810$,左旋圆偏振光的折肘率为 $n_{L}=1.55821$.求石英对此波长的光的旋光率.

    解: $\alpha=\frac{\pi\left( n_{L}-n_{R} \right) }{\lambda}\approx 49.9 \ \mathrm{^{circ}/mm}$.
    \section{24.16} 在激光冷却技术中,用到一种"偏振梯度效应”",它是使强度和频率都相同但偏振方向相互垂直的两束激光相向传播,从而能在叠加区域周期性地产生各种不同偏振态的光.设两束光分别沿 $+x$ 和 $-x$ 方向传播,光振动方向分别沿 $y$ 方向和 $z$ 方向.已知在 $x=0$ 处的合成偏振态为线偏振态,光振动方向与 $y$ 轴成 $45^{\circ}$.试说明沿 $+x$ 方向每经过 $\frac{\lambda}{8}$ 的距离处的偏振态,并画简图表示之.

    解:原点处两光同相,经过 $\frac{\lambda}{8}$ 后 $+x$ 方向的光相位落后 $\frac{\pi}{4}$, $-x$ 方向的光相位超前 $\frac{\pi}{4}$,故相位差 $\frac{\pi}{2}$,合成从 $+x$ 轴看的右旋圆偏振光,同理 $\frac{\lambda}{4}$ 处合成与原方向垂直的线偏振光, $\frac{3\lambda}{8}$ 处为左旋圆偏振光, $\frac{\lambda}{2}$ 处与原点相同,以此循环,简图如下:
    \begin{figure}[htbp]
        \centering
        \inkfig[0.7\columnwidth]{homework_06_figure_05}
    \end{figure}
\end{document}
