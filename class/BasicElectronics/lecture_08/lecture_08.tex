\input{../../lectures_preamble.tex}
\usepackage{../../lectures_preamble}

\begin{document}
    \subsection{复阻抗与复导纳}
        \begin{enumerate}
            \item 纯电阻仅有电阻 $Z=R$ ,电导 $Y=\frac{1}{R}$
            \item 纯电容仅有电抗 $Z=\frac{1}{j\omega C}=-\frac{j}{\omega C}$ ,电纳 $Y=j\omega C$
             \item 纯电感仅有电抗 $Z=j\omega L$ ,电纳 $Y=\frac{1}{j\omega L}=-\frac{j}{\omega L}$
        \end{enumerate}\sn{电容和电感的电抗方向相反}

        对于电阻,电容,电感串联, 
        \begin{align}
            Z=R+j(\omega L-\frac{1}{\omega C})\nonumber
        \end{align}\sn{相当于复阻抗求和}

        其复电压 $U=\sqrt{U_{R}^2+U_{X}^2}$,选电流为参考相量, $U,U_{R},U_{X}$ 形成一个直角三角形,称电压三角形. $U$ 辐角为正, $\omega L > \frac{1}{\omega C}$,电压领先电流,称电路呈感性,反之称容性,同相时称电阻性\sn{性质与 $\omega$ 有关,仅在当前电源频率下电阻性与电阻等效}.

        对于电阻,电容,电感并联: 
        \begin{align}
            Y=\frac{1}{R}+j(\omega C -\frac{1}{\omega L})\nonumber
        \end{align}\sn{相当于复导纳求和}

        同理可以绘制电流三角形.容性时 $\omega C> \frac{1}{\omega L}$,感性,电阻性同理.
        \begin{remark}
            (阻抗与导纳的互换) 电路的阻抗和导纳在复数意义下互为倒数,一般情况下电阻和电导,电抗和电纳并非倒数关系.
        \end{remark}
        \begin{remark}
            (相量与复数) 上附点 $\dot\ $ 表示相量,用于 $U,I$.阻抗 $Z$ 不加上附点,表示一个复数. $U,I$ 不加上附点时表示相量模长.
        \end{remark}
        
        \subsection{电路定律的相量形式与电路相量模型}
            电路元件的相量关系:
            \begin{align}
                u=Ri \Leftrightarrow& \dot{U}=R\dot{I}\nonumber\\
                u=L \frac{\mathrm{d}i}{\mathrm{d}t} \Leftrightarrow& \dot{U}=j\omega L\dot{I}\nonumber\\
                u=\frac{1}{C} \int_{}^{}i\mathrm{d}t \Leftrightarrow& \dot{U}=\frac{1}{j\omega C}\dot{I}\nonumber
            \end{align}
            相量图的绘制:
            \begin{itemize}
                \item 需要同频率正弦量
                \item 以 $\omega$ 角速度逆时针旋转
                \item 选定一个参考相量,初相位为零
            \end{itemize}
            求解交流电路的正弦稳态解:
            \begin{itemize}
                \item 求解电流,电压关系等,给出相量形式时,与直流相同方法求解即可
                \item 电流电压仅给出模长时,绘制相量图,求解角度和边长关系,以此推出相量解.
            \end{itemize}
            \begin{example}
                (移相桥电路) 将电桥电路一个支路上的一个电阻换成电容,另一个电阻可变,另一个支路上两电阻相等,电桥开路监测电压,求解可变电阻由 $0\rightarrow \infty$ 时电桥电压的变化.
            \newline\newline
                绘制向量图,选取不变量总电压为参考相量 $\dot{U}$,双电阻支路有:
                \begin{align}
                    \dot{U}_{1}=\dot{U}_{2}=\frac{\dot{U}}{2}\nonumber
                \end{align}
                .电阻-电容支路有:
                \begin{align}
                    \dot{U}_{R}\perp& \dot{U}_{C}\nonumber\\
                    \dot{U}_{R}+\dot{U}_{C}=&\dot{U}\nonumber
                \end{align}
                $\dot{U}_{R},\dot{U}_{C}$ 的交点在以 $\dot{U}$ 为直径的半圆上运动,电桥电压 $\dot{U}_{ab}$ 由 $\dot{U}_{1}$ 指向 $\dot{U}_{R}$ 为圆半径保持不变.
            \end{example}
\end{document}
