\input{../../lectures_preamble.tex}
\usepackage{../../lectures_preamble}

\begin{document}
    \section{Tauri 的使用}
        \subsection{配置和依赖处理}
        \subsection{TypeScript 的使用}
        \subsubsection{TS 到 JS 的编译}
        如果通过网络分发 TS 文件, \texttt{.ts} 后缀名可能被服务器识别为视频文件从而返回错误的响应头,为此可以使用 \texttt{tsc} 将其编译为 JS 文件:

        \texttt{tsc ---module es2015 ---target es5 yourfile.ts}

        指定模块为 \texttt{es2015} 即 ES6,目标为 \texttt{es5} 以适配旧版浏览器.
        \subsection{Rust 的使用}
        \subsection{生成与导出}
            支持生成 \texttt{.msi} 和 \texttt{.exe} 格式的 Windows 安装包,也可以生成 Android apk 包.

            \begin{enumerate}
                \item 构建发布版本: \texttt{cargo tauri build}
                \item 构建 Android apk 包: \texttt{cargo tauri android build --apk}

                    构建的包是没有签名的版本,需要生成电子签名后才能安装.

                    \texttt{jarsigner -keystore [.jks] -signedjar [signed.apk] [unsigned.apk] [keyname]}

                    签名后需要进行对齐:

                    \texttt{zipalign -p 4 [signed.apk] [aligned.apk]}

                    其中 \texttt{zipalign} 是 Android SDK 的一个组件,位于类似
                    
                    \texttt{C:/Users/[username]/AppData/Local/Android/Sdk/build-tools/[version]}

                    的目录下.
            \end{enumerate}
    \section{实际项目工程}
        \subsection{创建项目}
            使用 \texttt{create-tauri-app} 按照官方模板创建:
        \texttt{\\cargo install create-tauri-app ---locked\\cargo create-tauri-app .\\选择您的前端语言:TypeScript / JavaScript\\
选择您的包管理器:npm\\
选择您的UI模板:Vanilla\\
选择您的UI风格:TypeScript}

            注意 \texttt{cargo create-tauri-app .} 指定在当前目录下创建工程,否则会询问项目名并创建一个以其命名的新文件夹.
        \subsection{前端基础配置}
            在项目文件夹下的 \texttt{index.html} 内进行基础设置:

            中文页面: \texttt{<html lang="zh">}

            UTF-8 编码: \texttt{<meta charset="UTF-8"/>}

            内容安全策略 ( Content Security Policy, CSP ) 配置:

            \texttt{\\<meta http-equiv="Content-Security-Policy"\\
            content="default-src 'self';\\
            img-src data:;\\
            connect-src 'self' http://ipc.localhost"\\}

            可以类似设置 \texttt{script-src}, \texttt{font-src} 等内容源.

            导入模块: \texttt{<script type="module" src="/src/main.ts" defer></script>}
\end{document}
