\documentclass[12pt]{article}

\author{陶文华}

\usepackage{../../lectures_preamble}

\begin{document}
    \section{共产党宣言}
    一个幽灵,共产主义的幽灵,在欧洲游荡。为了对这个幽灵进行神圣的围剿,旧欧洲的一切势力,教皇和沙皇、梅特涅和基佐、法国的激进派和德国的警察,都联合起来了。

    有哪一个反对党不被它的当政的敌人骂为共产党呢?又有哪一个反对党不拿共产主义这个罪名去回敬更进步的反对党人和自己的反动敌人呢?

    从这一事实中可以得出两个结论:

    共产主义已经被欧洲的一切势力公认为一种势力;

    现在是共产党人向全世界公开说明自己的观点、自己的目的、自己的意图并且拿党自己的宣言来反驳关于共产主义幽灵的神话的时候了。

    为了这个目的,各国共产党人集会于伦敦,拟定了如下的宣言,用英文、法文、德文、意大利文、弗拉芒文和丹麦文公布于世。
    \subsection{资产者和无产者}
    至今一切社会的历史都是阶级斗争的历史。

    自由民和奴隶、贵族和平民、领主和农奴、行会师傅和帮工,一句话,压迫者和被压迫者,始终处于相互对立的地位,进行不断的、有时隐蔽有时公开的斗争,而每一次斗争的结局都是整个社会受到革命改造或者斗争的各阶级同归于尽。

    在过去的各个历史时代,我们几乎到处都可以看到社会完全划分为各个不同的等级,看到社会地位分成多种多样的层次。在古罗马,有贵族、骑士、平民、奴隶,在中世纪,有封建主、臣仆、行会师傅、帮工、农奴,而且几乎在每一个阶级内部又有一些特殊的阶层。

    从封建社会的灭亡中产生出来的现代资产阶级社会并没有消灭阶级对立。它只是用新的阶级、新的压迫条件、新的斗争形式代替了旧的。

    但是,我们的时代,资产阶级时代,却有一个特点:它使阶级对立简单化了。整个社会日益分裂为两大敌对的阵营,分裂为两大相互直接对立的阶级:资产阶级和无产阶级。

    从中世纪的农奴中产生了初期城市的城关市民;从这个市民等级中发展出最初的资产阶级分子。

    美洲的发现、绕过非洲的航行,给新兴的资产阶级开辟了新天地。东印度和中国的市场、美洲的殖民化、对殖民地的贸易、交换手段和一般商品的增加,使商业、航海业和工业空前高涨,因而使正在崩溃的封建社会内部的革命因素迅速发展。

    以前那种封建的或行会的工业经营方式已经不能满足随着新市场的出现而增加的需求了。工场手工业代替了这种经营方式。行会师傅被工业的中间等级排挤掉了;各种行业组织之间的分工随着各个作坊内部的分工的出现而消失了。

    但是,市场总是在扩大,需求总是在增加。甚至工场手工业也不再能满足需要了。于是,蒸汽和机器引起了工业生产的革命。现代大工业代替了工场手工业;工业中的百万富翁,一支一支产业大军的首领,现代资产者,代替了工业的中间等级。

    大工业建立了由美洲的发现所准备好的世界市场。世界市场使商业、航海业和陆路交通得到了巨大的发展。这种发展又反过来促进了工业的扩展。同时,随着工业、商业、航海业和铁路的扩展,资产阶级也在同一程度上得到发展,增加自己的资本,把中世纪遗留下来的一切阶级排挤到后面去。

    由此可见,现代资产阶级本身是一个长期发展过程的产物,是生产方式和交换方式的一系列变革的产物。

    资产阶级的这种发展的每一个阶段,都伴随着相应的政治上的进展。它在封建主统治下是被压迫的等级,在公社里是武装的和自治的团体,在一些地方组成独立的城市共和国,在另一些地方组成君主国中的纳税的第三等级;后来,在工场手工业时期,它是等级君主国或专制君主国中同贵族抗衡的势力,而且是大君主国的主要基础;最后,从大工业和世界市场建立的时候起,它在现代的代议制国家里夺得了独占的政治统治。现代的国家政权不过是管理整个资产阶级的共同事务的委员会罢了。

    资产阶级在历史上曾经起过非常革命的作用。

    资产阶级在它已经取得了统治的地方把一切封建的、宗法的和田园般的关系都破坏了。它无情地斩断了把人们束缚于天然尊长的形形色色的封建羁绊,它使人和人之间除了赤裸裸的利害关系,除了冷酷无情的“现金交易”,就再也没有任何别的联系了。它把宗教虔诚、骑士热忱、小市民伤感这些情感的神圣发作,淹没在利己主义打算的冰水之中。它把人的尊严变成了交换价值,用一种没有良心的贸易自由代替了无数特许的和自力挣得的自由。总而言之,它用公开的、无耻的、直接的、露骨的剥削代替了由宗教幻想和政治幻想掩盖着的剥削。

    资产阶级抹去了一切向来受人尊崇和令人敬畏的职业的神圣光环。它把医生、律师、教士、诗人和学者变成了它出钱招雇的雇佣劳动者。

    资产阶级撕下了罩在家庭关系上的温情脉脉的面纱,把这种关系变成了纯粹的金钱关系。

    资产阶级揭示了,在中世纪深受反动派称许的那种人力的野蛮使用,是以极端怠惰作为相应补充的。它第一个证明了,人的活动能够取得什么样的成就。它创造了完全不同于埃及金字塔、罗马水道和哥特式教堂的奇迹;它完成了完全不同于民族大迁徙和十字军征讨的远征。

    资产阶级除非对生产工具,从而对生产关系,从而对全部社会关系不断地进行革命,否则就不能生存下去。反之,原封不动地保持旧的生产方式,却是过去的一切工业阶级生存的首要条件。生产的不断变革,一切社会状况不停的动荡,永远的不安定和变动,这就是资产阶级时代不同于过去一切时代的地方。一切固定的僵化的关系以及与之相适应的素被尊崇的观念和见解都被消除了,一切新形成的关系等不到固定下来就陈旧了。一切等级的和固定的东西都烟消云散了,一切神圣的东西都被亵渎了。人们终于不得不用冷静的眼光来看他们的生活地位、他们的相互关系。

    不断扩大产品销路的需要,驱使资产阶级奔走于全球各地。它必须到处落户,到处开发,到处建立联系。

    资产阶级,由于开拓了世界市场,使一切国家的生产和消费都成为世界性的了。使反动派大为惋惜的是,资产阶级挖掉了工业脚下的民族基础。古老的民族工业被消灭了,并且每天都还在被消灭。它们被新的工业排挤掉了,新的工业的建立已经成为一切文明民族的生命攸关的问题;这些工业所加工的,已经不是本地的原料,而是来自极其遥远的地区的原料;它们的产品不仅供本国消费,而且同时供世界各地消费。旧的、靠本国产品来满足的需要,被新的、要靠极其遥远的国家和地带的产品来满足的需要所代替了。过去那种地方的和民族的自给自足和闭关自守状态,被各民族的各方面的互相往来和各方面的互相依赖所代替了。物质的生产是如此,精神的生产也是如此。各民族的精神产品成了公共的财产。民族的片面性和局限性日益成为不可能,于是由许多种民族的和地方的文学形成了一种世界的文学。

    资产阶级,由于一切生产工具的迅速改进,由于交通的极其便利,把一切民族甚至最野蛮的民族都卷到文明中来了。它的商品的低廉价格,是它用来摧毁一切万里长城、征服野蛮人最顽强的仇外心理的重炮。它迫使一切民族——如果它们不想灭亡的话——采用资产阶级的生产方式;它迫使它们在自己那里推行所谓的文明,即变成资产者。一句话,它按照自己的面貌为自己创造出一个世界。

    资产阶级使农村屈服于城市的统治。它创立了巨大的城市,使城市人口比农村人口大大增加起来,因而使很大一部分居民脱离了农村生活的愚昧状态。正像它使农村从属于城市一样,它使未开化和半开化的国家从属于文明的国家,使农民的民族从属于资产阶级的民族,使东方从属于西方。

    资产阶级日甚一日地消灭生产资料、财产和人口的分散状态。它使人口密集起来,使生产资料集中起来,使财产聚集在少数人的手里。由此必然产生的结果就是政治的集中。各自独立的、几乎只有同盟关系的、各有不同利益、不同法律、不同政府、不同关税的各个地区,现在已经结合为一个拥有统一的政府、统一的法律、统一的民族阶级利益和统一的关税的统一的民族。

    资产阶级在它的不到一百年的阶级统治中所创造的生产力,比过去一切世代创造的全部生产力还要多,还要大。自然力的征服,机器的采用,化学在工业和农业中的应用,轮船的行驶,铁路的通行,电报的使用,整个整个大陆的开垦,河川的通航,仿佛用法术从地下呼唤出来的大量人口,——过去哪一个世纪料想到在社会劳动里蕴藏有这样的生产力呢?

    由此可见,资产阶级赖以形成的生产资料和交换手段,是在封建社会里造成的。在这些生产资料和交换手段发展的一定阶段上,封建社会的生产和交换在其中进行的关系,封建的农业和工场手工业组织,一句话,封建的所有制关系,就不再适应已经发展的生产力了。这种关系已经在阻碍生产而不是促进生产了。它变成了束缚生产的桎梏。它必须被炸毁,它已经被炸毁了。

    起而代之的是自由竞争以及与自由竞争相适应的社会制度和政治制度、资产阶级的经济统治和政治统治。

    现在,我们眼前又进行着类似的运动。资产阶级的生产关系和交换关系,资产阶级的所有制关系,这个曾经仿佛用法术创造了如此庞大的生产资料和交换手段的现代资产阶级社会,现在像一个魔法师一样不能再支配自己用法术呼唤出来的魔鬼了。几十年来的工业和商业的历史,只不过是现代生产力反抗现代生产关系、反抗作为资产阶级及其统治的存在条件的所有制关系的历史。只要指出在周期性的重复中越来越危及整个资产阶级社会生存的商业危机就够了。在商业危机期间,总是不仅有很大一部分制成的产品被毁灭掉,而且有很大一部分已经造成的生产力被毁灭掉。在危机期间,发生一种在过去一切时代看来都好像是荒唐现象的社会瘟疫,即生产过剩的瘟疫。社会突然发现自己回到了一时的野蛮状态;仿佛是一次饥荒、一场普遍的毁灭性战争,使社会失去了全部生活资料;仿佛是工业和商业全被毁灭了,——这是什么缘故呢?因为社会上文明过度,生活资料太多,工业和商业太发达。社会所拥有的生产力已经不能再促进资产阶级文明和资产阶级所有制关系的发展;相反,生产力已经强大到这种关系所不能适应的地步,它已经受到这种关系的阻碍;而它一着手克服这种障碍,就使整个资产阶级社会陷入混乱,就使资产阶级所有制的存在受到威胁。资产阶级的关系已经太狭窄了,再容纳不了它本身所造成的财富了。——资产阶级用什么办法来克服这种危机呢?一方面不得不消灭大量生产力,另一方面夺取新的市场,更加彻底地利用旧的市场。这究竟是怎样的一种办法呢?这不过是资产阶级准备更全面更猛烈的危机的办法,不过是使防止危机的手段越来越少的办法。

    资产阶级用来推翻封建制度的武器,现在却对准资产阶级自己了。

    但是,资产阶级不仅锻造了置自身于死地的武器;它还产生了将要运用这种武器的人——现代的工人,即无产者。

    随着资产阶级即资本的发展,无产阶级即现代工人阶级也在同一程度上得到发展;现代的工人只有当他们找到工作的时候才能生存,而且只有当他们的劳动增殖资本的时候才能找到工作。这些不得不把自己零星出卖的工人,像其他任何货物一样,也是一种商品,所以他们同样地受到竞争的一切变化、市场的一切波动的影响。

    由于推广机器和分工,无产者的劳动已经失去了任何独立的性质,因而对工人也失去了任何吸引力。工人变成了机器的单纯的附属品,要求他做的只是极其简单、极其单调和极容易学会的操作。因此,花在工人身上的费用,几乎只限于维持工人生活和延续工人后代所必需的生活资料。但是,商品的价格,从而劳动的价格,是同它的生产费用相等的。因此,劳动越使人感到厌恶,工资也就越少。不仅如此,机器越推广,分工越细致,劳动量出就越增加,这或者是由于工作时间的延长,或者是由于在一定时间内所要求的劳动的增加,机器运转的加速,等等。

    现代工业已经把家长式的师傅的小作坊变成了工业资本家的大工厂。挤在工厂里的工人群众就像士兵一样被组织起来。他们是产业军的普通士兵,受着各级军士和军官的层层监视。他们不仅仅是资产阶级的、资产阶级国家的奴隶,他们每日每时都受机器、受监工、首先是受各个经营工厂的资产者本人的奴役。这种专制制度越是公开地把营利宣布为自己的最终目的,它就越是可鄙、可恨和可恶。

    手的操作所要求的技巧和气力越少,换句话说,现代工业越发达,男工也就越受到女工和童工的排挤。对工人阶级来说,性别和年龄的差别再没有什么社会意义了。他们都只是劳动工具,不过因为年龄和性别的不同而需要不同的费用罢了。

    当厂主对工人的剥削告一段落,工人领到了用现钱支付的工资的时候,马上就有资产阶级中的另一部分人——房东、小店主、当铺老板等等向他们扑来。

    以前的中间等级的下层,即小工业家、小商人和小食利者,手工业者和农民——所有这些阶级都降落到无产阶级的队伍里来了,有的是因为他们的小资本不足以经营大工业,经不起较大的资本家的竞争;有的是因为他们的手艺已经被新的生产方法弄得不值钱了。无产阶级就是这样从居民的所有阶级中得到补充的。

    无产阶级经历了各个不同的发展阶段。它反对资产阶级的斗争是和它的存在同时开始的。

    最初是单个的工人,然后是某一工厂的工人,然后是某一地方的某一劳动部门的工人,同直接剥削他们的单个资产者作斗争。他们不仅仅攻击资产阶级的生产关系,而且攻击生产工具本身;他们毁坏那些来竞争的外国商品,捣毁机器,烧毁工厂,力图恢复已经失去的中世纪工人的地位。

    在这个阶段上,工人是分散在全国各地并为竞争所分裂的群众。工人的大规模集结,还不是他们自己联合的结果,而是资产阶级联合的结果,当时资产阶级为了达到自己的政治目的必须而且暂时还能够把整个无产阶级发动起来。因此,在这个阶段上,无产者不是同自己的敌人作斗争,而是同自己的敌人的敌人作斗争,即同专制君主制的残余、地主、非工业资产者和小资产者作斗争。因此,整个历史运动都集中在资产阶级手里;在这种条件下取得的每一个胜利都是资产阶级的胜利。

    但是,随着工业的发展,无产阶级不仅人数增加了,而且它结合成更大的集体,它的力量日益增长,它越来越感觉到自己的力量。机器使劳动的差别越来越小,使工资几乎到处都降到同样低的水平,因而无产阶级内部的利益、生活状况也越来越趋于一致。资产者彼此间日益加剧的竞争以及由此引起的商业危机,使工人的工资越来越不稳定;机器的日益迅速的和继续不断的改良,使工人的整个生活地位越来越没有保障;单个工人和单个资产者之间的冲突越来越具有两个阶级的冲突的性质。工人开始成立反对资产者的同盟;他们联合起来保卫自己的工资。他们甚至建立了经常性的团体,以便为可能发生的反抗准备食品。有些地方,斗争爆发为起义。

    工人有时也得到胜利,但这种胜利只是暂时的。他们斗争的真正成果并不是直接取得的成功,而是工人的越来越扩大的联合。这种联合由于大工业所造成的日益发达的交通工具而得到发展,这种交通工具把各地的工人彼此联系起来。只要有了这种联系,就能把许多性质相同的地方性的斗争汇合成全国性的斗争,汇合成阶级斗争。而一切阶级斗争都是政治斗争。中世纪的市民靠乡间小道需要几百年才能达到的联合,现代的无产者利用铁路只要几年就可以达到了。

    无产者组织成为阶级,从而组织成为政党这件事,不断地由于工人的自相竞争而受到破坏。但是,这种组织总是重新产生,并且一次比一次更强大,更坚固,更有力。它利用资产阶级内部的分裂,迫使他们用法律形式承认工人的个别利益。英国的十小时工作日法案就是一个例子。

    旧社会内部的所有冲突在许多方面都促进了无产阶级的发展。资产阶级处于不断的斗争中:最初反对贵族;后来反对同工业进步有利害冲突的那部分资产阶级;经常反对一切外国的资产阶级。在这一切斗争中,资产阶级都不得不向无产阶级呼吁,要求无产阶级援助,这样就把无产阶级卷进了政治运动。于是,资产阶级自己就把自己的教育因素即反对自身的武器给予了无产阶级。

    其次,我们已经看到,工业的进步把统治阶级的整批成员抛到无产阶级队伍里去,或者至少也使他们的生活条件受到威胁。他们也给无产阶级带来了大量的教育因素。

    最后,在阶级斗争接近决战的时期,统治阶级内部的、整个旧社会内部的瓦解过程,就达到非常强烈、非常尖锐的程度,甚至使得统治阶级中的一小部分人脱离统治阶级而归附于革命的阶级,即掌握着未来的阶级。所以,正像过去贵族中有一部分人转到资产阶级方面一样,现在资产阶级中也有一部分人,特别是已经提高到从理论上认识整个历史运动这一水平的一部分资产阶级思想家,转到无产阶级方面来了。

    在当前同资产阶级对立的一切阶级中,只有无产阶级是真正革命的阶级。其余的阶级都随着大工业的发展而日趋没落和灭亡,无产阶级却是大工业本身的产物。

    中间等级,即小工业家、小商人、手工业者、农民,他们同资产阶级作斗争,都是为了维护他们这种中间等级的生存,以免于灭亡。所以,他们不是革命的,而是保守的。不仅如此,他们甚至是反动的,因为他们力图使历史的车轮倒转。如果说他们是革命的,那是鉴于他们行将转入无产阶级的队伍,这样,他们就不是维护他们目前的利益,而是维护他们将来的利益,他们就离开自己原来的立场,而站到无产阶级的立场上来。

    流氓无产阶级是旧社会最下层中消极的腐化的部分,他们在一些地方也被无产阶级革命卷到运动里来,但是,由于他们的整个生活状况,他们更甘心于被人收买,去干反动的勾当。

    在无产阶级的生活条件中,旧社会的生活条件已经被消灭了。无产者是没有财产的;他们和妻子儿女的关系同资产阶级的家庭关系再没有任何共同之处了;现代的工业劳动,现代的资本压迫,无论在英国或法国,无论在美国或德国,都有是一样的,都使无产者失去了任何民族性。法律、道德、宗教在他们看来全都是资产阶级偏见,隐藏在这些偏见后面的全都是资产阶级利益。

    过去一切阶级在争得统治之后,总是使整个社会服从于它们发财致富的条件,企图以此来巩固它们已获得的生活地位。无产者只有废除自己的现存的占有方式,从而废除全部现存的占有方式,才能取得社会生产力。无产者没有什么自己的东西必须加以保护,他们必须摧毁至今保护和保障私有财产的一切。

    过去的一切运动都是少数人的或者为少数人谋利益的运动。无产阶级的运动是绝大多数人的、为绝大多数人谋利益的独立的运动。无产阶级,现今社会的最下层,如果不炸毁构成官方社会的整个上层,就不能抬起头来,挺起胸来。

    如果不就内容而就形式来说,无产阶级反对资产阶级的斗争首先是一国范围内的斗争。每一个国家的无产阶级当然首先应该打倒本国的资产阶级。

    在叙述无产阶级发展的最一般的阶段的时候,我们循序探讨了现存社会内部或多或少隐蔽着的国内战争,直到这个战争爆发为公开的革命,无产阶级用暴力推翻资产阶级而建立自己的统治。

    我们已经看到,至今的一切社会都是建立在压迫阶级和被压迫阶级的对立之上的。但是,为了有可能压迫一个阶级,就必须保证这个阶级至少有能够勉强维持它的奴隶般的生存的条件。农奴曾经在农奴制度下挣扎到公社成员的地位,小资产者曾经在封建专制制度的束缚下挣扎到资产者的地位。现代的工人却相反,他们并不是随着工业的进步而上升,而是越来越降到本阶级的生存条件以下。工人变成赤贫者,贫困比人口和财富增长得还要快。由此可以明显地看出,资产阶级再不能做社会的统治阶级了,再不能把自己阶级的生存条件当作支配一切的规律强加于社会了。资产阶级不能统治下去了,因为它甚至不能保证自己的奴隶维持奴隶的生活,因为它不得不让自己的奴隶落到不能养活它反而要它来养活的地步。社会再不能在它统治下生存下去了,就是说,它的生存不再同社会相容了。

    资产阶级生存和统治的根本条件,是财富在私人手里的积累,是资本的形成和增殖;资本的条件是雇佣劳动。雇佣劳动完全是建立在工人的自相竞争之上的。资产阶级无意中造成而又无力抵抗的工业进步,使工人通过结社而达到的革命联合代替了他们由于竞争而造成的分散状态。于是,随着大工业的发展,资产阶级赖以生产和占有产品的基础本身也就从它的脚下被挖掉了。它首先生产的是它自身的掘墓人。资产阶级的灭亡和无产阶级的胜利是同样不可避免的。
    \subsection{无产者和共产党人}
    共产党人同全体无产者的关系是怎样的呢?

    共产党人不是同其他工人政党相对立的特殊政党。

    他们没有任何同整个无产阶级的利益不同的利益。

    他们不提出任何特殊的原则,用以塑造无产阶级的运动。

    共产党人同其他无产阶级政党不同的地方只是:一方面,在无产者不同的民族的斗争中,共产党人强调和坚持整个无产阶级共同的不分民族的利益;另一方面,在无产阶级和资产阶级的斗争所经历的各个发展阶段上,共产党人始终代表整个运动的利益。

    因此,在实践方面,共产党人是各国工人政党中最坚决的、始终起推动作用的部分;在理论方面,他们胜过其余无产阶级群众的地方在于他们了解无产阶级运动的条件、进程和一般结果。

    共产党人的最近目的是和其他一切无产阶级政党的最近目的一样的:使无产阶级形成为阶级,推翻资产阶级的统治,由无产阶级夺取政权。

    共产党人的理论原理,决不是以这个或那个世界改革家所发明或发现的思想、原则为根据的。

    这些原理不过是现存的阶级斗争、我们眼前的历史运动的真实关系的一般表述。废除先前存在的所有制关系,并不是共产主义所独具的特征。

    一切所有制关系都经历了经常的历史更替、经常的历史变更。

    例如,法国革命废除了封建的所有制,代之以资产阶级的所有制。

    共产主义的特征并不是要废除一般的所有制,而是要废除资产阶级的所有制。

    但是,现代的资产阶级私有制是建立在阶级对立上面、建立在一些人对另一些人的剥削上面的产品生产和占有的最后而又完备的表现。

    从这个意义上说,共产党人可以把自己的理论概括为一句话:消灭私有制。

    有人责备我们共产党人,说我们消灭个人挣得的、自己劳动得来的财产,要消灭构成个人的一切自由、活动和独立的基础的财产。

    好一个劳动得来的、自己挣得的、自己赚来的财产!你们说的是资产阶级财产出现以前的那种小资产阶级、小农的财产吗?那种财产用不着我们去消灭,工业的发展已经把它消灭了,而且每天都在消灭它。

    或者,你们说的是现代的资产阶级的私有财产吧?

    但是,难道雇佣劳动,无产者的劳动,会给无产者创造出财产来吗?没有的事。这种劳动所创造的资本,即剥削雇佣劳动的财产,只有在不断产生出新的雇佣劳动来重新加以剥削的条件下才能增殖的财产。现今的这种财产是在资本和雇佣劳动的对立中运动的。让我们来看看这种对立的两个方面吧。

    做一个资本家,这就是说,他在生产中不仅占有一种纯粹个人的地位,而且占有一种社会地位。资本是集体的产物,它只有通过社会许多成员的共同活动,而且归根到底只有通过社会全体成员的共同活动,才能运动起来。

    因此,资本不是一种个人力量,而是一种社会力量。

    因此,把资本变为公共的、属于社会全体成员的财产,这并不是把个人财产变为社会财产。这里所改变的只是财产的社会性质。它将失掉它的阶级性质。

    现在,我们来看看雇佣劳动。

    雇佣劳动的平均价格是最低限度的工资,即工人为维持其工人的生活所必需的生活资料的数额。因此,雇佣工人靠自己的劳动所占有的东西,只够勉强维持他的生命的再生产。我们决不打算消灭这种供直接生命再生产用的劳动产品的个人占有,这种占有并不会留下任何剩余的东西使人们有可能支配别人的劳动。我们要消灭的只是这种占有的可怜的性质,在这种占有下,工人仅仅为增殖资本而活着,只有在统治阶级的利益需要他活着的时候才能活着。

    在资产阶级社会里,活的劳动只是增殖已经积累起来的劳动的一种手段。在共产主义社会里,已经积累起来的劳动只是扩大、丰富和提高工人的生活的一种手段。

    因此,在资产阶级社会里是过去支配现在,在共产主义社会里是现在支配过去。在资产阶级社会里,资本具有独立性和个性,而活动着的个人却没有独立性和个性。

    而资产阶级却把消灭这种关系说成是消灭个性和自由!说对了。的确,正是要消灭资产者的个性、独立性和自由。

    在现今的资产阶级生产关系的范围内,所谓自由就是自由贸易,自由买卖。

    但是,买卖一消失,自由买卖也就会消失。关于自由买卖的言论,也像我们的资产阶级的其他一切关于自由的大话一样,仅仅对于不自由的买卖来说,对于中世纪被奴役的市民来说,才是有意义的,而对于共产主义要消灭买卖、消灭资产阶级生产关系和资产阶级本身这一点来说,却是毫无意义的。

    我们要消灭私有制,你们就惊慌起来。但是,在你们的现存社会里,私有财产对十分之九的成员来说已经被消灭了;这种私有制这所以存在,正是因为私有财产对十分之九的成员来说已经不存在。可见,你们责备我们,是说我们要消灭那种以社会上的绝大多数人没有财产为必要条件的所有制。

    总而言之,你们责备我们,是说我们要消灭你们的那种所有制。的确,我们是要这样做的。

    从劳动不再能变为资本、货币、地租,一句话,不再能变为可以垄断的社会力量的时候起,就是说,从个人财产不再能变为资产阶级财产的时候起,你们说,个性被消灭了。

    由此可见,你们是承认,你们所理解的个性,不外是资产者、资产阶级私有者。这样的个性确实应当被消灭。

    共产主义并不剥夺任何人占有社会产品的权力,它只剥夺利用这种占有去奴役他人劳动的权力。

    有人反驳说,私有制一消灭,一切活动就会停止,懒惰之风就会兴起。

    这样说来,资产阶级社会早就应该因懒惰而灭亡了,因为在这个社会里劳者不获,获者不劳。所有这些顾虑,都可以归结为这样一个同义反复:一旦没有资本,也就不再有雇佣劳动了。

    所有这些对共产主义的物质产品的占有方式和生产方式的责备,也被扩及到精神产品的占有和生产方面。正如阶级的所有制的终止在资产者看来是生产本身的终止一样,阶级的教育的终止在他们看来就等于一切教育的终止。

    资产者唯恐失去的那种教育,绝大多数人来说是把人训练成机器。

    但是,你们既然用你们资产阶级关于自由、教育、法等等的观念来衡量废除资产阶级所有制的主张,那就请你们不要同我们争论了。你们的观念本身是资产阶级的生产关系和所有制关系的产物,正像你们的法不过是被奉为法律的你们这个阶级的意志一样,而这种意志的内容是由你们这个阶级的物质生活条件决定的。

    你们的利己观念使你们把自己的生产关系和所有制关系从历史的、在生产过程中是暂时的关系变成永恒的自然规律和理性规律,这种利己观念是你们和一切灭亡了的统治阶级所共有的。谈到古代所有制的时候你们所能理解的,谈到封建所有制的时候你们所能理解的,一谈到资产阶级所有制你们就再也不能理解了。

    消灭家庭!连极端的激进派也对共产党人的这种可耻的意图表示愤慨。

    现代的、资产阶级的家庭是建立在什么基础上的呢?是建立在资本上面,建立在私人发财上面的。这种家庭只是在资产阶级那里才以充分发展的形式存在着,而无产者的被迫独居和公开的卖淫则是它的补充。

    资产者的家庭自然会随着它的这种补充的消失而消失,两者都要随着资本的消失而消失。

    你们是责备我们要消灭父母对子女的剥削吗?我们承认这种罪状。

    但是,你们说,我们用社会教育代替家庭教育,就是要消灭人们最亲密的关系。

    而你们的教育不也是由社会决定的吗?不也是由你们进行教育时所处的那种社会关系决定的吗?不也是由社会通过学校等等进行的直接的或间接的干涉决定的吗?共产党人并没有发明社会对教育的作用;他们仅仅是要改变这种作用的性质,要使教育摆脱统治阶级的影响。

    无产者的一切家庭联系越是由于大工业的发展而被破坏,他们的子女越是由于这种发展而被变成单纯的商品和劳动工具,资产阶级关于家庭和教育、关于父母和子女的亲密关系的空话就越是令人作呕。

    但是,你们共产党人是要实行公妻制的啊,——整个资产阶级异口同声地向我们这样叫喊。

    资产者是把自己的妻子看作单纯的生产工具的。他们听说生产工具将要公共使用,自然就不能不想到妇女也会遭到同样的命运。

    他们想也没有想到,问题正在于使妇女不再处于单纯生产工具的地位。

    其实,我们的资产者装得道貌岸然,对所谓的共产党人的正式公妻制表示惊讶,那是再可笑不过了。公妻制无需共产党人来实行,它差不多是一向就有的。

    我们的资产者不以他们的无产者的妻子和女儿受他们支配为满足,正式的卖淫更不必说了,他们还以互相诱奸妻子为最大的享乐。

    资产阶级的婚姻实际上是公妻制。人们至多只能责备共产党人,说他们想用正式的、公开的公妻制来代替伪善地掩蔽着的公妻制。其实,不言而喻,随着现在的生产关系的消灭,从这种关系中产生的公妻制,即正式的和非正式的卖淫,也就消失了。

    有人还责备共产党人,说他们要取消祖国,取消民族。

    工人没有祖国。决不能剥夺他们所没有的东西。因为无产阶级首先必须取得政治统治,上升为民族的阶级,把自身组织成为民族,所以它本身还是民族的,虽然完全不是资产阶级所理解的那种意思。

    随着资产阶级的发展,随着贸易自由的实现和世界市场的建立,随着工业生产以及与之相适应的生活条件的趋于一致,各国人民之间的民族分隔和对立日益消失。

    无产阶级的统治将使它们更快地消失。联合的行动,至少是各文明国家的联合的行动,是无产阶级获得解放的首要条件之一。

    人对人的剥削一消灭,民族对民族的剥削就会随之消灭。

    民族内部的阶级对立一消失,民族之间的敌对关系就会随之消失。

    从宗教的、哲学的和一切意识形态的观点对共产主义提出的种种责难,都不值得详细讨论了。

    人们的观念、观点和概念,一句话,人们的意识,随着人们的生活条件、人们的社会关系、人们的社会存在的改变而改变,这难道需要经过深思才能了解吗?

    思想的历史除了证明精神生产随着物质生产的改造而改造,还证明了什么呢?任何一个时代的统治思想始终都不过是统治阶级的思想。

    当人们谈到使整个社会革命化的思想时,他们只是表明了一个事实:在旧社会内部已经形成了新社会的因素,旧思想的瓦解是同旧生活条件的瓦解步调一致的。

    当古代世界走向灭亡的时候,古代的各种宗教就被基督教战胜了。当基督教思想在18世纪被启蒙思想击败的时候,封建社会正在同当时革命的资产阶级进行殊死的斗争。信仰自由和宗教自由的思想,不过表明竞争在信仰领域里占统治地位罢了。

    “但是”,有人会说,“宗教的、道德的、哲学的、政治的、法的观念等等在历史发展的进程中固然是不断改变的,而宗教、道德、哲学、政治和法在这种变化中却始终保存着。

    此外,还存在着一切社会状态所共有的永恒真理,如自由、正义等等。但是共产主义要废除永恒真理,它要废除宗教、道德,而不是加以革新,所以共产主义是同至今的全部历史发展相矛盾的。”

    这种责难归结为什么呢?至今的一切社会的历史都是在阶级对立中运动的,而这种对立在不同的时代具有不同的形式。

    但是,不管阶级对立具有什么样的形式,社会上一部分人对另一部分人的剥削却是过去各个世纪所共有的事实。因此,毫不奇怪,各个世纪的社会意识,尽管形形色色、千差万别,总是在某些共同的形式中运动的,这些形式,这些意识形式,只有当阶级对立完全消失的时候才会完全消失。

    共产主义革命就是同传统的所有制关系实行最彻底的决裂;毫不奇怪,它在自己的发展进程中要同传统的观念实行最彻底的决裂。

    不过,我们还是把资产阶级对共产主义的种种责难撇开吧。

    前面我们已经看到,工人革命的第一步就是使无产阶级上升为统治阶级,争得民主。

    无产阶级将利用自己的政治统治,一步一步地夺取资产阶级的全部资本,把一切生产工具集中在国家即组织成为统治阶级的无产阶级手里,并且尽可能快地增加生产力的总量。

    要做到这一点,当然首先必须对所有权和资产阶级生产关系实行强制性的干涉,也就是采取这样一些措施,这些措施在经济上似乎是不够充分的和没有力量的,但是在运动进程中它们会越出本身,而且作为变革全部生产方式的手段是必不可少的。

    这些措施在不同的国家里当然会是不同的。

    但是,最先进的国家几乎都可以采取下面的措施:

    1、剥夺地产,把地租用于国家支出。

    2、征收高额累进税。

    3、废除继承权。

    4、没收一切流亡分子和叛乱分子的财产。

    5、通过拥有国家资本和独享垄断权的国家银行,把信贷集中在国家手里。

    6、把全部运输业集中在国家的手里。

    7、按照总的计划增加国家工厂和生产工具,开垦荒地和改良土壤。

    8、实行普遍劳动义务制,成立产业军,特别是在农业方面。

    9、把农业和工业结合起来,促使城乡对立逐步消灭。

    10、对所有儿童实行公共的和免费的教育。取消现在这种形式的儿童的工厂劳动。把教育同物质生产结合起来,等等。

    当阶级差别在发展进程中已经消失而全部生产集中在联合起来的个人的手里的时候,公共权力就失去政治性质。原来意义上的政治权力,是一个阶级用以压迫另一个阶级的有组织的暴力。如果说无产阶级在反对资产阶级的斗争中一定要联合为阶级,如果说它通过革命使自己成为统治阶级,并以统治阶级的资格用暴力消灭旧的生产关系,那么它在消灭这种生产关系的同时,也就消灭了阶级对立的存在条件,消灭阶级本身的存在条件,从而消灭了它自己这个阶级的统治。

    代替那存在着阶级和阶级对立的资产阶级旧社会的,将是这样一个联合体,在那里,每个人的自由发展是一切人的自由发展的条件。
    \subsection{社会主义的和共产主义的文献}
    1.反动的社会主义

    (甲)封建的社会主义

    法国和英国的贵族,按照他们的历史地位所负的使命,就是写一些抨击现代资产阶级社会的作品。在法国的1830年七月革命和英国的改革运动中,他们再一次被可恨的暴发户打败了。从此就再谈不上严重的政治斗争了。他们还能进行的只是文字斗争。但是,即使在文字方面也不可能重弹复辟时期的老调了。为了激起同情,贵族们不得不装模作样,似乎他们已经不关心自身的利益,只是为了被剥削的工人阶级的利益才去写对资产阶级的控诉书。他们用来泄愤的手段是:唱唱诅咒他们的新统治者的歌,并向他叽叽咕咕地说一些或多或少凶险的预言。

    这样就产生了封建的社会主义,半是挽歌,半是谤文,半是过去的回音,半是未来的恫吓;它有时也能用辛辣、俏皮而尖刻的评论剌中资产阶级的心,但是它由于完全不能理解现代历史的进程而总是令人感到可笑。

    为了拉拢人民,贵族们把无产阶级的乞食袋当作旗帜来挥舞。但是,每当人民跟着他们走的时候,都发现他们的臀部带有旧的封建纹章,于是就哈哈大笑,一哄而散。

    一部分法国正统派和“青年英国”,都演过这出戏。

    封建主说,他们的剥削方式和资产阶级的剥削不同,那他们只是忘记了,他们是在完全不同的、目前已经过时的情况和条件下进行剥削的。他们说,在他们的统治下并没有出现过现代的无产阶级,那他们只是忘记了,现代的资产阶级正是他们的社会制度的必然产物。

    不过,他们毫不掩饰自己的批评的反动性质,他们控告资产阶级的主要罪状正是在于:在资产阶级的统治下有一个将把整个旧社会制度炸毁的阶级发展起来。

    他们责备资产阶级,与其说是因为它产生了无产阶级,不如说是因为它产生了革命的无产阶级。

    因此,在政治实践中,他们参与对工人阶级采取的一切暴力措施,在日常生活中,他们违背自己的那一套冠冕堂皇的言词,屈尊拾取金苹果,不顾信义、仁爱和名誉去做羊毛、甜菜和烧洒的买卖。

    正如僧侣总是同封建主携手同行一样,僧侣的社会主义也总是同封建的社会主义携手同行的。

    要给基督教禁欲主义涂上一层社会主义的色彩,是再容易不过了。基督教不是也激烈反对私有财产,反对婚姻,反对国家吗?它不是提倡用行善和求乞、独身和禁欲、修道和礼拜来代替这一切吗?基督教的社会主义,只不过是僧侣用来使贵族的怨愤神圣的圣水罢了。

    (乙)小资产阶级的社会主义

    封建贵族并不是被资产阶级所推翻的、其生活条件在现代资产阶级社会里日益恶化和消失的唯一阶级。中世纪的城关市民和小农等级是现代资产阶级的前身。在工商业不很发达的国家里,这个阶级还在新兴的资产阶级身旁勉强生存着。

    在现代文明已经发展的国家里,形成了一个新的小资产阶级,它摇摆于无产阶级和资产阶级之间,并且作为资产阶级社会的补充部分不断地重新组成。但是,这一阶级的成员经常被竞争抛到无产阶级队伍里去,而且,随着大工业的发展,他们甚至觉察到,他们很快就会完全失去他们作为现代社会中一个独立部分的地位,在商业、工业和农业中很快就会被监工和雇员所代替。

    在农民阶级远远超过人口半数的国家,例如在法国,那些站在无产阶级方面反对资产阶级的著作家,自然是用小资产阶级和小农的尺度去批判资产阶级制度的,是从小资产阶级的立场出发替工人说话的。这样就形成了小资产阶级的社会主义。西斯蒙第不仅对法国而且对英国来说都是这类著作家的首领。

    这种社会主义非常透彻地分析了现代生产关系中的矛盾。它揭穿了经济学家的虚伪的粉饰。它确凿地证明了机器和分工的破坏作用、资本和地产的积聚、生产过剩、危机、小资产者和小农的必然没落、无产阶级的贫困、生产的无政府状态、财富分配的极不平均、各民族之间的毁灭性的工业战争,以及旧风尚、旧家庭关系和旧民族性的解体。

    但是,这种社会主义按其实际内容来说,或者是企图恢复旧的生产资料和交换手段,从而恢复旧的所有制关系和旧的社会,或者是企图重新把现代的生产资料和交换手段硬塞到已被它们突破而且必然被突破的旧的所有制关系的框子里去。它在这两种场合都是反动的,同时又是空想的。

    工业中的行会制度,农业中的宗法经济,——这就是它的结论。

    这一思潮在它以后的发展中变成了一种怯懦的悲叹。

    (丙)德国的或“真正的”社会主义

    法国的社会主义和共产主义的文献是在居于统治地位的资产阶级的压迫下产生的,并且是同这种统治作斗争的文字表现,这种文献被搬到德国的时候,那里的资产阶级才刚刚开始进行反对封建专制制度的斗争。

    德国的哲学家、半哲学家和美文学家,贪婪地抓住了这种文献,不过他们忘记了:在这种著作从法国搬到德国的时候,法国的生活条件却没有同时搬过去。在德国的条件下,法国的文献完全失去了直接实践的意义,而只具有纯粹文献的形式。它必然表现为关于真正的社会、关于实现人的本质的无谓思辨。这样,第一次法国革命的要求,在18世纪的德国哲学家看来,不过是一般“实践理性”的要求,而革命的法国资产阶级的意志的表现,在他们心目中就是纯粹的意志、本来的意志、真正人的意志的规律。

    德国著作家的唯一工作,就是把新的法国的思想同他们的旧的哲学信仰调和起来,或者毋宁说,就是从他们的哲学观点出发去掌握法国的思想。

    这种掌握,就像掌握外国语一样,是通过翻译的。

    大家知道,僧侣们曾经在古代异教经典的手抄本上面写上荒诞的天主教圣徒传。德国著作家对世俗的法国文献采取相反的作法。他们在法国的原著下面写上自己的哲学胡说。例如,他们在法国人对货币关系的批判下面写上“人的本质的外化”,在法国人对资产阶级国家的批判下面写上所谓“抽象普遍物的统治的扬弃”,等等。

    这种在法国人的论述下面塞进自己哲学词句的做法,他们称之为“行动的哲学”、”真正的社会主义”、“德国的社会主义科学”、“社会主义的哲学论证”,等等。

    法国的社会主义和共产主义的文献就这样被完全阉割了。既然这种文献在德国人手里已不再表现一个阶级反对另一个阶级的斗争,于是德国人就认为:他们克服了“法国人的片面性”,他们不代表真实的要求,而代表真理的要求,不代表无产者的利益,而代表人的本质的利益,即一般人的利益,这种人不属于任何阶级,根本不存在于现实界,而只存在于云雾弥漫的哲学幻想的太空。

    这种曾经郑重其事地看待自己那一套拙劣的小学生作业并且大言不惭地加以吹嘘的德国社会主义,现在渐渐失去了它的自炫博学的天真。

    德国的特别是普鲁士的资产阶级反对封建主和专制王朝的斗争,一句话,自由主义运动,越来越严重了。

    于是,“真正的”社会主义就得到了一个好机会,把社会主义的要求同政治运动对立起来,用诅咒异端邪说的传统办法诅咒自由主义,诅咒代议制国家,诅咒资产阶级的竞争、资产阶级的新闻出版自由、资产阶级的法、资产阶级的自由和平等,并且向人民群众大肆宣扬,说什么在这个资产阶级运动中,人民群众非但一无所得,反而会失去一切。德国的社会主义恰好忘记了,法国的批判(德国的社会主义是这种批判的可怜的回声)是以现代的资产阶级社会以及相应的物质生活条件和相当的政治制度为前提的,而这一切前提当时在德国正是尚待争取的。

    这种社会主义成了德意志各邦专制政府及其随从——僧侣、教员、容克和官僚求之不得的、吓唬来势汹汹的资产阶级的稻草人。

    这种社会主义是这些政府用来镇压德国工人起义的毒辣的皮鞭和枪弹的甜蜜的补充。

    既然“真正的”社会主义就这样成了这些政府对付德国资产阶级的武器,那么它也就直接代表了一种反动的利益,即德国小市民的利益。在德国,16世纪遗留下来的、从那时起经常以不同形式重新出现的小资产阶级,是现存制度的真实的社会基础。

    保存这个小资产阶级,就是保存德国的现存制度。这个阶级胆战心惊地从资产阶级的工业统治和政治统治那里等候着无可幸免的灭亡,这一方面是由于资本的积聚,另一方面是由于革命无产阶级的兴起。在它看来,“真正的”社会主义能起一箭双雕的作用。“真正的”社会主义像瘟疫一样流行起来了。

    德国的社会主义者给自己的那几条干瘪的“永恒真理”披上一件用思辨的蛛丝织成的、绣满华丽辞藻的花朵和浸透甜情蜜意的甘露的外衣,这件光彩夺目的外衣只是使他们的货物在这些顾客中间增加销路罢了。

    同时,德国的社会主义也越来越认识到自己的使命就是充当这种小市民的夸夸其谈的代言人。

    它宣布德意志民族是模范的民族,德国小市民是模范的人。它给这些小市民的每一种丑行都加上奥秘的、高尚的、社会主义的意义,使之变成完全相反的东西。它发展到最后,就直接反对共产主义的“野蛮破坏的”倾向,并且宣布自己是不偏不倚地超乎任何阶级斗争之上的。现今在德国流行的一切所谓社会主义和共产主义的著作,除了极少数的例外,都属于这一类卑鄙龌龊的、令人委靡的文献。

    2.保守的或资产阶级的社会主义

    资产阶级中的一部分人想要消除社会的弊病,以便保障资产阶级社会的生存。

    这一部分人包括:经济学家、博爱主义者、人道主义者、劳动阶级状况改善派、慈善事业组织者、动物保护协会会员、戒酒协会发起人以及形形色色的小改良家。这种资产阶级的社会主义甚至被制成一些完整的体系。

    我们可以举蒲鲁东的《贫困的哲学》作为例子。

    社会主义的资产者愿意要现代社会的生存条件,但是不要由这些条件必然产生的斗争和危险。他们愿意要现存的社会,但是不要那些使这个社会革命化和瓦解的因素。他们愿意要资产阶级,但是不要无产阶级。在资产阶级看来,它所统治的世界自然是最美好的世界。资产阶级的社会主义把这种安慰人心的观念制成半套或整套的体系。它要求无产阶级实现它的体系,走进新的耶路撒冷,其实它不过是要求无产阶级停留在现今的社会里,但是要抛弃他们关于这个社会的可恶的观念。

    这种社会主义的另一种不够系统、但是比较实际的形式,力图使工人阶级厌弃一切革命运动,硬说能给工人阶级带来好处的并不是这样或那样的政治改革,而仅仅是物质生活条件即经济关系的改变。但是,这种社会主义所理解的物质生活条件的改变,绝对不是只有通过革命的途径才能实现的资产阶级生产关系的废除,而是一些在这种生产关系的基础上实行的行政上的改良,因而丝毫不会改变资本和雇佣劳动的关系,至多只能减少资产阶级的统治费用和简化它的财政管理。

    资产阶级的社会主义只有在它变成纯粹的演说辞令的时候,才获得自己的适当的表现。

    自由贸易!为了工人阶级的利益;保护关税!为了工人阶级的利益;单身牢房!为了工人阶级的利益。——这才是资产阶级的社会主义唯一认真说出的最后的话。

    资产阶级的社会主义就是这样一个论断:资产者之为资产者,是为了工人阶级的利益。

    3.批判的空想的社会主义和共产主义

    在这里,我们不谈在现代一切大革命中表达过无产阶级要求的文献(巴贝夫等人的著作)。

    无产阶级在普遍激动的时代、在推翻封建社会的时期直接实现自己阶级利益的最初尝试,都不可避免地遭到了失败,这是由于当时无产阶级本身还不够发展,由于无产阶级解放的物质条件还没具备,这些条件只是资产阶级时代的产物。随着这些早期的无产阶级运动而出现的革命文献,就其内容来说必然是反动的。这种文献倡导普遍的禁欲主义和粗陋的平均主义。

    本来意义的社会主义和共产主义的体系,圣西门、傅立叶、欧文等人的体系,是在无产阶级和资产阶级之间的斗争还不发展的最初时期出现的。关于这个时期,我们在前面已经叙述过了(见《资产阶级和无产阶级》)。

    诚然,这些体系的发明家看到了阶级的对立,以及占统治地位的社会本身中的瓦解因素的作用。但是,他们看不到无产阶级方面的任何历史主动性,看不到它所特有的任何政治运动。

    由于阶级对立的发展是同工业的发展步调一致的,所以这些发明家也不可能看到无产阶级解放的物质条件,于是他们就去探求某种社会科学、社会规律,以便创造这些条件。

    社会的活动要由他们个人的发明活动来代替,解放的历史条件要由幻想的条件来代替,无产阶级的逐步组织成为阶级要由一种特意设计出来的社会组织来代替。在他们看来,今后的世界历史不过是宣传和实施他们的社会计划。

    诚然,他们也意识到,他们的计划主要是代表工人阶级这一受苦最深的阶级的利益。在他们心目中,无产阶级只是一个受苦最深的阶级。

    但是,由于阶级斗争不发展,由于他们本身的生活状况,他们就以为自己是高高超乎这种阶级对立之上的。他们要改善社会一切成员的生活状况,甚至生活最优裕的成员也包括在内。因此,他们总是不加区别地向整个社会呼吁,而且主要是向统治阶级呼吁。他们以为,人们只要理解他们的体系,就会承认这种体系是最美好的社会的最美好的计划。

    因此,他们拒绝一切政治行动,特别是一切革命行动;他们想通过和平的途径达到自己的目的,并且企图通过一些小型的、当然不会成功的试验,通过示范的力量来为新的社会福音开辟道路。

    这种对未来社会的幻想的描绘,在无产阶级还很不发展、因而对本身的地位的认识还基于幻想的时候,是同无产阶级对社会普遍改造的最初的本能的渴望相适应的。

    但是,这些社会主义和共产主义的著作也含有批判的成分。这些著作抨击现存社会的全部基础。因此,它们提供了启发工人觉悟的极为宝贵的材料。它们关于未来社会的积极的主张,例如消灭城乡对立,消灭家庭,消灭私人营利,消灭雇佣劳动,提倡社会和谐,把国家变成纯粹的生产管理机构,——所有这些主张都只是表明要消灭阶级对立,而这种阶级对立在当时刚刚开始发展,它们所知道的只是这种对立的早期的、不明显的、不确定的形式。因此,这些主张本身还带有纯粹空想的性质。

    批判的空想的社会主义和共产主义的意义,是同历史的发展成反比的。阶级斗争越发展和越具有确定的形式,这种超乎阶级斗争的幻想,这种反对阶级斗争的幻想,就越失去任何实践意义和任何理论根据。所以,虽然这些体系的创始人在许多方面是革命的,但是他们的信徒总是组成一些反动的宗派。这些信徒无视无产阶级的历史进展,还是死守着老师们的旧观点。因此,他们一贯企图削弱阶级斗争,调和对立。他们还总是梦想用试验的办法来实现自己的社会空想,创办单个的法伦斯泰尔,建立国内移民区,创立小伊加利亚,即袖珍版的新耶路撒冷,——而为了建造这一切空中楼阁,他们就不得不呼吁资产阶级发善心和慷慨解囊。他们逐渐地堕落到上述反动的或保守的社会主义者的一伙中去了,所不同的只是他们更加系统地卖弄学问,狂热地迷信自己那一套社会科学的奇功异效。

    因此,他们激烈地反对工人的一切政治运动,认为这种运动只是由于盲目地不相信新福音才发生的。

    在英国,有欧文主义者反对宪章派,在法国,有傅立叶主义者反对改革派。
    \subsection{共产党人对各种反对党派的态度}
    看过第二章之后,就可以了解共产党人同已经形成的工人政党的关系,因而也就可以了解他们同英国宪章派和北美土地改革派的关系。

    共产党人为工人阶级的最近的目的和利益而斗争,但是他们在当前的运动中同时代表运动的未来。在法国,共产党人同社会主义民主党联合起来反对保守的和激进的资产阶级,但是并不因此放弃对那些从革命的传统中承袭下来的空谈和幻想采取批判态度的权利。

    在瑞士,共产党人支持激进派,但是并不忽略这个政党是由互相矛盾的分子组成的,其中一部分是法国式的民主社会主义者,一部分是激进的资产者。

    在波兰人中间,共产党人支持那个把土地革命当作民族解放的条件的政党,即发动过1846年克拉科夫起义的政党。

    在德国,只要资产阶级采取革命的行动,共产党就同它一起去反对专制君主制、封建土地所有制和小市民的反动性。

    但是,共产党一分钟也不忽略教育工人尽可能明确地意识到资产阶级和无产阶级的敌对的对立,以便德国工人能够立刻利用资产阶级统治所必然带来的社会的和政治的条件作为反对资产阶级的武器,以便在推翻德国的反动阶级之后立即开始反对资产阶级本身的斗争。

    共产党人把自己的主要注意力集中在德国,因为德国正处在资产阶级革命的前夜,因为同17世纪的英国和18世纪的法国相比,德国将在整个欧洲文明更进步的条件下,拥有发展得多的无产阶级去实现这个变革,因而德国的资产阶级革命只能是无产阶级革命的直接序幕。

    总之,共产党人到处都支持一切反对现存的社会制度和政治制度的革命运动。

    在所有这些运动中,他们都强调所有制问题是运动的基本问题,不管这个问题的发展程度怎样。

    最后,共产党人到处都努力争取全世界民主政党之间的团结和协调。

    共产党人不屑于隐瞒自己的观点和意图。他们公开宣布:他们的目的只有用暴力推翻全部现存的社会制度才能达到。让统治阶级在共产主义革命面前发抖吧。无产者在这个革命中失去的只是锁链。他们获得的将是整个世界。

    全世界无产者,联合起来!
    \section{中国共产党章程}
    \subsection{总纲}
  中国共产党是中国工人阶级的先锋队,同时是中国人民和中华民族的先锋队,是中国特色社会主义事业的领导核心,代表中国先进生产力的发展要求,代表中国先进文化的前进方向,代表中国最广大人民的根本利益。党的最高理想和最终目标是实现共产主义。

  中国共产党以马克思列宁主义、毛泽东思想、邓小平理论、“三个代表”重要思想、科学发展观、习近平新时代中国特色社会主义思想作为自己的行动指南。

  马克思列宁主义揭示了人类社会历史发展的规律,它的基本原理是正确的,具有强大的生命力。中国共产党人追求的共产主义最高理想,只有在社会主义社会充分发展和高度发达的基础上才能实现。社会主义制度的发展和完善是一个长期的历史过程。坚持马克思列宁主义的基本原理,走中国人民自愿选择的适合中国国情的道路,中国的社会主义事业必将取得最终的胜利。

  以毛泽东同志为主要代表的中国共产党人,把马克思列宁主义的基本原理同中国革命的具体实践结合起来,创立了毛泽东思想。毛泽东思想是马克思列宁主义在中国的运用和发展,是被实践证明了的关于中国革命和建设的正确的理论原则和经验总结,是中国共产党集体智慧的结晶。在毛泽东思想指引下,中国共产党领导全国各族人民,经过长期的反对帝国主义、封建主义、官僚资本主义的革命斗争,取得了新民主主义革命的胜利,建立了人民民主专政的中华人民共和国;新中国成立以后,顺利地进行了社会主义改造,完成了从新民主主义到社会主义的过渡,确立了社会主义基本制度,发展了社会主义的经济、政治和文化。

  十一届三中全会以来,以邓小平同志为主要代表的中国共产党人,总结新中国成立以来正反两方面的经验,解放思想,实事求是,实现全党工作中心向经济建设的转移,实行改革开放,开辟了社会主义事业发展的新时期,逐步形成了建设中国特色社会主义的路线、方针、政策,阐明了在中国建设社会主义、巩固和发展社会主义的基本问题,创立了邓小平理论。邓小平理论是马克思列宁主义的基本原理同当代中国实践和时代特征相结合的产物,是毛泽东思想在新的历史条件下的继承和发展,是马克思主义在中国发展的新阶段,是当代中国的马克思主义,是中国共产党集体智慧的结晶,引导着我国社会主义现代化事业不断前进。

  十三届四中全会以来,以江泽民同志为主要代表的中国共产党人,在建设中国特色社会主义的实践中,加深了对什么是社会主义、怎样建设社会主义和建设什么样的党、怎样建设党的认识,积累了治党治国新的宝贵经验,形成了“三个代表”重要思想。“三个代表”重要思想是对马克思列宁主义、毛泽东思想、邓小平理论的继承和发展,反映了当代世界和中国的发展变化对党和国家工作的新要求,是加强和改进党的建设、推进我国社会主义自我完善和发展的强大理论武器,是中国共产党集体智慧的结晶,是党必须长期坚持的指导思想。始终做到“三个代表”,是我们党的立党之本、执政之基、力量之源。

  十六大以来,以胡锦涛同志为主要代表的中国共产党人,坚持以邓小平理论和“三个代表”重要思想为指导,根据新的发展要求,深刻认识和回答了新形势下实现什么样的发展、怎样发展等重大问题,形成了以人为本、全面协调可持续发展的科学发展观。科学发展观是同马克思列宁主义、毛泽东思想、邓小平理论、“三个代表”重要思想既一脉相承又与时俱进的科学理论,是马克思主义关于发展的世界观和方法论的集中体现,是马克思主义中国化重大成果,是中国共产党集体智慧的结晶,是发展中国特色社会主义必须长期坚持的指导思想。

  十八大以来,以习近平同志为主要代表的中国共产党人,坚持把马克思主义基本原理同中国具体实际相结合、同中华优秀传统文化相结合,科学回答了新时代坚持和发展什么样的中国特色社会主义、怎样坚持和发展中国特色社会主义等重大时代课题,创立了习近平新时代中国特色社会主义思想。习近平新时代中国特色社会主义思想是对马克思列宁主义、毛泽东思想、邓小平理论、“三个代表”重要思想、科学发展观的继承和发展,是当代中国马克思主义、二十一世纪马克思主义,是中华文化和中国精神的时代精华,是党和人民实践经验和集体智慧的结晶,是中国特色社会主义理论体系的重要组成部分,是全党全国人民为实现中华民族伟大复兴而奋斗的行动指南,必须长期坚持并不断发展。在习近平新时代中国特色社会主义思想指导下,中国共产党领导全国各族人民,统揽伟大斗争、伟大工程、伟大事业、伟大梦想,推动中国特色社会主义进入了新时代,实现第一个百年奋斗目标,开启了实现第二个百年奋斗目标新征程。

  改革开放以来我们取得一切成绩和进步的根本原因,归结起来就是:开辟了中国特色社会主义道路,形成了中国特色社会主义理论体系,确立了中国特色社会主义制度,发展了中国特色社会主义文化。全党同志要倍加珍惜、长期坚持和不断发展党历经艰辛开创的这条道路、这个理论体系、这个制度、这个文化,高举中国特色社会主义伟大旗帜,坚定道路自信、理论自信、制度自信、文化自信,发扬斗争精神,增强斗争本领,贯彻党的基本理论、基本路线、基本方略,为实现推进现代化建设、完成祖国统一、维护世界和平与促进共同发展这三大历史任务,实现第二个百年奋斗目标、实现中华民族伟大复兴的中国梦而奋斗。

  中国共产党自成立以来,始终把为中国人民谋幸福、为中华民族谋复兴作为自己的初心使命,历经百年奋斗,从根本上改变了中国人民的前途命运,开辟了实现中华民族伟大复兴的正确道路,展示了马克思主义的强大生命力,深刻影响了世界历史进程,锻造了走在时代前列的中国共产党。经过长期实践,积累了坚持党的领导、坚持人民至上、坚持理论创新、坚持独立自主、坚持中国道路、坚持胸怀天下、坚持开拓创新、坚持敢于斗争、坚持统一战线、坚持自我革命的宝贵历史经验,这是党和人民共同创造的精神财富,必须倍加珍惜、长期坚持,并在实践中不断丰富和发展。

  我国正处于并将长期处于社会主义初级阶段。这是在原本经济文化落后的中国建设社会主义现代化不可逾越的历史阶段,需要上百年的时间。我国的社会主义建设,必须从我国的国情出发,走中国特色社会主义道路,以中国式现代化全面推进中华民族伟大复兴。在现阶段,我国社会的主要矛盾是人民日益增长的美好生活需要和不平衡不充分的发展之间的矛盾。由于国内的因素和国际的影响,阶级斗争还在一定范围内长期存在,在某种条件下还有可能激化,但已经不是主要矛盾。我国社会主义建设的根本任务,是进一步解放生产力,发展生产力,逐步实现社会主义现代化,并且为此而改革生产关系和上层建筑中不适应生产力发展的方面和环节。必须坚持和完善公有制为主体、多种所有制经济共同发展,按劳分配为主体、多种分配方式并存,社会主义市场经济体制等基本经济制度,鼓励一部分地区和一部分人先富起来,逐步实现全体人民共同富裕,在生产发展和社会财富增长的基础上不断满足人民日益增长的美好生活需要,促进人的全面发展。发展是我们党执政兴国的第一要务。必须坚持以人民为中心的发展思想,把握新发展阶段,贯彻创新、协调、绿色、开放、共享的新发展理念,加快构建以国内大循环为主体、国内国际双循环相互促进的新发展格局,推动高质量发展。各项工作都要把有利于发展社会主义社会的生产力,有利于增强社会主义国家的综合国力,有利于提高人民的生活水平,作为总的出发点和检验标准,尊重劳动、尊重知识、尊重人才、尊重创造,做到发展为了人民、发展依靠人民、发展成果由人民共享。必须按照中国特色社会主义事业“五位一体”总体布局和“四个全面”战略布局,统筹推进经济建设、政治建设、文化建设、社会建设、生态文明建设,协调推进全面建设社会主义现代化国家、全面深化改革、全面依法治国、全面从严治党。新时代新征程,经济和社会发展的战略目标是,到二〇三五年基本实现社会主义现代化,到本世纪中叶把我国建成社会主义现代化强国。

  中国共产党在社会主义初级阶段的基本路线是:领导和团结全国各族人民,以经济建设为中心,坚持四项基本原则,坚持改革开放,自力更生,艰苦创业,为把我国建设成为富强民主文明和谐美丽的社会主义现代化强国而奋斗。

  中国共产党在领导社会主义事业中,必须坚持以经济建设为中心,其他各项工作都服从和服务于这个中心。要实施科教兴国战略、人才强国战略、创新驱动发展战略、乡村振兴战略、区域协调发展战略、可持续发展战略、军民融合发展战略,充分发挥科学技术作为第一生产力的作用,充分发挥人才作为第一资源的作用,充分发挥创新作为引领发展第一动力的作用,依靠科技进步,提高劳动者素质,促进国民经济更高质量、更有效率、更加公平、更可持续、更为安全发展。

  坚持社会主义道路、坚持人民民主专政、坚持中国共产党的领导、坚持马克思列宁主义毛泽东思想这四项基本原则,是我们的立国之本。在社会主义现代化建设的整个过程中,必须坚持四项基本原则,反对资产阶级自由化。

  坚持改革开放,是我们的强国之路。只有改革开放,才能发展中国、发展社会主义、发展马克思主义。要全面深化改革,完善和发展中国特色社会主义制度,推进国家治理体系和治理能力现代化。要从根本上改革束缚生产力发展的经济体制,坚持和完善社会主义市场经济体制;与此相适应,要进行政治体制改革和其他领域的改革。要坚持对外开放的基本国策,吸收和借鉴人类社会创造的一切文明成果。改革开放应当大胆探索,勇于开拓,提高改革决策的科学性,更加注重改革的系统性、整体性、协同性,在实践中开创新路。

  中国共产党领导人民发展社会主义市场经济。毫不动摇地巩固和发展公有制经济,毫不动摇地鼓励、支持、引导非公有制经济发展。发挥市场在资源配置中的决定性作用,更好发挥政府作用,建立完善的宏观调控体系。统筹城乡发展、区域发展、经济社会发展、人与自然和谐发展、国内发展和对外开放,调整经济结构,转变经济发展方式,推进供给侧结构性改革。促进新型工业化、信息化、城镇化、农业现代化同步发展,建设社会主义新农村,走中国特色新型工业化道路,建设创新型国家和世界科技强国。

  中国共产党领导人民发展社会主义民主政治。坚持党的领导、人民当家作主、依法治国有机统一,走中国特色社会主义政治发展道路、中国特色社会主义法治道路,扩大社会主义民主,建设中国特色社会主义法治体系,建设社会主义法治国家,巩固人民民主专政,建设社会主义政治文明。坚持和完善人民代表大会制度、中国共产党领导的多党合作和政治协商制度、民族区域自治制度以及基层群众自治制度。发展更加广泛、更加充分、更加健全的全过程人民民主,推进协商民主广泛多层制度化发展,切实保障人民管理国家事务和社会事务、管理经济和文化事业的权利。尊重和保障人权。广开言路,建立健全民主选举、民主协商、民主决策、民主管理、民主监督的制度和程序。完善中国特色社会主义法律体系,加强法律实施工作,实现国家各项工作法治化。

  中国共产党领导人民发展社会主义先进文化。建设社会主义精神文明,实行依法治国和以德治国相结合,提高全民族的思想道德素质和科学文化素质,为改革开放和社会主义现代化建设提供强大的思想保证、精神动力和智力支持,建设社会主义文化强国。加强社会主义核心价值体系建设,坚持马克思主义指导思想,树立中国特色社会主义共同理想,弘扬以爱国主义为核心的民族精神和以改革创新为核心的时代精神,培育和践行社会主义核心价值观,倡导社会主义荣辱观,增强民族自尊、自信和自强精神,抵御资本主义和封建主义腐朽思想的侵蚀,扫除各种社会丑恶现象,努力使我国人民成为有理想、有道德、有文化、有纪律的人民。对党员要进行共产主义远大理想教育。大力发展教育、科学、文化事业,推动中华优秀传统文化创造性转化、创新性发展,继承革命文化,发展社会主义先进文化,提高国家文化软实力。牢牢掌握意识形态工作领导权,不断巩固马克思主义在意识形态领域的指导地位,巩固全党全国人民团结奋斗的共同思想基础。

  中国共产党领导人民构建社会主义和谐社会。按照民主法治、公平正义、诚信友爱、充满活力、安定有序、人与自然和谐相处的总要求和共同建设、共同享有的原则,以保障和改善民生为重点,解决好人民最关心、最直接、最现实的利益问题,使发展成果更多更公平惠及全体人民,不断增强人民群众获得感,努力形成全体人民各尽其能、各得其所而又和谐相处的局面。加强和创新社会治理。严格区分和正确处理敌我矛盾和人民内部矛盾这两类不同性质的矛盾。加强社会治安综合治理,依法坚决打击各种危害国家安全和利益、危害社会稳定和经济发展的犯罪活动和犯罪分子,保持社会长期稳定。坚持总体国家安全观,统筹发展和安全,坚决维护国家主权、安全、发展利益。

  中国共产党领导人民建设社会主义生态文明。树立尊重自然、顺应自然、保护自然的生态文明理念,增强绿水青山就是金山银山的意识,坚持节约资源和保护环境的基本国策,坚持节约优先、保护优先、自然恢复为主的方针,坚持生产发展、生活富裕、生态良好的文明发展道路。着力建设资源节约型、环境友好型社会,实行最严格的生态环境保护制度,形成节约资源和保护环境的空间格局、产业结构、生产方式、生活方式,为人民创造良好生产生活环境,实现中华民族永续发展。

  中国共产党坚持对人民解放军和其他人民武装力量的绝对领导,贯彻习近平强军思想,加强人民解放军的建设,坚持政治建军、改革强军、科技强军、人才强军、依法治军,建设一支听党指挥、能打胜仗、作风优良的人民军队,把人民军队建设成为世界一流军队,切实保证人民解放军有效履行新时代军队使命任务,充分发挥人民解放军在巩固国防、保卫祖国和参加社会主义现代化建设中的作用。

  中国共产党维护和发展平等团结互助和谐的社会主义民族关系,积极培养、选拔少数民族干部,帮助少数民族和民族地区发展经济、文化和社会事业,铸牢中华民族共同体意识,实现各民族共同团结奋斗、共同繁荣发展。全面贯彻党的宗教工作基本方针,团结信教群众为经济社会发展作贡献。

  中国共产党同全国各民族工人、农民、知识分子团结在一起,同各民主党派、无党派人士、各民族的爱国力量团结在一起,进一步发展和壮大由全体社会主义劳动者、社会主义事业的建设者、拥护社会主义的爱国者、拥护祖国统一和致力于中华民族伟大复兴的爱国者组成的最广泛的爱国统一战线。不断加强全国人民包括香港特别行政区同胞、澳门特别行政区同胞、台湾同胞和海外侨胞的团结。全面准确、坚定不移贯彻“一个国家、两种制度”的方针,促进香港、澳门长期繁荣稳定,坚决反对和遏制“台独”,完成祖国统一大业。

  中国共产党坚持独立自主的和平外交政策,坚持和平发展道路,坚持互利共赢的开放战略,统筹国内国际两个大局,积极发展对外关系,努力为我国的改革开放和现代化建设争取有利的国际环境。在国际事务中,弘扬和平、发展、公平、正义、民主、自由的全人类共同价值,坚持正确义利观,维护我国的独立和主权,反对霸权主义和强权政治,维护世界和平,促进人类进步,推动构建人类命运共同体,推动建设持久和平、普遍安全、共同繁荣、开放包容、清洁美丽的世界。在互相尊重主权和领土完整、互不侵犯、互不干涉内政、平等互利、和平共处五项原则的基础上,发展我国同世界各国的关系。不断发展我国同周边国家的睦邻友好关系,加强同发展中国家的团结与合作。遵循共商共建共享原则,推进“一带一路”建设。按照独立自主、完全平等、互相尊重、互不干涉内部事务的原则,发展我党同各国共产党和其他政党的关系。

  中国共产党要领导全国各族人民实现第二个百年奋斗目标、实现中华民族伟大复兴的中国梦,必须紧密围绕党的基本路线,坚持和加强党的全面领导,坚持党要管党、全面从严治党,弘扬坚持真理、坚守理想,践行初心、担当使命,不怕牺牲、英勇斗争,对党忠诚、不负人民的伟大建党精神,加强党的长期执政能力建设、先进性和纯洁性建设,以改革创新精神全面推进党的建设新的伟大工程,以党的政治建设为统领,全面推进党的政治建设、思想建设、组织建设、作风建设、纪律建设,把制度建设贯穿其中,深入推进反腐败斗争,全面提高党的建设科学化水平,以伟大自我革命引领伟大社会革命。坚持立党为公、执政为民,发扬党的优良传统和作风,不断提高党的领导水平和执政水平,提高拒腐防变和抵御风险的能力,不断增强自我净化、自我完善、自我革新、自我提高能力,不断增强党的阶级基础和扩大党的群众基础,不断提高党的创造力、凝聚力、战斗力,建设学习型、服务型、创新型的马克思主义执政党,使我们党始终走在时代前列,成为领导全国人民沿着中国特色社会主义道路不断前进的坚强核心。党的建设必须坚决实现以下六项基本要求:

  第一,坚持党的基本路线。全党要用邓小平理论、“三个代表”重要思想、科学发展观、习近平新时代中国特色社会主义思想和党的基本路线统一思想,统一行动,并且毫不动摇地长期坚持下去。必须把改革开放同四项基本原则统一起来,全面落实党的基本路线,反对一切“左”的和右的错误倾向,要警惕右,但主要是防止“左”。必须提高政治判断力、政治领悟力、政治执行力,增强贯彻落实党的理论和路线方针政策的自觉性和坚定性。

  第二,坚持解放思想,实事求是,与时俱进,求真务实。党的思想路线是一切从实际出发,理论联系实际,实事求是,在实践中检验真理和发展真理。全党必须坚持这条思想路线,积极探索,大胆试验,开拓创新,创造性地开展工作,不断研究新情况,总结新经验,解决新问题,在实践中丰富和发展马克思主义,推进马克思主义中国化时代化。

  第三,坚持新时代党的组织路线。全面贯彻习近平新时代中国特色社会主义思想,以组织体系建设为重点,着力培养忠诚干净担当的高素质干部,着力集聚爱国奉献的各方面优秀人才,坚持德才兼备、以德为先、任人唯贤,为坚持和加强党的全面领导、坚持和发展中国特色社会主义提供坚强组织保证。全党必须增强党组织的政治功能和组织功能,培养选拔党和人民需要的好干部,培养和造就大批堪当时代重任的社会主义事业接班人,聚天下英才而用之,从组织上保证党的基本理论、基本路线、基本方略的贯彻落实。

  第四,坚持全心全意为人民服务。党除了工人阶级和最广大人民群众的利益,没有自己特殊的利益。党在任何时候都把群众利益放在第一位,同群众同甘共苦,保持最密切的联系,坚持权为民所用、情为民所系、利为民所谋,不允许任何党员脱离群众,凌驾于群众之上。我们党的最大政治优势是密切联系群众,党执政后的最大危险是脱离群众。党风问题、党同人民群众联系问题是关系党生死存亡的问题。党在自己的工作中实行群众路线,一切为了群众,一切依靠群众,从群众中来,到群众中去,把党的正确主张变为群众的自觉行动。

  第五,坚持民主集中制。民主集中制是民主基础上的集中和集中指导下的民主相结合。它既是党的根本组织原则,也是群众路线在党的生活中的运用。必须充分发扬党内民主,尊重党员主体地位,保障党员民主权利,发挥各级党组织和广大党员的积极性创造性。必须实行正确的集中,牢固树立政治意识、大局意识、核心意识、看齐意识,坚定维护以习近平同志为核心的党中央权威和集中统一领导,保证全党的团结统一和行动一致,保证党的决定得到迅速有效的贯彻执行。加强和规范党内政治生活,增强党内政治生活的政治性、时代性、原则性、战斗性,发展积极健康的党内政治文化,营造风清气正的良好政治生态。党在自己的政治生活中正确地开展批评和自我批评,在原则问题上进行思想斗争,坚持真理,修正错误。努力造成又有集中又有民主,又有纪律又有自由,又有统一意志又有个人心情舒畅生动活泼的政治局面。

  第六,坚持从严管党治党。全面从严治党永远在路上,党的自我革命永远在路上。新形势下,党面临的执政考验、改革开放考验、市场经济考验、外部环境考验是长期的、复杂的、严峻的,精神懈怠危险、能力不足危险、脱离群众危险、消极腐败危险更加尖锐地摆在全党面前。要把严的标准、严的措施贯穿于管党治党全过程和各方面。坚持依规治党、标本兼治,不断健全党内法规体系,坚持把纪律挺在前面,加强组织性纪律性,在党的纪律面前人人平等。强化全面从严治党主体责任和监督责任,加强对党的领导机关和党员领导干部特别是主要领导干部的监督,不断完善党内监督体系。深入推进党风廉政建设和反腐败斗争,以零容忍态度惩治腐败,一体推进不敢腐、不能腐、不想腐。

  中国共产党的领导是中国特色社会主义最本质的特征,是中国特色社会主义制度的最大优势,党是最高政治领导力量。党政军民学,东西南北中,党是领导一切的。党要适应改革开放和社会主义现代化建设的要求,坚持科学执政、民主执政、依法执政,加强和改善党的领导。党必须按照总揽全局、协调各方的原则,在同级各种组织中发挥领导核心作用。党必须集中精力领导经济建设,组织、协调各方面的力量,同心协力,围绕经济建设开展工作,促进经济社会全面发展。党必须实行民主的科学的决策,制定和执行正确的路线、方针、政策,做好党的组织工作和宣传教育工作,发挥全体党员的先锋模范作用。党必须在宪法和法律的范围内活动。党必须保证国家的立法、司法、行政、监察机关,经济、文化组织和人民团体积极主动地、独立负责地、协调一致地工作。党必须加强对工会、共产主义青年团、妇女联合会等群团组织的领导,使它们保持和增强政治性、先进性、群众性,充分发挥作用。党必须适应形势的发展和情况的变化,完善领导体制,改进领导方式,增强执政能力。共产党员必须同党外群众亲密合作,共同为建设中国特色社会主义而奋斗。

    \subsection{党员}
  第一条 年满十八岁的中国工人、农民、军人、知识分子和其他社会阶层的先进分子,承认党的纲领和章程,愿意参加党的一个组织并在其中积极工作、执行党的决议和按期交纳党费的,可以申请加入中国共产党。

  第二条 中国共产党党员是中国工人阶级的有共产主义觉悟的先锋战士。

  中国共产党党员必须全心全意为人民服务,不惜牺牲个人的一切,为实现共产主义奋斗终身。

  中国共产党党员永远是劳动人民的普通一员。除了法律和政策规定范围内的个人利益和工作职权以外,所有共产党员都不得谋求任何私利和特权。

  第三条 党员必须履行下列义务:

  (一)认真学习马克思列宁主义、毛泽东思想、邓小平理论、“三个代表”重要思想、科学发展观、习近平新时代中国特色社会主义思想,学习党的路线、方针、政策和决议,学习党的基本知识和党的历史,学习科学、文化、法律和业务知识,努力提高为人民服务的本领。

  (二)增强“四个意识”、坚定“四个自信”、做到“两个维护”,贯彻执行党的基本路线和各项方针、政策,带头参加改革开放和社会主义现代化建设,带动群众为经济发展和社会进步艰苦奋斗,在生产、工作、学习和社会生活中起先锋模范作用。

  (三)坚持党和人民的利益高于一切,个人利益服从党和人民的利益,吃苦在前,享受在后,克己奉公,多做贡献。

  (四)自觉遵守党的纪律,首先是党的政治纪律和政治规矩,模范遵守国家的法律法规,严格保守党和国家的秘密,执行党的决定,服从组织分配,积极完成党的任务。

  (五)维护党的团结和统一,对党忠诚老实,言行一致,坚决反对一切派别组织和小集团活动,反对阳奉阴违的两面派行为和一切阴谋诡计。

  (六)切实开展批评和自我批评,勇于揭露和纠正违反党的原则的言行和工作中的缺点、错误,坚决同消极腐败现象作斗争。

  (七)密切联系群众,向群众宣传党的主张,遇事同群众商量,及时向党反映群众的意见和要求,维护群众的正当利益。

  (八)发扬社会主义新风尚,带头实践社会主义核心价值观和社会主义荣辱观,提倡共产主义道德,弘扬中华民族传统美德,为了保护国家和人民的利益,在一切困难和危险的时刻挺身而出,英勇斗争,不怕牺牲。

  第四条 党员享有下列权利:

  (一)参加党的有关会议,阅读党的有关文件,接受党的教育和培训。

  (二)在党的会议上和党报党刊上,参加关于党的政策问题的讨论。

  (三)对党的工作提出建议和倡议。

  (四)在党的会议上有根据地批评党的任何组织和任何党员,向党负责地揭发、检举党的任何组织和任何党员违法乱纪的事实,要求处分违法乱纪的党员,要求罢免或撤换不称职的干部。

  (五)行使表决权、选举权,有被选举权。

  (六)在党组织讨论决定对党员的党纪处分或作出鉴定时,本人有权参加和进行申辩,其他党员可以为他作证和辩护。

  (七)对党的决议和政策如有不同意见,在坚决执行的前提下,可以声明保留,并且可以把自己的意见向党的上级组织直至中央提出。

  (八)向党的上级组织直至中央提出请求、申诉和控告,并要求有关组织给以负责的答复。

  党的任何一级组织直至中央都无权剥夺党员的上述权利。

  第五条 发展党员,必须把政治标准放在首位,经过党的支部,坚持个别吸收的原则。

  申请入党的人,要填写入党志愿书,要有两名正式党员作介绍人,要经过支部大会通过和上级党组织批准,并且经过预备期的考察,才能成为正式党员。

  介绍人要认真了解申请人的思想、品质、经历和工作表现,向他解释党的纲领和党的章程,说明党员的条件、义务和权利,并向党组织作出负责的报告。

  党的支部委员会对申请入党的人,要注意征求党内外有关群众的意见,进行严格的审查,认为合格后再提交支部大会讨论。

  上级党组织在批准申请人入党以前,要派人同他谈话,作进一步的了解,并帮助他提高对党的认识。

  在特殊情况下,党的中央和省、自治区、直辖市委员会可以直接接收党员。

  第六条 预备党员必须面向党旗进行入党宣誓。誓词如下:我志愿加入中国共产党,拥护党的纲领,遵守党的章程,履行党员义务,执行党的决定,严守党的纪律,保守党的秘密,对党忠诚,积极工作,为共产主义奋斗终身,随时准备为党和人民牺牲一切,永不叛党。

  第七条 预备党员的预备期为一年。党组织对预备党员应当认真教育和考察。

  预备党员的义务同正式党员一样。预备党员的权利,除了没有表决权、选举权和被选举权以外,也同正式党员一样。

  预备党员预备期满,党的支部应当及时讨论他能否转为正式党员。认真履行党员义务,具备党员条件的,应当按期转为正式党员;需要继续考察和教育的,可以延长预备期,但不能超过一年;不履行党员义务,不具备党员条件的,应当取消预备党员资格。预备党员转为正式党员,或延长预备期,或取消预备党员资格,都应当经支部大会讨论通过和上级党组织批准。

  预备党员的预备期,从支部大会通过他为预备党员之日算起。党员的党龄,从预备期满转为正式党员之日算起。

  第八条 每个党员,不论职务高低,都必须编入党的一个支部、小组或其他特定组织,参加党的组织生活,接受党内外群众的监督。党员领导干部还必须参加党委、党组的民主生活会。不允许有任何不参加党的组织生活、不接受党内外群众监督的特殊党员。

  第九条 党员有退党的自由。党员要求退党,应当经支部大会讨论后宣布除名,并报上级党组织备案。

  党员缺乏革命意志,不履行党员义务,不符合党员条件,党的支部应当对他进行教育,要求他限期改正;经教育仍无转变的,应当劝他退党。劝党员退党,应当经支部大会讨论决定,并报上级党组织批准。如被劝告退党的党员坚持不退,应当提交支部大会讨论,决定把他除名,并报上级党组织批准。

  党员如果没有正当理由,连续六个月不参加党的组织生活,或不交纳党费,或不做党所分配的工作,就被认为是自行脱党。支部大会应当决定把这样的党员除名,并报上级党组织批准。

    \subsection{党的组织制度}
  第十条 党是根据自己的纲领和章程,按照民主集中制组织起来的统一整体。党的民主集中制的基本原则是:

  (一)党员个人服从党的组织,少数服从多数,下级组织服从上级组织,全党各个组织和全体党员服从党的全国代表大会和中央委员会。

  (二)党的各级领导机关,除它们派出的代表机关和在非党组织中的党组外,都由选举产生。

  (三)党的最高领导机关,是党的全国代表大会和它所产生的中央委员会。党的地方各级领导机关,是党的地方各级代表大会和它们所产生的委员会。党的各级委员会向同级的代表大会负责并报告工作。

  (四)党的上级组织要经常听取下级组织和党员群众的意见,及时解决他们提出的问题。党的下级组织既要向上级组织请示和报告工作,又要独立负责地解决自己职责范围内的问题。上下级组织之间要互通情报、互相支持和互相监督。党的各级组织要按规定实行党务公开,使党员对党内事务有更多的了解和参与。

  (五)党的各级委员会实行集体领导和个人分工负责相结合的制度。凡属重大问题都要按照集体领导、民主集中、个别酝酿、会议决定的原则,由党的委员会集体讨论,作出决定;委员会成员要根据集体的决定和分工,切实履行自己的职责。

  (六)党禁止任何形式的个人崇拜。要保证党的领导人的活动处于党和人民的监督之下,同时维护一切代表党和人民利益的领导人的威信。

  第十一条 党的各级代表大会的代表和委员会的产生,要体现选举人的意志。选举采用无记名投票的方式。候选人名单要由党组织和选举人充分酝酿讨论。可以直接采用候选人数多于应选人数的差额选举办法进行正式选举。也可以先采用差额选举办法进行预选,产生候选人名单,然后进行正式选举。选举人有了解候选人情况、要求改变候选人、不选任何一个候选人和另选他人的权利。任何组织和个人不得以任何方式强迫选举人选举或不选举某个人。

  党的地方各级代表大会和基层代表大会的选举,如果发生违反党章的情况,上一级党的委员会在调查核实后,应作出选举无效和采取相应措施的决定,并报再上一级党的委员会审查批准,正式宣布执行。

  党的各级代表大会代表实行任期制。

  第十二条 党的中央和地方各级委员会在必要时召集代表会议,讨论和决定需要及时解决的重大问题。代表会议代表的名额和产生办法,由召集代表会议的委员会决定。

  第十三条 凡是成立党的新组织,或是撤销党的原有组织,必须由上级党组织决定。

  在党的地方各级代表大会和基层代表大会闭会期间,上级党的组织认为有必要时,可以调动或者指派下级党组织的负责人。

  党的中央和地方各级委员会可以派出代表机关。

  第十四条 党的中央和省、自治区、直辖市委员会实行巡视制度,在一届任期内,对所管理的地方、部门、企事业单位党组织实现巡视全覆盖。

  中央有关部委和国家机关部门党组(党委)根据工作需要,开展巡视工作。

  党的市(地、州、盟)和县(市、区、旗)委员会建立巡察制度。

  第十五条 党的各级领导机关,对同下级组织有关的重要问题作出决定时,在通常情况下,要征求下级组织的意见。要保证下级组织能够正常行使他们的职权。凡属应由下级组织处理的问题,如无特殊情况,上级领导机关不要干预。

  第十六条 有关全国性的重大政策问题,只有党中央有权作出决定,各部门、各地方的党组织可以向中央提出建议,但不得擅自作出决定和对外发表主张。

  党的下级组织必须坚决执行上级组织的决定。下级组织如果认为上级组织的决定不符合本地区、本部门的实际情况,可以请求改变;如果上级组织坚持原决定,下级组织必须执行,并不得公开发表不同意见,但有权向再上一级组织报告。

  党的各级组织的报刊和其他宣传工具,必须宣传党的路线、方针、政策和决议。

  第十七条 党组织讨论决定问题,必须执行少数服从多数的原则。决定重要问题,要进行表决。对于少数人的不同意见,应当认真考虑。如对重要问题发生争论,双方人数接近,除了在紧急情况下必须按多数意见执行外,应当暂缓作出决定,进一步调查研究,交换意见,下次再表决;在特殊情况下,也可将争论情况向上级组织报告,请求裁决。

  党员个人代表党组织发表重要主张,如果超出党组织已有决定的范围,必须提交所在的党组织讨论决定,或向上级党组织请示。任何党员不论职务高低,都不能个人决定重大问题;如遇紧急情况,必须由个人作出决定时,事后要迅速向党组织报告。不允许任何领导人实行个人专断和把个人凌驾于组织之上。

  第十八条 党的中央、地方和基层组织,都必须重视党的建设,经常讨论和检查党的宣传工作、教育工作、组织工作、纪律检查工作、群众工作、统一战线工作等,注意研究党内外的思想政治状况。

    \subsection{党的中央组织}
  第十九条 党的全国代表大会每五年举行一次,由中央委员会召集。中央委员会认为有必要,或者有三分之一以上的省一级组织提出要求,全国代表大会可以提前举行;如无非常情况,不得延期举行。

  全国代表大会代表的名额和选举办法,由中央委员会决定。

  第二十条 党的全国代表大会的职权是:

  (一)听取和审查中央委员会的报告;

  (二)审查中央纪律检查委员会的报告;

  (三)讨论并决定党的重大问题;

  (四)修改党的章程;

  (五)选举中央委员会;

  (六)选举中央纪律检查委员会。

  第二十一条 党的全国代表会议的职权是:讨论和决定重大问题;调整和增选中央委员会、中央纪律检查委员会的部分成员。调整和增选中央委员及候补中央委员的数额,不得超过党的全国代表大会选出的中央委员及候补中央委员各自总数的五分之一。

  第二十二条 党的中央委员会每届任期五年。全国代表大会如提前或延期举行,它的任期相应地改变。中央委员会委员和候补委员必须有五年以上的党龄。中央委员会委员和候补委员的名额,由全国代表大会决定。中央委员会委员出缺,由中央委员会候补委员按照得票多少依次递补。

  中央委员会全体会议由中央政治局召集,每年至少举行一次。中央政治局向中央委员会全体会议报告工作,接受监督。

  在全国代表大会闭会期间,中央委员会执行全国代表大会的决议,领导党的全部工作,对外代表中国共产党。

  第二十三条 党的中央政治局、中央政治局常务委员会和中央委员会总书记,由中央委员会全体会议选举。中央委员会总书记必须从中央政治局常务委员会委员中产生。

  中央政治局和它的常务委员会在中央委员会全体会议闭会期间,行使中央委员会的职权。

  中央书记处是中央政治局和它的常务委员会的办事机构;成员由中央政治局常务委员会提名,中央委员会全体会议通过。

  中央委员会总书记负责召集中央政治局会议和中央政治局常务委员会会议,并主持中央书记处的工作。

  党的中央军事委员会组成人员由中央委员会决定,中央军事委员会实行主席负责制。

  每届中央委员会产生的中央领导机构和中央领导人,在下届全国代表大会开会期间,继续主持党的经常工作,直到下届中央委员会产生新的中央领导机构和中央领导人为止。

  第二十四条 中国人民解放军的党组织,根据中央委员会的指示进行工作。中央军事委员会负责军队中党的工作和政治工作,对军队中党的组织体制和机构作出规定。

    \subsection{党的地方组织}
  第二十五条 党的省、自治区、直辖市的代表大会,设区的市和自治州的代表大会,县(旗)、自治县、不设区的市和市辖区的代表大会,每五年举行一次。

  党的地方各级代表大会由同级党的委员会召集。在特殊情况下,经上一级委员会批准,可以提前或延期举行。

  党的地方各级代表大会代表的名额和选举办法,由同级党的委员会决定,并报上一级党的委员会批准。

  第二十六条 党的地方各级代表大会的职权是:

  (一)听取和审查同级委员会的报告;

  (二)审查同级纪律检查委员会的报告;

  (三)讨论本地区范围内的重大问题并作出决议;

  (四)选举同级党的委员会,选举同级党的纪律检查委员会。

  第二十七条 党的省、自治区、直辖市、设区的市和自治州的委员会,每届任期五年。这些委员会的委员和候补委员必须有五年以上的党龄。

  党的县(旗)、自治县、不设区的市和市辖区的委员会,每届任期五年。这些委员会的委员和候补委员必须有三年以上的党龄。

  党的地方各级代表大会如提前或延期举行,由它选举的委员会的任期相应地改变。

  党的地方各级委员会的委员和候补委员的名额,分别由上一级委员会决定。党的地方各级委员会委员出缺,由候补委员按照得票多少依次递补。

  党的地方各级委员会全体会议,每年至少召开两次。

  党的地方各级委员会在代表大会闭会期间,执行上级党组织的指示和同级党代表大会的决议,领导本地方的工作,定期向上级党的委员会报告工作。

  第二十八条 党的地方各级委员会全体会议,选举常务委员会和书记、副书记,并报上级党的委员会批准。党的地方各级委员会的常务委员会,在委员会全体会议闭会期间,行使委员会职权;在下届代表大会开会期间,继续主持经常工作,直到新的常务委员会产生为止。

  党的地方各级委员会的常务委员会定期向委员会全体会议报告工作,接受监督。

  第二十九条 党的地区委员会和相当于地区委员会的组织,是党的省、自治区委员会在几个县、自治县、市范围内派出的代表机关。它根据省、自治区委员会的授权,领导本地区的工作。

    \subsection{党的基层组织}
  第三十条 企业、农村、机关、学校、医院、科研院所、街道社区、社会组织、人民解放军连队和其他基层单位,凡是有正式党员三人以上的,都应当成立党的基层组织。

  党的基层组织,根据工作需要和党员人数,经上级党组织批准,分别设立党的基层委员会、总支部委员会、支部委员会。基层委员会由党员大会或代表大会选举产生,总支部委员会和支部委员会由党员大会选举产生,提出委员候选人要广泛征求党员和群众的意见。

  第三十一条 党的基层委员会、总支部委员会、支部委员会每届任期三年至五年。基层委员会、总支部委员会、支部委员会的书记、副书记选举产生后,应报上级党组织批准。

  第三十二条 党的基层组织是党在社会基层组织中的战斗堡垒,是党的全部工作和战斗力的基础。它的基本任务是:

  (一)宣传和执行党的路线、方针、政策,宣传和执行党中央、上级组织和本组织的决议,充分发挥党员的先锋模范作用,积极创先争优,团结、组织党内外的干部和群众,努力完成本单位所担负的任务。

  (二)组织党员认真学习马克思列宁主义、毛泽东思想、邓小平理论、“三个代表”重要思想、科学发展观、习近平新时代中国特色社会主义思想,推进“两学一做”学习教育、党史学习教育常态化制度化,学习党的路线、方针、政策和决议,学习党的基本知识,学习科学、文化、法律和业务知识。

  (三)对党员进行教育、管理、监督和服务,提高党员素质,坚定理想信念,增强党性,严格党的组织生活,开展批评和自我批评,维护和执行党的纪律,监督党员切实履行义务,保障党员的权利不受侵犯。加强和改进流动党员管理。

  (四)密切联系群众,经常了解群众对党员、党的工作的批评和意见,维护群众的正当权利和利益,做好群众的思想政治工作。

  (五)充分发挥党员和群众的积极性创造性,发现、培养和推荐他们中间的优秀人才,鼓励和支持他们在改革开放和社会主义现代化建设中贡献自己的聪明才智。

  (六)对要求入党的积极分子进行教育和培养,做好经常性的发展党员工作,重视在生产、工作第一线和青年中发展党员。

  (七)监督党员干部和其他任何工作人员严格遵守国家法律法规,严格遵守国家的财政经济法规和人事制度,不得侵占国家、集体和群众的利益。

  (八)教育党员和群众自觉抵制不良倾向,坚决同各种违纪违法行为作斗争。

  第三十三条 街道、乡、镇党的基层委员会和村、社区党组织,统一领导本地区基层各类组织和各项工作,加强基层社会治理,支持和保证行政组织、经济组织和群众性自治组织充分行使职权。

  国有企业党委(党组)发挥领导作用,把方向、管大局、保落实,依照规定讨论和决定企业重大事项。国有企业和集体企业中党的基层组织,围绕企业生产经营开展工作。保证监督党和国家的方针、政策在本企业的贯彻执行;支持股东会、董事会、监事会和经理(厂长)依法行使职权;全心全意依靠职工群众,支持职工代表大会开展工作;参与企业重大问题的决策;加强党组织的自身建设,领导思想政治工作、精神文明建设、统一战线工作和工会、共青团、妇女组织等群团组织。

  非公有制经济组织中党的基层组织,贯彻党的方针政策,引导和监督企业遵守国家的法律法规,领导工会、共青团等群团组织,团结凝聚职工群众,维护各方的合法权益,促进企业健康发展。

  社会组织中党的基层组织,宣传和执行党的路线、方针、政策,领导工会、共青团等群团组织,教育管理党员,引领服务群众,推动事业发展。

  实行行政领导人负责制的事业单位中党的基层组织,发挥战斗堡垒作用。实行党委领导下的行政领导人负责制的事业单位中党的基层组织,对重大问题进行讨论和作出决定,同时保证行政领导人充分行使自己的职权。

  各级党和国家机关中党的基层组织,协助行政负责人完成任务,改进工作,对包括行政负责人在内的每个党员进行教育、管理、监督,不领导本单位的业务工作。

  第三十四条 党支部是党的基础组织,担负直接教育党员、管理党员、监督党员和组织群众、宣传群众、凝聚群众、服务群众的职责。

    \subsection{党的干部}
  第三十五条 党的干部是党的事业的骨干,是人民的公仆,要做到忠诚干净担当。党按照德才兼备、以德为先的原则选拔干部,坚持五湖四海、任人唯贤,坚持事业为上、公道正派,反对任人唯亲,努力实现干部队伍的革命化、年轻化、知识化、专业化。

  党重视教育、培训、选拔、考核和监督干部,特别是培养、选拔优秀年轻干部。积极推进干部制度改革。

  党重视培养、选拔女干部和少数民族干部。

  第三十六条 党的各级领导干部必须信念坚定、为民服务、勤政务实、敢于担当、清正廉洁,模范地履行本章程第三条所规定的党员的各项义务,并且必须具备以下的基本条件:

  (一)具有履行职责所需要的马克思列宁主义、毛泽东思想、邓小平理论、“三个代表”重要思想、科学发展观的水平,带头贯彻落实习近平新时代中国特色社会主义思想,努力用马克思主义的立场、观点、方法分析和解决实际问题,坚持讲学习、讲政治、讲正气,经得起各种风浪的考验。

  (二)具有共产主义远大理想和中国特色社会主义坚定信念,坚决执行党的基本路线和各项方针、政策,立志改革开放,献身现代化事业,在社会主义建设中艰苦创业,树立正确政绩观,做出经得起实践、人民、历史检验的实绩。

  (三)坚持解放思想,实事求是,与时俱进,开拓创新,认真调查研究,能够把党的方针、政策同本地区、本部门的实际相结合,卓有成效地开展工作,讲实话,办实事,求实效。

  (四)有强烈的革命事业心和政治责任感,有实践经验,有胜任领导工作的组织能力、文化水平和专业知识。

  (五)正确行使人民赋予的权力,坚持原则,依法办事,清正廉洁,勤政为民,以身作则,艰苦朴素,密切联系群众,坚持党的群众路线,自觉地接受党和群众的批评和监督,加强道德修养,讲党性、重品行、作表率,做到自重、自省、自警、自励,反对形式主义、官僚主义、享乐主义和奢靡之风,反对特权思想和特权现象,反对任何滥用职权、谋求私利的行为。

  (六)坚持和维护党的民主集中制,有民主作风,有全局观念,善于团结同志,包括团结同自己有不同意见的同志一道工作。

  第三十七条 党员干部要善于同党外干部合作共事,尊重他们,虚心学习他们的长处。

  党的各级组织要善于发现和推荐有真才实学的党外干部担任领导工作,保证他们有职有权,充分发挥他们的作用。

  第三十八条 党的各级领导干部,无论是由民主选举产生的,或是由领导机关任命的,他们的职务都不是终身的,都可以变动或解除。

  年龄和健康状况不适宜于继续担任工作的干部,应当按照国家的规定退、离休。

    \subsection{党的纪律}
  第三十九条 党的纪律是党的各级组织和全体党员必须遵守的行为规则,是维护党的团结统一、完成党的任务的保证。党组织必须严格执行和维护党的纪律,共产党员必须自觉接受党的纪律的约束。

  第四十条 党的纪律主要包括政治纪律、组织纪律、廉洁纪律、群众纪律、工作纪律、生活纪律。

  坚持惩前毖后、治病救人,执纪必严、违纪必究,抓早抓小、防微杜渐,按照错误性质和情节轻重,给以批评教育、责令检查、诫勉直至纪律处分。运用监督执纪“四种形态”,让“红红脸、出出汗”成为常态,党纪处分、组织调整成为管党治党的重要手段,严重违纪、严重触犯刑律的党员必须开除党籍。

  党内严格禁止用违反党章和国家法律的手段对待党员,严格禁止打击报复和诬告陷害。违反这些规定的组织或个人必须受到党的纪律和国家法律的追究。

  第四十一条 对党员的纪律处分有五种:警告、严重警告、撤销党内职务、留党察看、开除党籍。

  留党察看最长不超过两年。党员在留党察看期间没有表决权、选举权和被选举权。党员经过留党察看,确已改正错误的,应当恢复其党员的权利;坚持错误不改的,应当开除党籍。

  开除党籍是党内的最高处分。各级党组织在决定或批准开除党员党籍的时候,应当全面研究有关的材料和意见,采取十分慎重的态度。

  第四十二条 对党员的纪律处分,必须经过支部大会讨论决定,报党的基层委员会批准;如果涉及的问题比较重要或复杂,或给党员以开除党籍的处分,应分别不同情况,报县级或县级以上党的纪律检查委员会审查批准。在特殊情况下,县级和县级以上各级党的委员会和纪律检查委员会有权直接决定给党员以纪律处分。

  对党的中央委员会委员、候补委员,给以警告、严重警告处分,由中央纪律检查委员会常务委员会审议后,报党中央批准。对地方各级党的委员会委员、候补委员,给以警告、严重警告处分,应由上一级纪律检查委员会批准,并报它的同级党的委员会备案。

  对党的中央委员会和地方各级委员会的委员、候补委员,给以撤销党内职务、留党察看或开除党籍的处分,必须由本人所在的委员会全体会议三分之二以上的多数决定。在全体会议闭会期间,可以先由中央政治局和地方各级委员会常务委员会作出处理决定,待召开委员会全体会议时予以追认。对地方各级委员会委员和候补委员的上述处分,必须经过上级纪律检查委员会常务委员会审议,由这一级纪律检查委员会报同级党的委员会批准。

  严重触犯刑律的中央委员会委员、候补委员,由中央政治局决定开除其党籍;严重触犯刑律的地方各级委员会委员、候补委员,由同级委员会常务委员会决定开除其党籍。

  第四十三条 党组织对党员作出处分决定,应当实事求是地查清事实。处分决定所依据的事实材料和处分决定必须同本人见面,听取本人说明情况和申辩。如果本人对处分决定不服,可以提出申诉,有关党组织必须负责处理或者迅速转递,不得扣压。对于确属坚持错误意见和无理要求的人,要给以批评教育。

  第四十四条 党组织如果在维护党的纪律方面失职,必须问责。

  对于严重违犯党的纪律、本身又不能纠正的党组织,上一级党的委员会在查明核实后,应根据情节严重的程度,作出进行改组或予以解散的决定,并报再上一级党的委员会审查批准,正式宣布执行。

    \subsection{党的纪律检查机关}
  第四十五条 党的中央纪律检查委员会在党的中央委员会领导下进行工作。党的地方各级纪律检查委员会和基层纪律检查委员会在同级党的委员会和上级纪律检查委员会双重领导下进行工作。上级党的纪律检查委员会加强对下级纪律检查委员会的领导。

  党的各级纪律检查委员会每届任期和同级党的委员会相同。

  党的中央纪律检查委员会全体会议,选举常务委员会和书记、副书记,并报党的中央委员会批准。党的地方各级纪律检查委员会全体会议,选举常务委员会和书记、副书记,并由同级党的委员会通过,报上级党的委员会批准。党的基层委员会是设立纪律检查委员会,还是设立纪律检查委员,由它的上一级党组织根据具体情况决定。党的总支部委员会和支部委员会设纪律检查委员。

  党的中央和地方纪律检查委员会向同级党和国家机关全面派驻党的纪律检查组,按照规定向有关国有企业、事业单位派驻党的纪律检查组。纪律检查组组长参加驻在单位党的领导组织的有关会议。他们的工作必须受到该单位党的领导组织的支持。

  第四十六条 党的各级纪律检查委员会是党内监督专责机关,主要任务是:维护党的章程和其他党内法规,检查党的路线、方针、政策和决议的执行情况,协助党的委员会推进全面从严治党、加强党风建设和组织协调反腐败工作,推动完善党和国家监督体系。

  党的各级纪律检查委员会的职责是监督、执纪、问责,要经常对党员进行遵守纪律的教育,作出关于维护党纪的决定;对党的组织和党员领导干部履行职责、行使权力进行监督,受理处置党员群众检举举报,开展谈话提醒、约谈函询;检查和处理党的组织和党员违反党的章程和其他党内法规的比较重要或复杂的案件,决定或取消对这些案件中的党员的处分;进行问责或提出责任追究的建议;受理党员的控告和申诉;保障党员的权利。

  各级纪律检查委员会要把处理特别重要或复杂的案件中的问题和处理的结果,向同级党的委员会报告。党的地方各级纪律检查委员会和基层纪律检查委员会要同时向上级纪律检查委员会报告。

  各级纪律检查委员会发现同级党的委员会委员有违犯党的纪律的行为,可以先进行初步核实,如果需要立案检查的,应当在向同级党的委员会报告的同时向上一级纪律检查委员会报告;涉及常务委员的,报告上一级纪律检查委员会,由上一级纪律检查委员会进行初步核实,需要审查的,由上一级纪律检查委员会报它的同级党的委员会批准。

  第四十七条 上级纪律检查委员会有权检查下级纪律检查委员会的工作,并且有权批准和改变下级纪律检查委员会对于案件所作的决定。如果所要改变的该下级纪律检查委员会的决定,已经得到它的同级党的委员会的批准,这种改变必须经过它的上一级党的委员会批准。

  党的地方各级纪律检查委员会和基层纪律检查委员会如果对同级党的委员会处理案件的决定有不同意见,可以请求上一级纪律检查委员会予以复查;如果发现同级党的委员会或它的成员有违犯党的纪律的情况,在同级党的委员会不给予解决或不给予正确解决的时候,有权向上级纪律检查委员会提出申诉,请求协助处理。

    \subsection{党组}
  第四十八条 在中央和地方国家机关、人民团体、经济组织、文化组织和其他非党组织的领导机关中,可以成立党组。党组发挥领导作用。党组的任务,主要是负责贯彻执行党的路线、方针、政策;加强对本单位党的建设的领导,履行全面从严治党责任;讨论和决定本单位的重大问题;做好干部管理工作;讨论和决定基层党组织设置调整和发展党员、处分党员等重要事项;团结党外干部和群众,完成党和国家交给的任务;领导机关和直属单位党组织的工作。

  第四十九条 党组的成员,由批准成立党组的党组织决定。党组设书记,必要时还可以设副书记。

  党组必须服从批准它成立的党组织领导。

  第五十条 在对下属单位实行集中统一领导的国家工作部门和有关单位的领导机关中,可以建立党委,党委的产生办法、职权和工作任务,由中央另行规定。

    \subsection{党和共产主义青年团的关系}
  第五十一条 中国共产主义青年团是中国共产党领导的先进青年的群团组织,是广大青年在实践中学习中国特色社会主义和共产主义的学校,是党的助手和后备军。共青团中央委员会受党中央委员会领导。共青团的地方各级组织受同级党的委员会领导,同时受共青团上级组织领导。

  第五十二条 党的各级委员会要加强对共青团的领导,注意团的干部的选拔和培训。党要坚决支持共青团根据广大青年的特点和需要,生动活泼地、富于创造性地进行工作,充分发挥团的突击队作用和联系广大青年的桥梁作用。

  团的县级和县级以下各级委员会书记,企业事业单位的团委员会书记,是党员的,可以列席同级党的委员会和常务委员会的会议。

    \subsection{党徽党旗}
  第五十三条 中国共产党党徽为镰刀和锤头组成的图案。

  第五十四条 中国共产党党旗为旗面缀有金黄色党徽图案的红旗。

  第五十五条 中国共产党的党徽党旗是中国共产党的象征和标志。党的各级组织和每一个党员都要维护党徽党旗的尊严。要按照规定制作和使用党徽党旗。
    \section{中共中央关于进一步全面深化改革 推进中国式现代化的决定}
    (2024年7月18日中国共产党第二十届中央委员会第三次全体会议通过)

    为贯彻落实党的二十大作出的战略部署,二十届中央委员会第三次全体会议研究了进一步全面深化改革、推进中国式现代化问题,作出如下决定。

    \subsection{进一步全面深化改革、推进中国式现代化的重大意义和总体要求}

    (1)进一步全面深化改革的重要性和必要性。改革开放是党和人民事业大踏步赶上时代的重要法宝。党的十一届三中全会是划时代的,开启了改革开放和社会主义现代化建设新时期。党的十八届三中全会也是划时代的,开启了新时代全面深化改革、系统整体设计推进改革新征程,开创了我国改革开放全新局面。

    以习近平同志为核心的党中央团结带领全党全军全国各族人民,以伟大的历史主动、巨大的政治勇气、强烈的责任担当,冲破思想观念束缚,突破利益固化藩篱,敢于突进深水区,敢于啃硬骨头,敢于涉险滩,坚决破除各方面体制机制弊端,实现改革由局部探索、破冰突围到系统集成、全面深化的转变,各领域基础性制度框架基本建立,许多领域实现历史性变革、系统性重塑、整体性重构,总体完成党的十八届三中全会确定的改革任务,实现到党成立一百周年时各方面制度更加成熟更加定型取得明显成效的目标,为全面建成小康社会、实现党的第一个百年奋斗目标提供有力制度保障,推动我国迈上全面建设社会主义现代化国家新征程。

    当前和今后一个时期是以中国式现代化全面推进强国建设、民族复兴伟业的关键时期。中国式现代化是在改革开放中不断推进的,也必将在改革开放中开辟广阔前景。面对纷繁复杂的国际国内形势,面对新一轮科技革命和产业变革,面对人民群众新期待,必须继续把改革推向前进。这是坚持和完善中国特色社会主义制度、推进国家治理体系和治理能力现代化的必然要求,是贯彻新发展理念、更好适应我国社会主要矛盾变化的必然要求,是坚持以人民为中心、让现代化建设成果更多更公平惠及全体人民的必然要求,是应对重大风险挑战、推动党和国家事业行稳致远的必然要求,是推动构建人类命运共同体、在百年变局加速演进中赢得战略主动的必然要求,是深入推进新时代党的建设新的伟大工程、建设更加坚强有力的马克思主义政党的必然要求。改革开放只有进行时,没有完成时。全党必须自觉把改革摆在更加突出位置,紧紧围绕推进中国式现代化进一步全面深化改革。

    (2)进一步全面深化改革的指导思想。坚持马克思列宁主义、毛泽东思想、邓小平理论、“三个代表”重要思想、科学发展观,全面贯彻习近平新时代中国特色社会主义思想,深入学习贯彻习近平总书记关于全面深化改革的一系列新思想、新观点、新论断,完整准确全面贯彻新发展理念,坚持稳中求进工作总基调,坚持解放思想、实事求是、与时俱进、求真务实,进一步解放和发展社会生产力、激发和增强社会活力,统筹国内国际两个大局,统筹推进“五位一体”总体布局,协调推进“四个全面”战略布局,以经济体制改革为牵引,以促进社会公平正义、增进人民福祉为出发点和落脚点,更加注重系统集成,更加注重突出重点,更加注重改革实效,推动生产关系和生产力、上层建筑和经济基础、国家治理和社会发展更好相适应,为中国式现代化提供强大动力和制度保障。

    (3)进一步全面深化改革的总目标。继续完善和发展中国特色社会主义制度,推进国家治理体系和治理能力现代化。到二〇三五年,全面建成高水平社会主义市场经济体制,中国特色社会主义制度更加完善,基本实现国家治理体系和治理能力现代化,基本实现社会主义现代化,为到本世纪中叶全面建成社会主义现代化强国奠定坚实基础。

    ——聚焦构建高水平社会主义市场经济体制,充分发挥市场在资源配置中的决定性作用,更好发挥政府作用,坚持和完善社会主义基本经济制度,推进高水平科技自立自强,推进高水平对外开放,建成现代化经济体系,加快构建新发展格局,推动高质量发展。

    ——聚焦发展全过程人民民主,坚持党的领导、人民当家作主、依法治国有机统一,推动人民当家作主制度更加健全、协商民主广泛多层制度化发展、中国特色社会主义法治体系更加完善,社会主义法治国家建设达到更高水平。

    ——聚焦建设社会主义文化强国,坚持马克思主义在意识形态领域指导地位的根本制度,健全文化事业、文化产业发展体制机制,推动文化繁荣,丰富人民精神文化生活,提升国家文化软实力和中华文化影响力。

    ——聚焦提高人民生活品质,完善收入分配和就业制度,健全社会保障体系,增强基本公共服务均衡性和可及性,推动人的全面发展、全体人民共同富裕取得更为明显的实质性进展。

    ——聚焦建设美丽中国,加快经济社会发展全面绿色转型,健全生态环境治理体系,推进生态优先、节约集约、绿色低碳发展,促进人与自然和谐共生。

    ——聚焦建设更高水平平安中国,健全国家安全体系,强化一体化国家战略体系,增强维护国家安全能力,创新社会治理体制机制和手段,有效构建新安全格局。

    ——聚焦提高党的领导水平和长期执政能力,创新和改进领导方式和执政方式,深化党的建设制度改革,健全全面从严治党体系。

    到二〇二九年中华人民共和国成立八十周年时,完成本决定提出的改革任务。

    (4)进一步全面深化改革的原则。总结和运用改革开放以来特别是新时代全面深化改革的宝贵经验,贯彻以下原则:坚持党的全面领导,坚定维护党中央权威和集中统一领导,发挥党总揽全局、协调各方的领导核心作用,把党的领导贯穿改革各方面全过程,确保改革始终沿着正确政治方向前进;坚持以人民为中心,尊重人民主体地位和首创精神,人民有所呼、改革有所应,做到改革为了人民、改革依靠人民、改革成果由人民共享;坚持守正创新,坚持中国特色社会主义不动摇,紧跟时代步伐,顺应实践发展,突出问题导向,在新的起点上推进理论创新、实践创新、制度创新、文化创新以及其他各方面创新;坚持以制度建设为主线,加强顶层设计、总体谋划,破立并举、先立后破,筑牢根本制度,完善基本制度,创新重要制度;坚持全面依法治国,在法治轨道上深化改革、推进中国式现代化,做到改革和法治相统一,重大改革于法有据、及时把改革成果上升为法律制度;坚持系统观念,处理好经济和社会、政府和市场、效率和公平、活力和秩序、发展和安全等重大关系,增强改革系统性、整体性、协同性。

    \subsection{构建高水平社会主义市场经济体制}

    高水平社会主义市场经济体制是中国式现代化的重要保障。必须更好发挥市场机制作用,创造更加公平、更有活力的市场环境,实现资源配置效率最优化和效益最大化,既“放得活”又“管得住”,更好维护市场秩序、弥补市场失灵,畅通国民经济循环,激发全社会内生动力和创新活力。

    (5)坚持和落实“两个毫不动摇”。毫不动摇巩固和发展公有制经济,毫不动摇鼓励、支持、引导非公有制经济发展,保证各种所有制经济依法平等使用生产要素、公平参与市场竞争、同等受到法律保护,促进各种所有制经济优势互补、共同发展。

    深化国资国企改革,完善管理监督体制机制,增强各有关管理部门战略协同,推进国有经济布局优化和结构调整,推动国有资本和国有企业做强做优做大,增强核心功能,提升核心竞争力。进一步明晰不同类型国有企业功能定位,完善主责主业管理,明确国有资本重点投资领域和方向。推动国有资本向关系国家安全、国民经济命脉的重要行业和关键领域集中,向关系国计民生的公共服务、应急能力、公益性领域等集中,向前瞻性战略性新兴产业集中。健全国有企业推进原始创新制度安排。深化国有资本投资、运营公司改革。建立国有企业履行战略使命评价制度,完善国有企业分类考核评价体系,开展国有经济增加值核算。推进能源、铁路、电信、水利、公用事业等行业自然垄断环节独立运营和竞争性环节市场化改革,健全监管体制机制。

    坚持致力于为非公有制经济发展营造良好环境和提供更多机会的方针政策。制定民营经济促进法。深入破除市场准入壁垒,推进基础设施竞争性领域向经营主体公平开放,完善民营企业参与国家重大项目建设长效机制。支持有能力的民营企业牵头承担国家重大技术攻关任务,向民营企业进一步开放国家重大科研基础设施。完善民营企业融资支持政策制度,破解融资难、融资贵问题。健全涉企收费长效监管和拖欠企业账款清偿法律法规体系。加快建立民营企业信用状况综合评价体系,健全民营中小企业增信制度。支持引导民营企业完善治理结构和管理制度,加强企业合规建设和廉洁风险防控。加强事中事后监管,规范涉民营企业行政检查。

    完善中国特色现代企业制度,弘扬企业家精神,支持和引导各类企业提高资源要素利用效率和经营管理水平、履行社会责任,加快建设更多世界一流企业。

    (6)构建全国统一大市场。推动市场基础制度规则统一、市场监管公平统一、市场设施高标准联通。加强公平竞争审查刚性约束,强化反垄断和反不正当竞争,清理和废除妨碍全国统一市场和公平竞争的各种规定和做法。规范地方招商引资法规制度,严禁违法违规给予政策优惠行为。建立健全统一规范、信息共享的招标投标和政府、事业单位、国有企业采购等公共资源交易平台体系,实现项目全流程公开管理。提升市场综合监管能力和水平。健全国家标准体系,深化地方标准管理制度改革。

    完善要素市场制度和规则,推动生产要素畅通流动、各类资源高效配置、市场潜力充分释放。构建城乡统一的建设用地市场。完善促进资本市场规范发展基础制度。培育全国一体化技术和数据市场。完善主要由市场供求关系决定要素价格机制,防止政府对价格形成的不当干预。健全劳动、资本、土地、知识、技术、管理、数据等生产要素由市场评价贡献、按贡献决定报酬的机制。推进水、能源、交通等领域价格改革,优化居民阶梯水价、电价、气价制度,完善成品油定价机制。

    完善流通体制,加快发展物联网,健全一体衔接的流通规则和标准,降低全社会物流成本。深化能源管理体制改革,建设全国统一电力市场,优化油气管网运行调度机制。

    加快培育完整内需体系,建立政府投资支持基础性、公益性、长远性重大项目建设长效机制,健全政府投资有效带动社会投资体制机制,深化投资审批制度改革,完善激发社会资本投资活力和促进投资落地机制,形成市场主导的有效投资内生增长机制。完善扩大消费长效机制,减少限制性措施,合理增加公共消费,积极推进首发经济。

    (7)完善市场经济基础制度。完善产权制度,依法平等长久保护各种所有制经济产权,建立高效的知识产权综合管理体制。完善市场信息披露制度,构建商业秘密保护制度。对侵犯各种所有制经济产权和合法利益的行为实行同责同罪同罚,完善惩罚性赔偿制度。加强产权执法司法保护,防止和纠正利用行政、刑事手段干预经济纠纷,健全依法甄别纠正涉企冤错案件机制。

    完善市场准入制度,优化新业态新领域市场准入环境。深化注册资本认缴登记制度改革,实行依法按期认缴。健全企业破产机制,探索建立个人破产制度,推进企业注销配套改革,完善企业退出制度。健全社会信用体系和监管制度。

    \subsection{健全推动经济高质量发展体制机制健全推动经济高质量发展体制机制}

    高质量发展是全面建设社会主义现代化国家的首要任务。必须以新发展理念引领改革,立足新发展阶段,深化供给侧结构性改革,完善推动高质量发展激励约束机制,塑造发展新动能新优势。

    (8)健全因地制宜发展新质生产力体制机制。推动技术革命性突破、生产要素创新性配置、产业深度转型升级,推动劳动者、劳动资料、劳动对象优化组合和更新跃升,催生新产业、新模式、新动能,发展以高技术、高效能、高质量为特征的生产力。加强关键共性技术、前沿引领技术、现代工程技术、颠覆性技术创新,加强新领域新赛道制度供给,建立未来产业投入增长机制,完善推动新一代信息技术、人工智能、航空航天、新能源、新材料、高端装备、生物医药、量子科技等战略性产业发展政策和治理体系,引导新兴产业健康有序发展。以国家标准提升引领传统产业优化升级,支持企业用数智技术、绿色技术改造提升传统产业。强化环保、安全等制度约束。

    健全相关规则和政策,加快形成同新质生产力更相适应的生产关系,促进各类先进生产要素向发展新质生产力集聚,大幅提升全要素生产率。鼓励和规范发展天使投资、风险投资、私募股权投资,更好发挥政府投资基金作用,发展耐心资本。

    (9)健全促进实体经济和数字经济深度融合制度。加快推进新型工业化,培育壮大先进制造业集群,推动制造业高端化、智能化、绿色化发展。建设一批行业共性技术平台,加快产业模式和企业组织形态变革,健全提升优势产业领先地位体制机制。优化重大产业基金运作和监管机制,确保资金投向符合国家战略要求。建立保持制造业合理比重投入机制,合理降低制造业综合成本和税费负担。

    加快构建促进数字经济发展体制机制,完善促进数字产业化和产业数字化政策体系。加快新一代信息技术全方位全链条普及应用,发展工业互联网,打造具有国际竞争力的数字产业集群。促进平台经济创新发展,健全平台经济常态化监管制度。建设和运营国家数据基础设施,促进数据共享。加快建立数据产权归属认定、市场交易、权益分配、利益保护制度,提升数据安全治理监管能力,建立高效便利安全的数据跨境流动机制。

    (10)完善发展服务业体制机制。完善支持服务业发展政策体系,优化服务业核算,推进服务业标准化建设。聚焦重点环节分领域推进生产性服务业高质量发展,发展产业互联网平台,破除跨地区经营行政壁垒,推进生产性服务业融合发展。健全加快生活性服务业多样化发展机制。完善中介服务机构法规制度体系,促进中介服务机构诚实守信、依法履责。

    (11)健全现代化基础设施建设体制机制。构建新型基础设施规划和标准体系,健全新型基础设施融合利用机制,推进传统基础设施数字化改造,拓宽多元化投融资渠道,健全重大基础设施建设协调机制。深化综合交通运输体系改革,推进铁路体制改革,发展通用航空和低空经济,推动收费公路政策优化。提高航运保险承保能力和全球服务水平,推进海事仲裁制度规则创新。健全重大水利工程建设、运行、管理机制。

    (12)健全提升产业链供应链韧性和安全水平制度。抓紧打造自主可控的产业链供应链,健全强化集成电路、工业母机、医疗装备、仪器仪表、基础软件、工业软件、先进材料等重点产业链发展体制机制,全链条推进技术攻关、成果应用。建立产业链供应链安全风险评估和应对机制。完善产业在国内梯度有序转移的协作机制,推动转出地和承接地利益共享。建设国家战略腹地和关键产业备份。加快完善国家储备体系。完善战略性矿产资源探产供储销统筹和衔接体系。

    \subsection{构建支持全面创新体制机制}

    教育、科技、人才是中国式现代化的基础性、战略性支撑。必须深入实施科教兴国战略、人才强国战略、创新驱动发展战略,统筹推进教育科技人才体制机制一体改革,健全新型举国体制,提升国家创新体系整体效能。

    (13)深化教育综合改革。加快建设高质量教育体系,统筹推进育人方式、办学模式、管理体制、保障机制改革。完善立德树人机制,推进大中小学思政课一体化改革创新,健全德智体美劳全面培养体系,提升教师教书育人能力,健全师德师风建设长效机制,深化教育评价改革。优化高等教育布局,加快建设中国特色、世界一流的大学和优势学科。分类推进高校改革,建立科技发展、国家战略需求牵引的学科设置调整机制和人才培养模式,超常布局急需学科专业,加强基础学科、新兴学科、交叉学科建设和拔尖人才培养,着力加强创新能力培养。完善高校科技创新机制,提高成果转化效能。强化科技教育和人文教育协同。加快构建职普融通、产教融合的职业教育体系。完善学生实习实践制度。引导规范民办教育发展。推进高水平教育开放,鼓励国外高水平理工类大学来华合作办学。

    优化区域教育资源配置,建立同人口变化相协调的基本公共教育服务供给机制。完善义务教育优质均衡推进机制,探索逐步扩大免费教育范围。健全学前教育和特殊教育、专门教育保障机制。推进教育数字化,赋能学习型社会建设,加强终身教育保障。

    (14)深化科技体制改革。坚持面向世界科技前沿、面向经济主战场、面向国家重大需求、面向人民生命健康,优化重大科技创新组织机制,统筹强化关键核心技术攻关,推动科技创新力量、要素配置、人才队伍体系化、建制化、协同化。加强国家战略科技力量建设,完善国家实验室体系,优化国家科研机构、高水平研究型大学、科技领军企业定位和布局,推进科技创新央地协同,统筹各类科创平台建设,鼓励和规范发展新型研发机构,发挥我国超大规模市场引领作用,加强创新资源统筹和力量组织,推动科技创新和产业创新融合发展。构建科技安全风险监测预警和应对体系,加强科技基础条件自主保障。健全科技社团管理制度。扩大国际科技交流合作,鼓励在华设立国际科技组织,优化高校、科研院所、科技社团对外专业交流合作管理机制。

    改进科技计划管理,强化基础研究领域、交叉前沿领域、重点领域前瞻性、引领性布局。加强有组织的基础研究,提高科技支出用于基础研究比重,完善竞争性支持和稳定支持相结合的基础研究投入机制,鼓励有条件的地方、企业、社会组织、个人支持基础研究,支持基础研究选题多样化,鼓励开展高风险、高价值基础研究。深化科技评价体系改革,加强科技伦理治理,严肃整治学术不端行为。

    强化企业科技创新主体地位,建立培育壮大科技领军企业机制,加强企业主导的产学研深度融合,建立企业研发准备金制度,支持企业主动牵头或参与国家科技攻关任务。构建促进专精特新中小企业发展壮大机制。鼓励科技型中小企业加大研发投入,提高研发费用加计扣除比例。鼓励和引导高校、科研院所按照先使用后付费方式把科技成果许可给中小微企业使用。

    完善中央财政科技经费分配和管理使用机制,健全中央财政科技计划执行和专业机构管理体制。扩大财政科研项目经费“包干制”范围,赋予科学家更大技术路线决定权、更大经费支配权、更大资源调度权。建立专家实名推荐的非共识项目筛选机制。允许科研类事业单位实行比一般事业单位更灵活的管理制度,探索实行企业化管理。

    深化科技成果转化机制改革,加强国家技术转移体系建设,加快布局建设一批概念验证、中试验证平台,完善首台(套)、首批次、首版次应用政策,加大政府采购自主创新产品力度。加强技术经理人队伍建设。

    允许科技人员在科技成果转化收益分配上有更大自主权,建立职务科技成果资产单列管理制度,深化职务科技成果赋权改革。深化高校、科研院所收入分配改革。允许更多符合条件的国有企业以创新创造为导向,在科研人员中开展多种形式中长期激励。

    构建同科技创新相适应的科技金融体制,加强对国家重大科技任务和科技型中小企业的金融支持,完善长期资本投早、投小、投长期、投硬科技的支持政策。健全重大技术攻关风险分散机制,建立科技保险政策体系。提高外资在华开展股权投资、风险投资便利性。

    (15)深化人才发展体制机制改革。实施更加积极、更加开放、更加有效的人才政策,完善人才自主培养机制,加快建设国家高水平人才高地和吸引集聚人才平台。加快建设国家战略人才力量,着力培养造就战略科学家、一流科技领军人才和创新团队,着力培养造就卓越工程师、大国工匠、高技能人才,提高各类人才素质。建设一流产业技术工人队伍。完善人才有序流动机制,促进人才区域合理布局,深化东中西部人才协作。完善青年创新人才发现、选拔、培养机制,更好保障青年科技人员待遇。健全保障科研人员专心科研制度。

    强化人才激励机制,坚持向用人主体授权、为人才松绑。建立以创新能力、质量、实效、贡献为导向的人才评价体系。打通高校、科研院所和企业人才交流通道。完善海外引进人才支持保障机制,形成具有国际竞争力的人才制度体系。探索建立高技术人才移民制度。

    \subsection{健全宏观经济治理体系}

    科学的宏观调控、有效的政府治理是发挥社会主义市场经济体制优势的内在要求。必须完善宏观调控制度体系,统筹推进财税、金融等重点领域改革,增强宏观政策取向一致性。

    (16)完善国家战略规划体系和政策统筹协调机制。构建国家战略制定和实施机制,加强国家重大战略深度融合,增强国家战略宏观引导、统筹协调功能。健全国家经济社会发展规划制度体系,强化规划衔接落实机制,发挥国家发展规划战略导向作用,强化国土空间规划基础作用,增强专项规划和区域规划实施支撑作用。健全专家参与公共决策制度。

    围绕实施国家发展规划、重大战略促进财政、货币、产业、价格、就业等政策协同发力,优化各类增量资源配置和存量结构调整。探索实行国家宏观资产负债表管理。把经济政策和非经济性政策都纳入宏观政策取向一致性评估。健全预期管理机制。健全支撑高质量发展的统计指标核算体系,加强新经济新领域纳统覆盖。加强产业活动单位统计基础建设,优化总部和分支机构统计办法,逐步推广经营主体活动发生地统计。健全国际宏观政策协调机制。

    (17)深化财税体制改革。健全预算制度,加强财政资源和预算统筹,把依托行政权力、政府信用、国有资源资产获取的收入全部纳入政府预算管理。完善国有资本经营预算和绩效评价制度,强化国家重大战略任务和基本民生财力保障。强化对预算编制和财政政策的宏观指导。加强公共服务绩效管理,强化事前功能评估。深化零基预算改革。统一预算分配权,提高预算管理统一性、规范性,完善预算公开和监督制度。完善权责发生制政府综合财务报告制度。

    健全有利于高质量发展、社会公平、市场统一的税收制度,优化税制结构。研究同新业态相适应的税收制度。全面落实税收法定原则,规范税收优惠政策,完善对重点领域和关键环节支持机制。健全直接税体系,完善综合和分类相结合的个人所得税制度,规范经营所得、资本所得、财产所得税收政策,实行劳动性所得统一征税。深化税收征管改革。

    建立权责清晰、财力协调、区域均衡的中央和地方财政关系。增加地方自主财力,拓展地方税源,适当扩大地方税收管理权限。完善财政转移支付体系,清理规范专项转移支付,增加一般性转移支付,提升市县财力同事权相匹配程度。建立促进高质量发展转移支付激励约束机制。推进消费税征收环节后移并稳步下划地方,完善增值税留抵退税政策和抵扣链条,优化共享税分享比例。研究把城市维护建设税、教育费附加、地方教育附加合并为地方附加税,授权地方在一定幅度内确定具体适用税率。合理扩大地方政府专项债券支持范围,适当扩大用作资本金的领域、规模、比例。完善政府债务管理制度,建立全口径地方债务监测监管体系和防范化解隐性债务风险长效机制,加快地方融资平台改革转型。规范非税收入管理,适当下沉部分非税收入管理权限,由地方结合实际差别化管理。

    适当加强中央事权、提高中央财政支出比例。中央财政事权原则上通过中央本级安排支出,减少委托地方代行的中央财政事权。不得违规要求地方安排配套资金,确需委托地方行使事权的,通过专项转移支付安排资金。

    (18)深化金融体制改革。加快完善中央银行制度,畅通货币政策传导机制。积极发展科技金融、绿色金融、普惠金融、养老金融、数字金融,加强对重大战略、重点领域、薄弱环节的优质金融服务。完善金融机构定位和治理,健全服务实体经济的激励约束机制。发展多元股权融资,加快多层次债券市场发展,提高直接融资比重。优化国有金融资本管理体制。

    健全投资和融资相协调的资本市场功能,防风险、强监管,促进资本市场健康稳定发展。支持长期资金入市。提高上市公司质量,强化上市公司监管和退市制度。建立增强资本市场内在稳定性长效机制。完善大股东、实际控制人行为规范约束机制。完善上市公司分红激励约束机制。健全投资者保护机制。推动区域性股权市场规则对接、标准统一。

    制定金融法。完善金融监管体系,依法将所有金融活动纳入监管,强化监管责任和问责制度,加强中央和地方监管协同。建设安全高效的金融基础设施,统一金融市场登记托管、结算清算规则制度,建立风险早期纠正硬约束制度,筑牢有效防控系统性风险的金融稳定保障体系。健全金融消费者保护和打击非法金融活动机制,构建产业资本和金融资本“防火墙”。推动金融高水平开放,稳慎扎实推进人民币国际化,发展人民币离岸市场。稳妥推进数字人民币研发和应用。加快建设上海国际金融中心。

    完善准入前国民待遇加负面清单管理模式,支持符合条件的外资机构参与金融业务试点。稳慎拓展金融市场互联互通,优化合格境外投资者制度。推进自主可控的跨境支付体系建设,强化开放条件下金融安全机制。建立统一的全口径外债监管体系。积极参与国际金融治理。

    (19)完善实施区域协调发展战略机制。构建优势互补的区域经济布局和国土空间体系。健全推动西部大开发形成新格局、东北全面振兴取得新突破、中部地区加快崛起、东部地区加快推进现代化的制度和政策体系。推动京津冀、长三角、粤港澳大湾区等地区更好发挥高质量发展动力源作用,优化长江经济带发展、黄河流域生态保护和高质量发展机制。高标准高质量推进雄安新区建设。推动成渝地区双城经济圈建设走深走实。健全主体功能区制度体系,强化国土空间优化发展保障机制。完善区域一体化发展机制,构建跨行政区合作发展新机制,深化东中西部产业协作。完善促进海洋经济发展体制机制。

    \subsection{完善城乡融合发展体制机制}

    城乡融合发展是中国式现代化的必然要求。必须统筹新型工业化、新型城镇化和乡村全面振兴,全面提高城乡规划、建设、治理融合水平,促进城乡要素平等交换、双向流动,缩小城乡差别,促进城乡共同繁荣发展。

    (20)健全推进新型城镇化体制机制。构建产业升级、人口集聚、城镇发展良性互动机制。推行由常住地登记户口提供基本公共服务制度,推动符合条件的农业转移人口社会保险、住房保障、随迁子女义务教育等享有同迁入地户籍人口同等权利,加快农业转移人口市民化。保障进城落户农民合法土地权益,依法维护进城落户农民的土地承包权、宅基地使用权、集体收益分配权,探索建立自愿有偿退出的办法。

    坚持人民城市人民建、人民城市为人民。健全城市规划体系,引导大中小城市和小城镇协调发展、集约紧凑布局。深化城市建设、运营、治理体制改革,加快转变城市发展方式。推动形成超大特大城市智慧高效治理新体系,建立都市圈同城化发展体制机制。深化赋予特大镇同人口和经济规模相适应的经济社会管理权改革。建立可持续的城市更新模式和政策法规,加强地下综合管廊建设和老旧管线改造升级,深化城市安全韧性提升行动。

    (21)巩固和完善农村基本经营制度。有序推进第二轮土地承包到期后再延长三十年试点,深化承包地所有权、承包权、经营权分置改革,发展农业适度规模经营。完善农业经营体系,完善承包地经营权流转价格形成机制,促进农民合作经营,推动新型农业经营主体扶持政策同带动农户增收挂钩。健全便捷高效的农业社会化服务体系。发展新型农村集体经济,构建产权明晰、分配合理的运行机制,赋予农民更加充分的财产权益。

    (22)完善强农惠农富农支持制度。坚持农业农村优先发展,完善乡村振兴投入机制。壮大县域富民产业,构建多元化食物供给体系,培育乡村新产业新业态。优化农业补贴政策体系,发展多层次农业保险。完善覆盖农村人口的常态化防止返贫致贫机制,建立农村低收入人口和欠发达地区分层分类帮扶制度。健全脱贫攻坚国家投入形成资产的长效管理机制。运用“千万工程”经验,健全推动乡村全面振兴长效机制。

    加快健全种粮农民收益保障机制,推动粮食等重要农产品价格保持在合理水平。统筹建立粮食产销区省际横向利益补偿机制,在主产区利益补偿上迈出实质步伐。统筹推进粮食购销和储备管理体制机制改革,建立监管新模式。健全粮食和食物节约长效机制。

    (23)深化土地制度改革。改革完善耕地占补平衡制度,各类耕地占用纳入统一管理,完善补充耕地质量验收机制,确保达到平衡标准。完善高标准农田建设、验收、管护机制。健全保障耕地用于种植基本农作物管理体系。允许农户合法拥有的住房通过出租、入股、合作等方式盘活利用。有序推进农村集体经营性建设用地入市改革,健全土地增值收益分配机制。

    优化土地管理,健全同宏观政策和区域发展高效衔接的土地管理制度,优先保障主导产业、重大项目合理用地,使优势地区有更大发展空间。建立新增城镇建设用地指标配置同常住人口增加协调机制。探索国家集中垦造耕地定向用于特定项目和地区落实占补平衡机制。优化城市工商业土地利用,加快发展建设用地二级市场,推动土地混合开发利用、用途合理转换,盘活存量土地和低效用地。开展各类产业园区用地专项治理。制定工商业用地使用权延期和到期后续期政策。

    \subsection{完善高水平对外开放体制机制}

    开放是中国式现代化的鲜明标识。必须坚持对外开放基本国策,坚持以开放促改革,依托我国超大规模市场优势,在扩大国际合作中提升开放能力,建设更高水平开放型经济新体制。

    (24)稳步扩大制度型开放。主动对接国际高标准经贸规则,在产权保护、产业补贴、环境标准、劳动保护、政府采购、电子商务、金融领域等实现规则、规制、管理、标准相通相容,打造透明稳定可预期的制度环境。扩大自主开放,有序扩大我国商品市场、服务市场、资本市场、劳务市场等对外开放,扩大对最不发达国家单边开放。深化援外体制机制改革,实现全链条管理。

    维护以世界贸易组织为核心的多边贸易体制,积极参与全球经济治理体系改革,提供更多全球公共产品。扩大面向全球的高标准自由贸易区网络,建立同国际通行规则衔接的合规机制,优化开放合作环境。

    (25)深化外贸体制改革。强化贸易政策和财税、金融、产业政策协同,打造贸易强国制度支撑和政策支持体系,加快内外贸一体化改革,积极应对贸易数字化、绿色化趋势。推进通关、税务、外汇等监管创新,营造有利于新业态新模式发展的制度环境。创新发展数字贸易,推进跨境电商综合试验区建设。建设大宗商品交易中心,建设全球集散分拨中心,支持各类主体有序布局海外流通设施,支持有条件的地区建设国际物流枢纽中心和大宗商品资源配置枢纽。健全贸易风险防控机制,完善出口管制体系和贸易救济制度。

    创新提升服务贸易,全面实施跨境服务贸易负面清单,推进服务业扩大开放综合试点示范,鼓励专业服务机构提升国际化服务能力。加快推进离岸贸易发展,发展新型离岸国际贸易业务。建立健全跨境金融服务体系,丰富金融产品和服务供给。

    (26)深化外商投资和对外投资管理体制改革。营造市场化、法治化、国际化一流营商环境,依法保护外商投资权益。扩大鼓励外商投资产业目录,合理缩减外资准入负面清单,落实全面取消制造业领域外资准入限制措施,推动电信、互联网、教育、文化、医疗等领域有序扩大开放。深化外商投资促进体制机制改革,保障外资企业在要素获取、资质许可、标准制定、政府采购等方面的国民待遇,支持参与产业链上下游配套协作。完善境外人员入境居住、医疗、支付等生活便利制度。完善促进和保障对外投资体制机制,健全对外投资管理服务体系,推动产业链供应链国际合作。

    (27)优化区域开放布局。巩固东部沿海地区开放先导地位,提高中西部和东北地区开放水平,加快形成陆海内外联动、东西双向互济的全面开放格局。发挥沿海、沿边、沿江和交通干线等优势,优化区域开放功能分工,打造形态多样的开放高地。实施自由贸易试验区提升战略,鼓励首创性、集成式探索。加快建设海南自由贸易港。

    发挥“一国两制”制度优势,巩固提升香港国际金融、航运、贸易中心地位,支持香港、澳门打造国际高端人才集聚高地,健全香港、澳门在国家对外开放中更好发挥作用机制。深化粤港澳大湾区合作,强化规则衔接、机制对接。完善促进两岸经济文化交流合作制度和政策,深化两岸融合发展。

    (28)完善推进高质量共建“一带一路”机制。继续实施“一带一路”科技创新行动计划,加强绿色发展、数字经济、人工智能、能源、税收、金融、减灾等领域的多边合作平台建设。完善陆海天网一体化布局,构建“一带一路”立体互联互通网络。统筹推进重大标志性工程和“小而美”民生项目。

    \subsection{健全全过程人民民主制度体系}

    发展全过程人民民主是中国式现代化的本质要求。必须坚定不移走中国特色社会主义政治发展道路,坚持和完善我国根本政治制度、基本政治制度、重要政治制度,丰富各层级民主形式,把人民当家作主具体、现实体现到国家政治生活和社会生活各方面。

    (29)加强人民当家作主制度建设。坚持好、完善好、运行好人民代表大会制度。健全人大对行政机关、监察机关、审判机关、检察机关监督制度,完善监督法及其实施机制,强化人大预算决算审查监督和国有资产管理、政府债务管理监督。健全人大议事规则和论证、评估、评议、听证制度。丰富人大代表联系人民群众的内容和形式。健全吸纳民意、汇集民智工作机制。发挥工会、共青团、妇联等群团组织联系服务群众的桥梁纽带作用。

    (30)健全协商民主机制。发挥人民政协作为专门协商机构作用,健全深度协商互动、意见充分表达、广泛凝聚共识的机制,加强人民政协反映社情民意、联系群众、服务人民机制建设。完善人民政协民主监督机制。

    完善协商民主体系,丰富协商方式,健全政党协商、人大协商、政府协商、政协协商、人民团体协商、基层协商以及社会组织协商制度化平台,加强各种协商渠道协同配合。健全协商于决策之前和决策实施之中的落实机制,完善协商成果采纳、落实、反馈机制。

    (31)健全基层民主制度。健全基层党组织领导的基层群众自治机制,完善基层民主制度体系和工作体系,拓宽基层各类组织和群众有序参与基层治理渠道。完善办事公开制度。健全以职工代表大会为基本形式的企事业单位民主管理制度,完善企业职工参与管理的有效形式。

    (32)完善大统战工作格局。完善发挥统一战线凝聚人心、汇聚力量政治作用的政策举措。坚持好、发展好、完善好中国新型政党制度。更好发挥党外人士作用,健全党外代表人士队伍建设制度。制定民族团结进步促进法,健全铸牢中华民族共同体意识制度机制,增强中华民族凝聚力。系统推进我国宗教中国化,加强宗教事务治理法治化。完善党外知识分子和新的社会阶层人士政治引领机制。全面构建亲清政商关系,健全促进非公有制经济健康发展、非公有制经济人士健康成长工作机制。完善港澳台和侨务工作机制。

    \subsection{完善中国特色社会主义法治体系}

    法治是中国式现代化的重要保障。必须全面贯彻实施宪法,维护宪法权威,协同推进立法、执法、司法、守法各环节改革,健全法律面前人人平等保障机制,弘扬社会主义法治精神,维护社会公平正义,全面推进国家各方面工作法治化。

    (33)深化立法领域改革。完善以宪法为核心的中国特色社会主义法律体系,健全保证宪法全面实施制度体系,建立宪法实施情况报告制度。完善党委领导、人大主导、政府依托、各方参与的立法工作格局。统筹立改废释纂,加强重点领域、新兴领域、涉外领域立法,完善合宪性审查、备案审查制度,提高立法质量。探索区域协同立法。健全党内法规同国家法律法规衔接协调机制。建设全国统一的法律法规和规范性文件信息平台。

    (34)深入推进依法行政。推进政府机构、职能、权限、程序、责任法定化,促进政务服务标准化、规范化、便利化,完善覆盖全国的一体化在线政务服务平台。完善重大决策、规范性文件合法性审查机制。加强政府立法审查。深化行政执法体制改革,完善基层综合执法体制机制,健全行政执法监督体制机制。完善行政处罚等领域行政裁量权基准制度,推动行政执法标准跨区域衔接。完善行政处罚和刑事处罚双向衔接制度。健全行政复议体制机制。完善行政裁决制度。完善垂直管理体制和地方分级管理体制,健全垂直管理机构和地方协作配合机制。稳妥推进人口小县机构优化。深化开发区管理制度改革。优化事业单位结构布局,强化公益性。

    (35)健全公正执法司法体制机制。健全监察机关、公安机关、检察机关、审判机关、司法行政机关各司其职,监察权、侦查权、检察权、审判权、执行权相互配合、相互制约的体制机制,确保执法司法各环节全过程在有效制约监督下运行。深化审判权和执行权分离改革,健全国家执行体制,强化当事人、检察机关和社会公众对执行活动的全程监督。完善执法司法救济保护制度,完善国家赔偿制度。深化和规范司法公开,落实和完善司法责任制。规范专门法院设置。深化行政案件级别管辖、集中管辖、异地管辖改革。构建协同高效的警务体制机制,推进地方公安机关机构编制管理改革,继续推进民航公安机关和海关缉私部门管理体制改革。规范警务辅助人员管理制度。

    坚持正确人权观,加强人权执法司法保障,完善事前审查、事中监督、事后纠正等工作机制,完善涉及公民人身权利强制措施以及查封、扣押、冻结等强制措施的制度,依法查处利用职权徇私枉法、非法拘禁、刑讯逼供等犯罪行为。推进刑事案件律师辩护全覆盖。建立轻微犯罪记录封存制度。

    (36)完善推进法治社会建设机制。健全覆盖城乡的公共法律服务体系,深化律师制度、公证体制、仲裁制度、调解制度、司法鉴定管理体制改革。改进法治宣传教育,完善以实践为导向的法学院校教育培养机制。加强和改进未成年人权益保护,强化未成年人犯罪预防和治理,制定专门矫治教育规定。

    (37)加强涉外法治建设。建立一体推进涉外立法、执法、司法、守法和法律服务、法治人才培养的工作机制。完善涉外法律法规体系和法治实施体系,深化执法司法国际合作。完善涉外民事法律关系中当事人依法约定管辖、选择适用域外法等司法审判制度。健全国际商事仲裁和调解制度,培育国际一流仲裁机构、律师事务所。积极参与国际规则制定。

    \subsection{深化文化体制机制改革}

    中国式现代化是物质文明和精神文明相协调的现代化。必须增强文化自信,发展社会主义先进文化,弘扬革命文化,传承中华优秀传统文化,加快适应信息技术迅猛发展新形势,培育形成规模宏大的优秀文化人才队伍,激发全民族文化创新创造活力。

    (38)完善意识形态工作责任制。健全用党的创新理论武装全党、教育人民、指导实践工作体系,完善党委(党组)理论学习中心组学习制度,完善思想政治工作体系。创新马克思主义理论研究和建设工程,实施哲学社会科学创新工程,构建中国哲学社会科学自主知识体系。完善新闻发言人制度。构建适应全媒体生产传播工作机制和评价体系,推进主流媒体系统性变革。完善舆论引导机制和舆情应对协同机制。

    推动理想信念教育常态化制度化。完善培育和践行社会主义核心价值观制度机制。改进创新文明培育、文明实践、文明创建工作机制。实施文明乡风建设工程。优化英模人物宣传学习机制,创新爱国主义教育和各类群众性主题活动组织机制,推动全社会崇尚英雄、缅怀先烈、争做先锋。构建中华传统美德传承体系,健全社会公德、职业道德、家庭美德、个人品德建设体制机制,健全诚信建设长效机制,教育引导全社会自觉遵守法律、遵循公序良俗,坚决反对拜金主义、享乐主义、极端个人主义和历史虚无主义。形成网上思想道德教育分众化、精准化实施机制。建立健全道德领域突出问题协同治理机制,完善“扫黄打非”长效机制。

    (39)优化文化服务和文化产品供给机制。完善公共文化服务体系,建立优质文化资源直达基层机制,健全社会力量参与公共文化服务机制,推进公共文化设施所有权和使用权分置改革。深化文化领域国资国企改革,分类推进文化事业单位深化内部改革,完善文艺院团建设发展机制。

    坚持以人民为中心的创作导向,坚持出成果和出人才相结合、抓作品和抓环境相贯通,改进文艺创作生产服务、引导、组织工作机制。健全文化产业体系和市场体系,完善文化经济政策。探索文化和科技融合的有效机制,加快发展新型文化业态。深化文化领域行政审批备案制度改革,加强事中事后监管。深化文娱领域综合治理。

    建立文化遗产保护传承工作协调机构,建立文化遗产保护督察制度,推动文化遗产系统性保护和统一监管。构建中华文明标识体系。健全文化和旅游深度融合发展体制机制。完善全民健身公共服务体系,改革完善竞技体育管理体制和运行机制。

    (40)健全网络综合治理体系。深化网络管理体制改革,整合网络内容建设和管理职能,推进新闻宣传和网络舆论一体化管理。完善生成式人工智能发展和管理机制。加强网络空间法治建设,健全网络生态治理长效机制,健全未成年人网络保护工作体系。

    (41)构建更有效力的国际传播体系。推进国际传播格局重构,深化主流媒体国际传播机制改革创新,加快构建多渠道、立体式对外传播格局。加快构建中国话语和中国叙事体系,全面提升国际传播效能。建设全球文明倡议践行机制。推动走出去、请进来管理便利化,扩大国际人文交流合作。

    \subsection{健全保障和改善民生制度体系}

    在发展中保障和改善民生是中国式现代化的重大任务。必须坚持尽力而为、量力而行,完善基本公共服务制度体系,加强普惠性、基础性、兜底性民生建设,解决好人民最关心最直接最现实的利益问题,不断满足人民对美好生活的向往。

    (42)完善收入分配制度。构建初次分配、再分配、第三次分配协调配套的制度体系,提高居民收入在国民收入分配中的比重,提高劳动报酬在初次分配中的比重。完善劳动者工资决定、合理增长、支付保障机制,健全按要素分配政策制度。完善税收、社会保障、转移支付等再分配调节机制。支持发展公益慈善事业。

    规范收入分配秩序,规范财富积累机制,多渠道增加城乡居民财产性收入,形成有效增加低收入群体收入、稳步扩大中等收入群体规模、合理调节过高收入的制度体系。深化国有企业工资决定机制改革,合理确定并严格规范国有企业各级负责人薪酬、津贴补贴等。

    (43)完善就业优先政策。健全高质量充分就业促进机制,完善就业公共服务体系,着力解决结构性就业矛盾。完善高校毕业生、农民工、退役军人等重点群体就业支持体系,健全终身职业技能培训制度。统筹城乡就业政策体系,同步推进户籍、用人、档案等服务改革,优化创业促进就业政策环境,支持和规范发展新就业形态。完善促进机会公平制度机制,畅通社会流动渠道。完善劳动关系协商协调机制,加强劳动者权益保障。

    (44)健全社会保障体系。完善基本养老保险全国统筹制度,健全全国统一的社保公共服务平台。健全社保基金保值增值和安全监管体系。健全基本养老、基本医疗保险筹资和待遇合理调整机制,逐步提高城乡居民基本养老保险基础养老金。健全灵活就业人员、农民工、新就业形态人员社保制度,扩大失业、工伤、生育保险覆盖面,全面取消在就业地参保户籍限制,完善社保关系转移接续政策。加快发展多层次多支柱养老保险体系,扩大年金制度覆盖范围,推行个人养老金制度。发挥各类商业保险补充保障作用。推进基本医疗保险省级统筹,深化医保支付方式改革,完善大病保险和医疗救助制度,加强医保基金监管。健全社会救助体系。健全保障妇女儿童合法权益制度。完善残疾人社会保障制度和关爱服务体系。

    加快建立租购并举的住房制度,加快构建房地产发展新模式。加大保障性住房建设和供给,满足工薪群体刚性住房需求。支持城乡居民多样化改善性住房需求。充分赋予各城市政府房地产市场调控自主权,因城施策,允许有关城市取消或调减住房限购政策、取消普通住宅和非普通住宅标准。改革房地产开发融资方式和商品房预售制度。完善房地产税收制度。

    (45)深化医药卫生体制改革。实施健康优先发展战略,健全公共卫生体系,促进社会共治、医防协同、医防融合,强化监测预警、风险评估、流行病学调查、检验检测、应急处置、医疗救治等能力。促进医疗、医保、医药协同发展和治理。促进优质医疗资源扩容下沉和区域均衡布局,加快建设分级诊疗体系,推进紧密型医联体建设,强化基层医疗卫生服务。深化以公益性为导向的公立医院改革,建立以医疗服务为主导的收费机制,完善薪酬制度,建立编制动态调整机制。引导规范民营医院发展。创新医疗卫生监管手段。健全支持创新药和医疗器械发展机制,完善中医药传承创新发展机制。

    (46)健全人口发展支持和服务体系。以应对老龄化、少子化为重点完善人口发展战略,健全覆盖全人群、全生命周期的人口服务体系,促进人口高质量发展。完善生育支持政策体系和激励机制,推动建设生育友好型社会。有效降低生育、养育、教育成本,完善生育休假制度,建立生育补贴制度,提高基本生育和儿童医疗公共服务水平,加大个人所得税抵扣力度。加强普惠育幼服务体系建设,支持用人单位办托、社区嵌入式托育、家庭托育点等多种模式发展。把握人口流动客观规律,推动相关公共服务随人走,促进城乡、区域人口合理集聚、有序流动。

    积极应对人口老龄化,完善发展养老事业和养老产业政策机制。发展银发经济,创造适合老年人的多样化、个性化就业岗位。按照自愿、弹性原则,稳妥有序推进渐进式延迟法定退休年龄改革。优化基本养老服务供给,培育社区养老服务机构,健全公办养老机构运营机制,鼓励和引导企业等社会力量积极参与,推进互助性养老服务,促进医养结合。加快补齐农村养老服务短板。改善对孤寡、残障失能等特殊困难老年人的服务,加快建立长期护理保险制度。

    \subsection{深化生态文明体制改革}

    中国式现代化是人与自然和谐共生的现代化。必须完善生态文明制度体系,协同推进降碳、减污、扩绿、增长,积极应对气候变化,加快完善落实绿水青山就是金山银山理念的体制机制。

    (47)完善生态文明基础体制。实施分区域、差异化、精准管控的生态环境管理制度,健全生态环境监测和评价制度。建立健全覆盖全域全类型、统一衔接的国土空间用途管制和规划许可制度。健全自然资源资产产权制度和管理制度体系,完善全民所有自然资源资产所有权委托代理机制,建立生态环境保护、自然资源保护利用和资产保值增值等责任考核监督制度。完善国家生态安全工作协调机制。编纂生态环境法典。

    (48)健全生态环境治理体系。推进生态环境治理责任体系、监管体系、市场体系、法律法规政策体系建设。完善精准治污、科学治污、依法治污制度机制,落实以排污许可制为核心的固定污染源监管制度,建立新污染物协同治理和环境风险管控体系,推进多污染物协同减排。深化环境信息依法披露制度改革,构建环境信用监管体系。推动重要流域构建上下游贯通一体的生态环境治理体系。全面推进以国家公园为主体的自然保护地体系建设。

    落实生态保护红线管理制度,健全山水林田湖草沙一体化保护和系统治理机制,建设多元化生态保护修复投入机制。落实水资源刚性约束制度,全面推行水资源费改税。强化生物多样性保护工作协调机制。健全海洋资源开发保护制度。健全生态产品价值实现机制。深化自然资源有偿使用制度改革。推进生态综合补偿,健全横向生态保护补偿机制,统筹推进生态环境损害赔偿。

    (49)健全绿色低碳发展机制。实施支持绿色低碳发展的财税、金融、投资、价格政策和标准体系,发展绿色低碳产业,健全绿色消费激励机制,促进绿色低碳循环发展经济体系建设。优化政府绿色采购政策,完善绿色税制。完善资源总量管理和全面节约制度,健全废弃物循环利用体系。健全煤炭清洁高效利用机制。加快规划建设新型能源体系,完善新能源消纳和调控政策措施。完善适应气候变化工作体系。建立能耗双控向碳排放双控全面转型新机制。构建碳排放统计核算体系、产品碳标识认证制度、产品碳足迹管理体系,健全碳市场交易制度、温室气体自愿减排交易制度,积极稳妥推进碳达峰碳中和。

    \subsection{推进国家安全体系和能力现代化}

    国家安全是中国式现代化行稳致远的重要基础。必须全面贯彻总体国家安全观,完善维护国家安全体制机制,实现高质量发展和高水平安全良性互动,切实保障国家长治久安。

    (50)健全国家安全体系。强化国家安全工作协调机制,完善国家安全法治体系、战略体系、政策体系、风险监测预警体系,完善重点领域安全保障体系和重要专项协调指挥体系。构建联动高效的国家安全防护体系,推进国家安全科技赋能。

    (51)完善公共安全治理机制。健全重大突发公共事件处置保障体系,完善大安全大应急框架下应急指挥机制,强化基层应急基础和力量,提高防灾减灾救灾能力。完善安全生产风险排查整治和责任倒查机制。完善食品药品安全责任体系。健全生物安全监管预警防控体系。加强网络安全体制建设,建立人工智能安全监管制度。

    (52)健全社会治理体系。坚持和发展新时代“枫桥经验”,健全党组织领导的自治、法治、德治相结合的城乡基层治理体系,完善共建共治共享的社会治理制度。探索建立全国统一的人口管理制度。健全社会工作体制机制,加强党建引领基层治理,加强社会工作者队伍建设,推动志愿服务体系建设。推进信访工作法治化。提高市域社会治理能力,强化市民热线等公共服务平台功能,健全“高效办成一件事”重点事项清单管理机制和常态化推进机制。健全社会心理服务体系和危机干预机制。健全发挥家庭家教家风建设在基层治理中作用的机制。深化行业协会商会改革。健全社会组织管理制度。

    健全乡镇(街道)职责和权力、资源相匹配制度,加强乡镇(街道)服务管理力量。完善社会治安整体防控体系,健全扫黑除恶常态化机制,依法严惩群众反映强烈的违法犯罪活动。

    (53)完善涉外国家安全机制。建立健全周边安全工作协调机制。强化海外利益和投资风险预警、防控、保护体制机制,深化安全领域国际执法合作,维护我国公民、法人在海外合法权益。健全反制裁、反干涉、反“长臂管辖”机制。健全维护海洋权益机制。完善参与全球安全治理机制。

    \subsection{持续深化国防和军队改革}

    国防和军队现代化是中国式现代化的重要组成部分。必须坚持党对人民军队的绝对领导,深入实施改革强军战略,为如期实现建军一百年奋斗目标、基本实现国防和军队现代化提供有力保障。

    (54)完善人民军队领导管理体制机制。健全贯彻军委主席负责制的制度机制,深入推进政治建军。优化军委机关部门职能配置,健全战建备统筹推进机制,完善重大决策咨询评估机制,深化战略管理创新,完善军事治理体系。健全依法治军工作机制。完善作战战备、军事人力资源等领域配套政策制度。深化军队院校改革,推动院校内涵式发展。实施军队企事业单位调整改革。

    (55)深化联合作战体系改革。完善军委联合作战指挥中心职能,健全重大安全领域指挥功能,建立同中央和国家机关协调运行机制。优化战区联合作战指挥中心编成,完善任务部队联合作战指挥编组模式。加强网络信息体系建设运用统筹。构建新型军兵种结构布局,加快发展战略威慑力量,大力发展新域新质作战力量,统筹加强传统作战力量建设。优化武警部队力量编成。

    (56)深化跨军地改革。健全一体化国家战略体系和能力建设工作机制,完善涉军决策议事协调体制机制。健全国防建设军事需求提报和军地对接机制,完善国防动员体系。深化国防科技工业体制改革,优化国防科技工业布局,改进武器装备采购制度,建立军品设计回报机制,构建武器装备现代化管理体系。完善军地标准化工作统筹机制。加强航天、军贸等领域建设和管理统筹。优化边海防领导管理体制机制,完善党政军警民合力治边机制。深化民兵制度改革。完善双拥工作机制。

    \subsection{提高党对进一步全面深化改革、推进中国式现代化的领导水平}

    党的领导是进一步全面深化改革、推进中国式现代化的根本保证。必须深刻领悟“两个确立”的决定性意义,增强“四个意识”、坚定“四个自信”、做到“两个维护”,保持以党的自我革命引领社会革命的高度自觉,坚持用改革精神和严的标准管党治党,完善党的自我革命制度规范体系,不断推进党的自我净化、自我完善、自我革新、自我提高,确保党始终成为中国特色社会主义事业的坚强领导核心。

    (57)坚持党中央对进一步全面深化改革的集中统一领导。党中央领导改革的总体设计、统筹协调、整体推进。完善党中央重大决策部署落实机制,确保党中央令行禁止。各级党委(党组)负责落实党中央决策部署,谋划推进本地区本部门改革,鼓励结合实际开拓创新,创造可复制、可推广的新鲜经验。走好新时代党的群众路线,把社会期盼、群众智慧、专家意见、基层经验充分吸收到改革设计中来。围绕解决突出矛盾设置改革议题,优化重点改革方案生成机制,坚持真理、修正错误,及时发现问题、纠正偏差。完善改革激励和舆论引导机制,营造良好改革氛围。

    (58)深化党的建设制度改革。以调动全党抓改革、促发展的积极性、主动性、创造性为着力点,完善党的建设制度机制。加强党的创新理论武装,建立健全以学铸魂、以学增智、以学正风、以学促干长效机制。深化干部人事制度改革,鲜明树立选人用人正确导向,大力选拔政治过硬、敢于担当、锐意改革、实绩突出、清正廉洁的干部,着力解决干部乱作为、不作为、不敢为、不善为问题。树立和践行正确政绩观,健全有效防范和纠治政绩观偏差工作机制。落实“三个区分开来”,激励干部开拓进取、干事创业。推进领导干部能上能下常态化,加大调整不适宜担任现职干部力度。健全常态化培训特别是基本培训机制,强化专业训练和实践锻炼,全面提高干部现代化建设能力。完善和落实领导干部任期制,健全领导班子主要负责人变动交接制度。增强党组织政治功能和组织功能。探索加强新经济组织、新社会组织、新就业群体党的建设有效途径。完善党员教育管理、作用发挥机制。完善党内法规,增强党内法规权威性和执行力。

    (59)深入推进党风廉政建设和反腐败斗争。健全政治监督具体化、精准化、常态化机制。锲而不舍落实中央八项规定精神,健全防治形式主义、官僚主义制度机制。持续精简规范会议文件和各类创建示范、评比达标、节庆展会论坛活动,严格控制面向基层的督查、检查、考核总量,提高调研质量,下大气力解决过频过繁问题。制定乡镇(街道)履行职责事项清单,健全为基层减负长效机制。建立经常性和集中性相结合的纪律教育机制,深化运用监督执纪“四种形态”,综合发挥党的纪律教育约束、保障激励作用。

    完善一体推进不敢腐、不能腐、不想腐工作机制,着力铲除腐败滋生的土壤和条件。健全不正之风和腐败问题同查同治机制,深化整治权力集中、资金密集、资源富集领域腐败,严肃查处政商勾连破坏政治生态和经济发展环境问题,完善对重点行贿人的联合惩戒机制,丰富防治新型腐败和隐性腐败的有效办法。加强诬告行为治理。健全追逃防逃追赃机制。加强新时代廉洁文化建设。

    完善党和国家监督体系。强化全面从严治党主体责任和监督责任。健全加强对“一把手”和领导班子监督配套制度。完善权力配置和运行制约机制,反对特权思想和特权现象。推进执纪执法和刑事司法有机衔接。健全巡视巡察工作体制机制。优化监督检查和审查调查机构职能,完善垂直管理单位纪检监察体制,推进向中管企业全面派驻纪检监察组。深化基层监督体制机制改革。推进反腐败国家立法,修改监察法,出台反跨境腐败法。

    (60)以钉钉子精神抓好改革落实。对党中央进一步全面深化改革的决策部署,全党必须求真务实抓落实、敢作善为抓落实,坚持上下协同、条块结合,科学制定改革任务书、时间表、优先序,明确各项改革实施主体和责任,把重大改革落实情况纳入监督检查和巡视巡察内容,以实绩实效和人民群众满意度检验改革。

    中国式现代化是走和平发展道路的现代化。对外工作必须坚定奉行独立自主的和平外交政策,推动构建人类命运共同体,践行全人类共同价值,落实全球发展倡议、全球安全倡议、全球文明倡议,倡导平等有序的世界多极化、普惠包容的经济全球化,深化外事工作机制改革,参与引领全球治理体系改革和建设,坚定维护国家主权、安全、发展利益,为进一步全面深化改革、推进中国式现代化营造良好外部环境。

    全党全军全国各族人民要更加紧密地团结在以习近平同志为核心的党中央周围,高举改革开放旗帜,凝心聚力、奋发进取,为全面建成社会主义现代化强国、实现第二个百年奋斗目标,以中国式现代化全面推进中华民族伟大复兴而努力奋斗。
\end{document}
