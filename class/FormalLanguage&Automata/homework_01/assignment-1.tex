\documentclass[10pt]{homework}

\name{Tao Wenhua} % Replace (Student Name) with your name.
\id{2023010782}
\term{2024 Autumn}
\course{Formal Languages and Automata}
\hwnum{1}

%\hwname{(Name)}          % Uncomment and replace (Name) with the type of the
                          % homework (e.g, Assignment, Problem Set, etc.) if you
                          % don't want the document to be labeled as "Homework."
%\problemname{(Name)}     % Uncomment and replace (Name) with the desired label
                          % for problems created with the problem environment.
%\solutionname{(Name)}    % Uncomment and replace (Name) with the desired label
                          % for solutions created with the solution environment.

% Load any other packages you need here.

\begin{document}


\textit{Note}: You may choose to use LaTeX to complete your homework.
This will benefit you by familiarizing yourself with the key scientific
publishing tool.
If you are new to LaTeX, please read
\href{https://ctan.org/tex-archive/info/lshort/}{this tutorial} first.
You may also choose to use pen and paper to complete your homework and scan
your solutions.

\begin{wrapfigure}{r}{0.45\textwidth}
  \begin{tikzpicture}[node distance=1cm]
    \node[state,initial]   (q0) {$q_0$};
    \node[state,accepting] (q1) [above right=of q0] {$q_1$};
    \node[state,accepting] (q2) [below right=of q0] {$q_2$};

    \path[->] (q0) edge [bend left]  node [above left]  {0} (q1)
                   edge              node [below left]  {1} (q2)
              (q1) edge [bend left]  node [above left]  {0} (q0)
                   edge [loop right] node [right]       {1} ()
              (q2) edge              node [left]        {1} (q1)
                   edge [loop right] node               {0} ();
  \end{tikzpicture}
  \caption{An example of a finite automata.}\label{fig:fa-example}
\end{wrapfigure}

In this and following assignments, you may need to draw pictures of FA's, PDA's,
and TM's in LaTeX.
This may be challenging but \verb|tikz| and the \verb|automata| package are the
tools that can make things easier.
Please find examples at the \href{https://tikz.dev/library-automata}{manual
  page} to get an idea.
On the right (Figure~\ref{fig:fa-example}), we give an example of a finite
automata diagram generated by the automata package.

\vspace{2em}

\begin{problem} (Exercise 2.2.2)
  Let $\hat{\delta}$ be the extended transition function
  defined in class.
  Prove that
  \begin{equation*}
    \hat{\delta} (p, xy) = \hat{\delta} (\hat{\delta} (p, x), y),
  \end{equation*}
  for all state $p$ and strings $x$ and $y$.
\end{problem}

\begin{solution}
    From definition:
    \[
        \hat{\delta}(p,ax)=\hat{\delta}(\delta(p,a),x)
    .\] 
    Where $a\in \Sigma,x\in \Sigma^{*}$, use that to expand $\hat{\delta}(p,xy)$, since $y$ is a suffix of $xy$, we have:
    \[
        \hat{\delta}(p,xy)=\hat{\delta}(p',y)
    .\] 
    Where $p'=\delta(\delta(\ldots\delta(p,x_1),x_2)\ldots,x_n)$, $i$ is the i-th letter of $x$ and $n$ is the length of $x$, from the definition we can also expand $\hat{\delta}(p,x)$ as:
    \[
        \hat{\delta}(p,x)=\delta(\delta(\ldots\delta(p,x_1),x_2)\ldots,x_n)
    .\] 
    So there is:
    \[
        \hat{\delta} (p, xy) = \hat{\delta} (\hat{\delta} (p, x), y),
    .\] 
\end{solution}

\begin{problem}
  Give DFA's accepting the following languages over the alphabet $\{0, 1\}$:
  \begin{parts}
    \part\label{2.a} (Exercise 2.2.4 (b)) The set of all strings that has three
    consecutive $0$'s (not necessarily at the end).
    \part\label{2.b} (Exercise 2.2.4 (c)) The set of strings with $011$ as a substring.
    \part\label{2.c} (Exercise 2.2.5 (b)) The set of all strings whose tenth symbol from
    the right end is a $1$.
    \part\label{2.d} (Exercise 2.2.5 (d)) The set of strings such that the number of
    $0$’s is divisible by five, and the number of $1$’s is divisible by $3$.
  \end{parts}
\end{problem}

\begin{solution}

  \ref{2.a} % chktex 2
  Answer like:
  \begin{table}[h]
    \centering
    \begin{tabular}{r||c|c} % chktex 44
      & $0$ & $1$\\\hline\hline  % chktex 44
      $\rightarrow 0$ & 1 & 0\\
      1 & 2 & 0\\
      2 & 3 & 0\\
      *3 & 3 & 3
    \end{tabular}
  \end{table}

  \ref{2.b} % chktex 2
  Answer like:
  \begin{table}[h]
    \centering
    \begin{tabular}{r||c|c} % chktex 44
      & $0$ & $1$\\\hline\hline  % chktex 44
      $\rightarrow 0$ & 1 & 0\\
      1 & 0 & 2\\
      2 & 0 & 3\\
      *3 & 3 & 3
    \end{tabular}
  \end{table}

  \ref{2.c} % chktex 3
  $A=\{Q,\Sigma,\delta,q_0,F\}$ is the answer, where we define:
  \begin{align}
      Q&=\{x|x\in \mathbb{N},x<2^{10}\}\\
      \Sigma&=\{0,1\}\\
      \delta(q,a)&=q\times 2+a \mod 2^{10}\\
      q_0&=0\\
      F&=\{x|x\in \mathbb{N},2^{9}\le x<2^{10}\}
  \end{align}
  

  \ref{2.d} % chktex 4
  Answer like:
  \begin{table}[h]
    \centering
    \begin{tabular}{r||c|c} % chktex 44
      & $0$ & $1$\\\hline\hline  % chktex 44
      $\rightarrow$ *$(0,0)$ & $(1,0)$ & $(0,1)$\\
      $(1,0)$ & $(2,0)$ & $(1,1)$\\
      $(2,0)$ & $(3,0)$ & $(2,1)$\\
      $(3,0)$ & $(4,0)$ & $(3,1)$\\
      $(4,0)$ & $(0,0)$ & $(4,1)$\\
      $(0,1)$ & $(1,1)$ & $(0,2)$\\
      $(1,1)$ & $(2,1)$ & $(1,2)$\\
      $(2,1)$ & $(3,1)$ & $(2,2)$\\
      $(3,1)$ & $(4,1)$ & $(3,2)$\\
      $(4,1)$ & $(0,1)$ & $(4,2)$\\
      $(0,2)$ & $(1,2)$ & $(0,0)$\\
      $(1,2)$ & $(2,2)$ & $(1,0)$\\
      $(2,2)$ & $(3,2)$ & $(2,0)$\\
      $(3,2)$ & $(4,2)$ & $(3,0)$\\
      $(4,2)$ & $(0,2)$ & $(4,0)$\\
    \end{tabular}
  \end{table}
\end{solution}

\begin{problem} (Exercise 2.2.9) Let
  $A = (Q, \Sigma, \delta, q_{0}, \{q_{f} \})$ be a DFA, and suppose that for
  all $a$ in $\Sigma$ we have $\delta(q_{0}, a) = \delta(q_{f}, a)$.
  \begin{parts}
      \part\label{3.a} Show that for all $w \ne \epsilon$, we have
    $\hat{\delta}(q_{0}, w) = \hat{\delta}(q_{f}, w)$.
    \part\label{3.b} Show that if $x$ is a nonempty string in $L(A)$, then for all
    $k > 0$, $x^{k}$ (i.e., $x$ written $k$ times) is also in $L(A)$.
  \end{parts}
\end{problem}

\begin{solution}

    \ref{3.a}
    Since $w\ne\epsilon$ there is $w=ax,a\in\Sigma,x\in\Sigma^{*}$, from the definition of $\hat{\delta}$ we can expand as:
    \[
    \hat{\delta}(q_0,w)=\hat{\delta}(\delta(q_0,a),x)
    .\] 
    Also:
    \[
    \hat{\delta}(q_f,w)=\hat{\delta}(\delta(q_f,a),x)
    .\] 
    Then we apply $\forall a\in\Sigma,\delta(q_0,a)=\delta(q_f,a)=q_a$, and $\hat{\delta}(q_a,x)=\hat{\delta}(q_a,x)$ apparently, so:
    \[
    \hat{\delta}(q_0,w)=\hat{\delta}(q_f,w)
    .\] 

    \ref{3.b}
    $x\in L(A)$ means that $\hat{\delta}(q_0,x)=q_f$, from Problem 1 we can declare that $\forall k>1$ :
    \[
    \hat{\delta}(q_0,x^{k})=\hat{\delta}(\hat{\delta}(q_0,x),x^{k-1})=\hat{\delta}(q_f,x^{k-1})
    .\] 
    Then from Problem 3.a we have:
    \[
    \hat{\delta}(q_f,x^{k-1})=\hat{\delta}(q_0,x^{k-1})
    .\] 
    So we have $\hat{\delta}(q_0,x^{k})=\hat{\delta}(q_0,x^{k-1})$ for all $k>1$. Since we already know that for $k=1,\hat{\delta}(q_0,x)=q_f$, we can do induction that $\forall k>0,\hat{\delta}(q_0,x^{k})=q_f$, which means that $x^{k}\in L(A)$.

\end{solution}

\begin{problem} (Exercise 2.3.2)
  Convert to a DFA the following NFA:\@
  \begin{table}[h]
    \centering
    \begin{tabular}{r||c|c} % chktex 44
      & $0$ & $1$\\\hline\hline  % chktex 44
      $\rightarrow p$ & $\{q, s\}$ & $\{q\}$\\
      $*q$ & $\{r\}$ & $\{q, r\}$\\
      $r$ & $\{s\}$ & $\{p\}$\\
      $*s$ & $\emptyset$ & $\{p\}$
    \end{tabular}
  \end{table}
\end{problem}

\begin{solution}
    Do subset construction we can build DFA like \hyperref[tab:problem-4]{Table 4.1}. Notice that some states can be deleted because they can't be reached from start $\{q\}$,like $\emptyset$ or $*\{p,q,r,s\}$
  \begin{table}[!h]
    \centering
    \label{tab:problem-4}
    \begin{tabular}{r||c|c} % chktex 44
      & $0$ & $1$\\\hline\hline  % chktex 44
        $\emptyset$ & $\emptyset$ & $\emptyset$\\
        $\rightarrow \{p\}$ & $\{q, s\}$ & $\{q\}$\\
        $*\{q\}$ & $\{r\}$ & $\{q, r\}$\\
        $\{r\}$ & $\{s\}$ & $\{p\}$\\
        $*\{s\}$ & $\emptyset$ & $\{p\}$\\
        $*\{p,q\}$ & $\{q,r,s\}$ & $\{q,r\}$\\
        $\{p,r\}$ & $\{q,s\}$ & $\{p,q\}$\\
        $*\{p,s\}$ & $\{q,s\}$ & $\{p,q\}$\\
        $*\{q,r\}$ & $\{r,s\}$ & $\{p,q,r\}$\\
        $*\{q,s\}$ & $\{r\}$ & $\{p,q,r\}$\\
        $*\{r,s\}$ & $\{s\}$ & $\{p\}$\\
        $*\{q,r,s\}$ & $\{r,s\}$ & $\{p,q,r\}$\\
        $*\{p,r,s\}$ & $\{q,s\}$ & $\{p,q\}$ \\
        $*\{p,q,s\}$ & $\{q,r,s\}$ & $\{p,q,r\}$ \\
        $*\{p,q,r\}$ & $\{q,r,s\}$ & $\{p,q,r\}$\\
        $*\{p,q,r,s\}$ & $\{q,r,s\}$ & $\{p,q,r\}$
    \end{tabular}
    \caption{Answer DFA of Problem 4}
  \end{table}
\end{solution}

\newpage

\begin{problem} (Exercise 2.3.4) Give nondeterministic finite automata to accept
  the following languages.
  Try to take advantage of nondeterminism as much as possible.
  \begin{parts}
      \part\label{5.a} The set of strings over alphabet $\{0, 1, \ldots, 9\}$ such that the
    final digit has not appeared before.
    \part\label{5.b} The set of strings of $0$'s and $1$'s such that there are two $0$'s
    separated by a number of positions that is a multiple of $4$.
    Note that $0$ is an allowable multiple of $4$.
  \end{parts}
\end{problem}

\begin{solution}

    \ref{5.a}
    Answer like $A=(Q,\Sigma,\delta,q_0,F)$, where the automata choose the state named by the final digit from $s$. If there is only one digit in $s$, the automata jumps to $f$ directly.
    \[
    \begin{aligned}
        Q=&\{s,0,1,\ldots,9,f\}\\
        \Sigma=& \{0,1,\ldots,9\}\\
        \delta(s,a)=&(\Sigma\textbackslash \{a\})\cup \{f\},\forall a\in\Sigma\\
        \delta(i,a)=&\{i\},\forall a\ne i,i\in \{0,1,\ldots,9\}\\
        \delta(i,a)=&\{f\},\forall a=i,i\in \{0,1,\ldots,9\}\\
        \delta(f,a)=&\emptyset,\forall a\in\Sigma\\
        q_0=&s\\
        F=&\{f\}
    \end{aligned}
    .\] 

    \ref{5.b}
    Answer like \hyperref[tab:problem-5-b]{Table 5.1}, where the automata jump to state $4$ at the first zero of the legal zeros pair in the problem.

      \begin{table}[!h]
        \centering
        \label{tab:problem-5-b}
        \begin{tabular}{r||c|c} % chktex 44
          & $0$ & $1$\\\hline\hline  % chktex 44
            $\rightarrow 0$ & $\{0,4\}$ & $\{0\}$\\
            $1$ & $\{2\}$ & $\{2\}$\\
            $2$ & $\{3\}$ & $\{3\}$\\
            $3$ & $\{4\}$ & $\{4\}$\\
            $4$ & $\{5\}$ & $\{1\}$\\
            $*5$& $\emptyset$ & $\emptyset$\\
        \end{tabular}
        \caption{Answer NFA of Problem 5.b}
      \end{table}

\end{solution}

\end{document}
