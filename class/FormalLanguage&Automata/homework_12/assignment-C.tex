\documentclass[10pt]{homework}

\name{Tao Wenhua} % Replace (Student Name) with your name.
\id{2023010782}
\term{2024 Autumn}
\course{Formal Languages and Automata}
\hwnum{12}

%\hwname{(Name)}          % Uncomment and replace (Name) with the type of the
                          % homework (e.g, Assignment, Problem Set, etc.) if you
                          % don't want the document to be labeled as "Homework."
%\problemname{(Name)}     % Uncomment and replace (Name) with the desired label
                          % for problems created with the problem environment.
%\solutionname{(Name)}    % Uncomment and replace (Name) with the desired label
                          % for solutions created with the solution environment.

% Load any other packages you need here.

\begin{document}

\begin{problem} (Exercise 9.2.1) Show that the halting problem---the set of
  \((M, w)\) pairs such that \(M\) halts (with or without accepting) when given
  input \(w\)---is RE but not recursive.
\end{problem}

\begin{solution}
    Use universal TM $U$ with input $\left( M,w \right) $. If $U$ halts, that equals to $M$ halts. So $U$ (with all states accepting) can determine halting problem. The problem is RE.

    For recursive we can reduce universal language problem to halting problem by making all TMs reject by jumping into a infinity loop. Since $L_{u}$ is not recursive, halting problem is not recursive.
\end{solution}

\begin{problem}
  (Exercise 9.2.5) Let \(L\) be recursively enumerable and let \(\overline{L}\)
  be non-RE\@.
  Consider the language
  \begin{equation*}
    L' = \{0w \mid w \text{ is in } L\} \cup \{1w \mid w \text{ is not in } L\}.
  \end{equation*}
  Can you say for certain whether \(L'\) or its complement are recursive, RE, or
  non-RE\@?
  Justify your answer.
\end{problem}

\begin{solution}
    $L'$ is non-RE because non-RE problem $\overline{L}$ can be reduce to $L'$ by determine whether $1w$ in $L'$.

    $\overline{L'}=\{w\ |\ w\text{ not begin with }0\text{ or }1\}\cup\{0w\ |\ w\notin L\}\cup\{1w\ |\ w\in L\}$ is non-RE because non-RE problem $\overline{L}$ can be reduce to $\overline{L'}$ by determine whether $0w$ in $\overline{L'}$.
\end{solution}

\begin{problem}
  We know by Rice's theorem that none of the following problems are decidable.
  However, are they recursively enumerable or non-RE\@?
  \begin{parts}
    \part\ (Exercise 9.3.4 (a)) Does \(L(M)\) contain at least two strings?
    \part\ (Exercise 9.3.4 (b)) Is \(L(M)\) infinite?
  \end{parts}
\end{problem}

\begin{solution}
    \begin{parts}
        \part The problem is RE. Because we can use a universal TM with all pairs $\left( M,w \right),w\in \Sigma^{*}$ input until it finds two accepted strings. If $L\left( M \right) $ contains at least two strings, the TM must halt and accept, otherwise it will not halt.
        \part The problem is non-RE. Since halting problem (with arbitrary inputs) is non-RE, and we can reduce standard halting problem to the problem above, by building a TM $M_{i}$ accepts $\left( M,w \right)$'s ID sequence prefixes. If $M$ with input $w$ halts, ID sequence is limited, $L\left( M_{i} \right) $ is limited. So if we can determine whether $L\left( M_{i} \right) $ is infinite, we can determine whether $M$ with input $w$ halts.
    \end{parts}
\end{solution}

\begin{problem}
  (Exercise 9.3.6 (b)) Show that the following question is decidable: The set of codes for TMs that never make a move left on any input.
\end{problem}

\begin{solution}
    When a TM cannot move left, the input on the tape is only read once from left to right just like DFA does. So after replacing the move left transitions by transitions to a rejected state, the TM can be reduced to a DFA. Then in DFA we can do searching like DFS or BFS to determine whether the new rejected state (reached by move left transitions) in $O\left( n+m \right) $ time, where $n$ is the state number and $m$ is the transition number. All things above can be done by a TM in limited steps, so the question is decidable.
\end{solution}

% \begin{problem}
%   (Exercise 9.4.2) We showed that PCP was undecidable, but we assumed that the
%   alphabet \(\Sigma\) could be arbitrary.
%   Show that PCP is undecidable even if we limit the alphabet to
%   \(\Sigma = \{0,1\}\) by reducing PCP to this special case of PCP\@.
% \end{problem}

% \begin{solution}

% \end{solution}

\end{document}
