\documentclass[10pt]{homework}

\name{Tao Wenhua} % Replace (Student Name) with your name.
\id{2023010782}
\term{2024 Autumn}
\course{Formal Languages and Automata}
\hwnum{5}

%\hwname{(Name)}          % Uncomment and replace (Name) with the type of the
                          % homework (e.g, Assignment, Problem Set, etc.) if you
                          % don't want the document to be labeled as "Homework."
%\problemname{(Name)}     % Uncomment and replace (Name) with the desired label
                          % for problems created with the problem environment.
%\solutionname{(Name)}    % Uncomment and replace (Name) with the desired label
                          % for solutions created with the solution environment.

% Load any other packages you need here.

\begin{document}

% \begin{problem}
%   Suppose $h$ is the homomorphism from the alphabet $\{0, 1, 2\}$ to the
%   alphabet $\{a,b\}$ defined by: $h(0) = a$, $h(1) = ab$, and $h(2) = ba$.
%   \begin{parts}
%     \part\ (Exercise 4.2.1 (d)) If $L$ is the language $L(0+12)$, what is
%     $h(L)$?
%     \part\ (Exercise 4.2.1 (f)) If $L$ is the language $L(a{(ba)}^{*})$, what is
%     $h^{-1}(L)$?
%   \end{parts}
% \end{problem}

\begin{problem}
  Lazy evaluation in subset state construction when converting an NFA to a DFA
  can often be more efficient than using the straightforward subset state
  conversion method.
  One notable example is the keyword search problem discussed in textbooks,
  where the resulting DFA has no more states than the original NFA\@.
  Perform the lazy evaluation variant of the subset state method to convert the
  following NFA for keyword \textit{fa} and \textit{dfa} to an equivalent DFA
  with the same number of states.
  In the diagram, $\Sigma = \{a,b,\ldots, z\}$ is the alphabet.

  \begin{center}
    \begin{tikzpicture}[node distance=1cm]
      \node[state, initial] (q1) {$1$};
      \node[state] (q2) [above right=.5cm and 1cm of q1] {$2$};
      \node[state, accepting] (q3) [right=of q2] {$3$};
      \node[state] (q4) [below right=.5cm and 1cm of q1] {$4$};
      \node[state] (q5) [right=of q4] {$5$};
      \node[state, accepting] (q6) [right=of q5] {$6$};
      \path[->] (q1) edge[loop above] node[above] {$\Sigma$} ()
                     edge node[above] {$f$} (q2)
                     edge node[above] {$d$} (q4)
                     (q2) edge node[above] {$a$} (q3)
                     (q4) edge node[above] {$f$} (q5)
                     (q5) edge node[above] {$a$} (q6);
    \end{tikzpicture}
  \end{center}
\end{problem}

\begin{solution}
    The table below shows the subset founded in lazy evaluation in order:
    \begin{figure}[htbp]
        \centering
        \begin{tabular}{c||c|c|c}
             & $d$ & $f$ & $a$ \\\hline\hline
             $\rightarrow \{1\}$ & $\{1,4\}$ & $\{1,2\}$ & $\{1\}$ \\
             $\{1,4\}$ & $\{1,4\}$ & $\{1,2,5\}$ & $\{1\}$ \\
             $\{1,2\}$ & $\{1,4\}$ & $\{1,2\}$ & $\{1,3\}$ \\
             $\{1,2,5\}$ & $\{1,4\}$ & $\{1,2\}$ & $\{1,3,6\}$ \\
             $*\{1,3\}$ & $\{1,4\}$ & $\{1,2\}$ & $\{1\}$ \\
             $*\{1,3,6\}$ & $\{1,4\}$ & $\{1,2\}$ & $\{1\}$
        \end{tabular}
    \end{figure}
\end{solution}

\begin{problem}
  We can use closure properties to help prove certain languages are not regular.
  Starting with the fact that the language
  \begin{equation*}
    L_{0n1n} = \{ 0^{n}1^{n} \mid n \ge 0 \}
  \end{equation*}
  is not a regular set, prove the following languages are not regular by
  transforming them, using operations known to preserve regularity:
  \begin{parts}
    \part\ (Exercise 4.2.13 (a)) $\{0^{i} 1^{j} \mid i \ne j\}$.
    \part\ (Exercise 4.2.13 (b)) $\{ 0^{n}1^{m} 2^{n-m} \mid n \ge m \ge 0\}$.
  \end{parts}
\end{problem}

\begin{solution}
    \begin{parts}
        \part The language is $\overline{L}_{0n1n}$. So if the language is regular, $L_{0 n 1 n}=\overline{\overline{L}}_{0 n 1 n}$ is regular, contradict.
        \part Use homomorphism $h(0)=0,h(1)=h(2)=1$, then if the language $L$ is regualar, $L_{0 n 1 n}=h(L)$ is regular, contradict.
    \end{parts}
\end{solution}

\begin{problem} (Exercise 4.3.2) Give an algorithm to tell whether a regular
  language $L$ contains at least $100$ strings.
\end{problem}

\begin{solution}
    Calculate by induction:

    Basis: $\epsilon$ and all letters in $\Sigma$ contains only one string, $\phi$ contains no string.

    Induction: if $E$ and $F$ are RE contain strings $a$ and $b$,then:
    \begin{itemize}
        \item $E+F$ contains $a+b$ strings.
        \item $EF$ contains $a\times b$ strings.
        \item $E^{*}$ contains infinity strings.
    \end{itemize}
    From algorithm above, we can directly calculate the number of strings from RE of the RL. Then just simply compare that with 100 we can get the answer.
\end{solution}

\begin{problem}
  A finite automaton is given as the following transition table:
  \begin{table}[H]
    \centering
    \begin{tabular}{r||c|c} % chktex 44
      & $0$ & $1$\\\hline\hline  % chktex 44
      $\rightarrow A$ & $B$ & $E$\\
      $B$ & $C$ & $F$\\
      $*C$ & $D$ & $H$\\
      $D$ & $E$ & $H$\\
      $E$ & $F$ & $I$\\
      $*F$ & $G$ & $B$\\
      $G$ & $H$ & $B$\\
      $H$ & $I$ & $C$\\
      $*I$ & $A$ & $E$\\
    \end{tabular}
  \end{table}
  \begin{parts}
    \part\ Draw the table of distinguishable pairs for this automaton.
    \part\ Construct the minimum-state equivalent DFA\@.
  \end{parts}
\end{problem}

\begin{solution}
    \begin{parts}
        \part Answer like below, $\times ,\Delta$ in order of signment:
        \begin{figure}[htbp]
            \centering
            \begin{tabular}{c||c|c|c|c|c|c|c|c|c}
                 & $A$ & $B$ & $C$ & $D$ & $E$ & $F$ & $G$ & $H$ & $I$ \\\hline\hline
                $A$ &  & $\Delta$ & $\times $ &  & $\Delta$ & $\times $ &  & $\Delta$ & $\times $ \\\hline
                $B$ & $\Delta$ &  & $\times $ & $\Delta$ &  & $\times $ & $\Delta$ &  & $\times $ \\\hline
                $C$ & $\times $ & $\times $ &  & $\times $ & $\times $ &  & $\times $ & $\times $ &  \\\hline
                $D$ &  & $\Delta$ & $\times $ &  & $\Delta$ & $\times $ &  & $\Delta$ & $\times $ \\\hline
                $E$ & $\Delta$ &  & $\times $ & $\Delta$ &  & $\times $ & $\Delta$ &  & $\times $ \\\hline
                $F$ & $\times $ & $\times $ &  & $\times $ & $\times $ &  & $\times $ & $\times $ &  \\\hline
                $G$ &  & $\Delta$ & $\times $ &  & $\Delta$ & $\times $ &  & $\Delta$ & $\times $ \\\hline
                $H$ & $\Delta$ &  & $\times $ & $\Delta$ &  & $\times $ & $\Delta$ &  & $\times $ \\\hline
                $I$ & $\times $ & $\times $ &  & $\times $ & $\times $ &  & $\times $ & $\times $ &  \\\hline
            \end{tabular}
        \end{figure}
        \newpage
        \part From solution above, we can union $A,D,G$ and $B,E,H$ and $C,F,I$. so the minimum-state DFA like:
        \begin{figure}[htbp]
            \centering
            \begin{tabular}{c||c|c}
                 & $0$ & $1$ \\\hline\hline
                 $\rightarrow A$ & $B$ & $B$ \\
                 $B$ & $C$ & $C$\\
                 $*C$ & $A$ & $B$
            \end{tabular}
        \end{figure}
    \end{parts}
\end{solution}

% \begin{problem}
%   Define mapping $h: \{a, b\} \rightarrow {\{0,1\}}^*$ as $h(a) = \eps$ and
%   $h(b)= 10$.
%   Define regular expression $E = \eps + (a+b){(ba)}^*$.
%   \begin{parts}
%     \part\ Give a regular expression $E^R$ such that $L(E^R) = {(L(E))}^R$.
%     \part\ Give a regular expression $h(E)$ such that $L(h(E)) = h(L(E))$.
%     \part\ Construct a DFA $A$, such that $L(A)$ is the complement of $L(h(E))$.
%    \end{parts}
% \end{problem}

\begin{problem}
  Define map $h: \{0, 1\} \rightarrow {\{a, b\}}^*$ as $h(0) = ab$ and
  $h(1) = ba$.
  Given an automata $A$ as in the figure.
  Construct DFA B over $\{0,1\}$, such that $L(B) = h^{-1}(L(A))$.

  \begin{figure}[H]
    \centering
    \begin{tikzpicture}[node distance=2cm]
      \node[state, initial] (q1) {};
      \node[state, accepting, above right of=q1] (q2) {};
      \node[state, accepting, below right of=q1] (q3) {};

      \draw (q1) edge[bend left, above] node{$a$} (q2)
      (q2) edge[bend left, below] node{$b$} (q1)
      (q1) edge[below] node{$b$} (q3)
      (q3) edge[right] node{$a$} (q2)
      (q2) edge[right, loop right, looseness=10] node{$a$} (q2)
      (q3) edge[right, loop right, looseness=10] node{$b$} (q3);
    \end{tikzpicture}
  \end{figure}
\end{problem}

\begin{solution}
    Use the definition $\delta_1(q_1,0)=\delta(\delta(q,a),b)$ and $\delta_1(q_1,1)=\delta(\delta(q,b),a)$ to shrink the DFA like:
    \begin{figure}[htbp]
        \centering
        \begin{tabular}{c||c|c}
             & $0$ & $1$ \\\hline\hline
             $\rightarrow A$ & $A$ & $B$ \\
             $*B$ & $A$ & $B$ \\
        \end{tabular}
    \end{figure}

    Notice that the node in the right down side of problem graph is unreachable under $h$ so it does not appear.
\end{solution}

% \begin{problem}
%   Minimize the following DFA using the table-filling algorithm.

%   \begin{figure}[H]
%     \centering
%     \begin{tikzpicture}[node distance=1.3cm]
%       \node[state, initial, accepting] (q1) {$1$};
%       \node[state, above right=of q1] (q2) {$2$};
%       \node[state, accepting, below right=of q1] (q3) {$3$};
%       \node[state, node distance=2cm, right of=q2] (q4) {$4$};
%       \node[state, node distance=2cm, right of=q3] (q5) {$5$};
%       \node[state, accepting, below right=of q4] (q6) {$6$};

%       \draw[->] (q1) edge[above] node{$a$} (q2)
%       (q1) edge[below] node{$b$} (q3)
%       (q2) edge[above] node{$a$} (q4)
%       (q2) edge[above] node{$b$} (q5)
%       (q3) edge[left] node{$a$} (q2)
%       (q3) edge[loop right, right] node{$b$} (q3)
%       (q4) edge[loop above, above] node{$a$} (q4)
%       (q4) edge[bend left, above] node{$b$} (q6)
%       (q5) edge[left] node{$a$} (q4)
%       (q5) edge[below] node{$b$} (q6)
%       (q6) edge[bend left, below] node{$a$} (q4)
%       (q6) edge[loop right, right] node{$b$} (q6);
%     \end{tikzpicture}
%   \end{figure}
% \end{problem}

\end{document}
