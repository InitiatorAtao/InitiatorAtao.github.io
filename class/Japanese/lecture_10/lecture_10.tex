\documentclass[12pt]{article}

\author{陶文华}

\usepackage{../../lectures_preamble}

\begin{document}
    \section{新しい単語}
    \begin{figure}[htbp]
        \centering
        \begin{tabular}{l|l|l}
            がいこくご〔外国語〕 & (名) & 外语\\\hline
            しんゆう〔親友〕 & (名) & 亲密的朋友\\\hline
            とうきゅうせい〔同級生〕 &  (名) & 同年级同学\\\hline
            てんしんたいがく〔天津大学〕 & (名) & 天津大学\\\hline
            ある & (连体)  &某  \\\hline
            ひさしぶり〔久しふり〕& (名)   &许久   \\\hline
            まちあわせる〔待ち合わせる〕 &  (他一) &约会,碰头\\\hline
            てんあんもん〔天安門〕& (名)  &  天安冂   \\\hline
            しやしん〔写真〕 &(名) &   照片   \\\hline
            とる〔撮る〕  & (他五) & 拍照    \\\hline
            ワンフーチン〔王府井〕& (名) &  王府井  \\\hline
            あるく〔歩く〕& (自五) & 走\\\hline
            きっさてん〔喫茶店〕 &  (名)  &  咖啡馆 \\\hline
            コーヒー  & (名)  &  咖啡\\\hline
            ちゅうもん〔注文〕& (他サ)  & 订购,订货   \\\hline
            ミルク& (名)  &  牛奶\\\hline
            さとう〔砂糖〕& (名)  &  糖,砂糖 \\\hline
            いれる〔入れる〕 &  (他一) & 放入\\\hline
            コート   & (名) & 外套,大衣   \\\hline
            ほしいしい〕 &  (形)  &  想要    \\\hline
            かわり〔代わり〕 &  (名) &   代替    \\\hline
            セーター  & (名)  &  毛衣    \\\hline
            こうえん〔公園〕 &  (名) &   公同    \\\hline
            しようらい〔将来〕& (名)  &  将来    \\\hline
            かたりあう〔語り合う〕& (他五)  & 交谈\\\hline   
            たのしい〔楽しい〕& (形) &   快乐\\\hline
        なる &(自五) & 成为    \\\hline
        アナウンサー & (名) & 播音员\\\hline
        そっきよう〔卒業〕& (名、自サ)&毕业\\\hline
        だいがくいん〔大学院〕& (名) &   研究生院   \\\hline
        エムビーエー〔MBA〕& (名) &   MBA(工商管理硬士)	\\\hline
        とる〔取る〕  & (他五)	&取\\\hline
        すごい	&(形)	&厉害,非常	\\\hline
        もっと	&(副)&	更加	\\\hline
        おたがい〔お互い〕	&(名)	&互相	\\\hline
        がんばる〔頑張る〕&	(自五)&	拼命努力,加油\\\hline
        \end{tabular}
    \end{figure}
    \section{本文}
    王梅芳さんは北京外国語大学の3年生です。王さんには親友がいます。高校時代の同級生のさんです。楊さん天津大学の3年生です。

    ある冬の土曜日、2人は久しぶりに会いました。2人は9時に北京駅で待ち合わせました。

    2人はまず天安門まで地下鉄で行きました。そこで写真を撮りました。天安門から王府井まで歩きました。喫茶店でコーヒーを注文しました。楊さんはコーヒーミルクも砂糖も入れません。

    そのあと、王さんはコートがほしかったので、デパトへ買いに行きました。いいコートがありましたが、高かったので買いませんでした。代わりにセーターを買いました。

    それから公園で将来の夢を語り合いました。とても寒さむかったです。でも楽しかったです。
    \section{会話}
    孫:松下さんは将来、何になりたいてすか。

    松下:わたしはアナウンサーになりたいてす。ても難しいてしようね。孫さんは?

    孫:僕は卒業後、アメリカに留学したいです。大学院でもMBAを取りたいです。

    松下:すごいですね。わたしはもっと中国語が上手になりたい
です。

    孫:お互いにがんばりましよう。
\end{document}
