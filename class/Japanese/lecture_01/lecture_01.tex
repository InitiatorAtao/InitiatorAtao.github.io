\documentclass[12pt]{article}

\author{陶文华}

\usepackage{../../lectures_preamble}

\begin{document}
    \section{五十音}
        \subsection{五十音表}
        括号内为输入法罗马音键位,也对应于大致读音.括号右侧是片假名.
        \subsubsection{清音}
        \begin{figure}[htbp]
            \centering
            \begin{tabular}{c||c|c|c|c|c}
                 & a & i & u & e & o\\\hline\hline
                 & あ(a)ア & い(i)イ & う(u)ウ & え(e)エ & お(o)オ\\\hline
                k & か(ka)カ & き(ki)キ & く(ku)ク & け(ke)ケ & こ(ko)コ\\\hline
                s & さ(sa)サ & し(shi)シ & す(su)ス & せ(se)セ & そ(so)ソ\\\hline
                t & た(ta)タ & ち(chi)チ & つ(tsu)ツ & て(te)テ & と(to)ト\\\hline
                n & な(na)ナ & に(ni)ニ & ぬ(nu)ヌ & ね(ne)ネ & の(no)ノ\\\hline
                h & は(ha)ハ & ひ(hi)ヒ & ふ(hu)フ & へ(he)ヘ & ほ(ho)ホ\\\hline
                m & ま(ma)マ & み(mi)ミ & む(mu)ム & め(me)メ & も(mo)モ\\\hline
                y & や(ya)ヤ & い(i)イ & ゆ(yu)ユ & え(e)エ & よ(yo)ヨ\\\hline
                r & ら(ra)ラ & り(ri)リ & る(ru)ル & れ(re)レ & ろ(ro)ロ\\\hline
                w & わ(wa)ワ & い(i)イ & う(u)ウ & え(e)エ & を(wo)ヲ\\\hline
                 & ん(nn)ン
            \end{tabular}
        \end{figure}
        \subsubsection{浊音/半浊音}
        p 行为半浊音.
        \begin{figure}[htbp]
            \centering
            \begin{tabular}{c||c|c|c|c|c}
                 & a & i & u & e & o\\\hline\hline
                k-g & が(ga) & ぎ(gi) & ぐ(gu) & げ(ge) & ご(go)\\\hline
                s-z & ざ(za) & じ(ji) & ず(zu) & ぜ(ze) & ぞ(zo)\\\hline
                t-d & だ(da) & ぢ(di) & づ(du) & で(de) & ど(do)\\\hline
                h-b & ば(ba) & び(bi) & ぶ(bu) & べ(be) & ぼ(bo)\\\hline
                h-p & ぱ(pa) & ぴ(pi) & ぷ(pu) & ぺ(pe) & ぽ(po)\\\hline
            \end{tabular}
        \end{figure}
        \subsubsection{拗音}
        注音无特例,直接按照对应行列元辅音输入即可.
        \begin{figure}[htbp]
            \centering
            \begin{tabular}{c||c|c|c}
                 & や(ya) & ゆ(yu) & よ(yo)\\\hline\hline
                 k & きゃ & きゅ & きょ\\\hline
                 s & しゃ & しゅ & しょ\\\hline
                 t & ちゃ & ちゅ & ちょ\\\hline
                 n & にゃ & にゅ & にょ\\\hline
                 h & ひゃ & ひゅ & ひょ\\\hline
                 m & みゃ & みゅ & みょ\\\hline
                 r & りゃ & りゅ & りょ\\\hline
                 k-g & ぎゃ & ぎゅ & ぎょ\\\hline
                 s-z & じゃ & じゅ & じょ\\\hline
                 t-d & ぢゃ & ぢゅ & ぢょ\\\hline
                 h-b & びゃ & びゅ & びょ\\\hline
                 h-p & ぴゃ & ぴゅ & ぴょ\\\hline
            \end{tabular}
        \end{figure}
\end{document}
