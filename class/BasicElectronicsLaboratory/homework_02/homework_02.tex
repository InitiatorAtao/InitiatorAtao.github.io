\documentclass[12pt]{article}

\author{陶文华}

\usepackage{../../homeworks_preamble}
\title{电子学基础实验二报告}

\begin{document}
    \pagestyle{empty}
    \maketitle
    \section{实验内容}
        使用给出的元件及便携式实验设备,按图 \ref{figure:01} 接线,完成以下任务,截屏记录数据,并对实验得到的曲线要做必要解释.
        \begin{figure}[htbp]
            \centering
            \includegraphics[width=0.4\linewidth]{./figures/homework_02_figure_01.pdf}
            \caption{实验接线电路图}
            \label{figure:01}
        \end{figure}
        \begin{enumerate}
            \item 输入端口特性: $U_{GS} – I_{GS}$ (其中 $U_{S}=5V$ 固定不变, $U_{GS}$ 变化范围 $0-5V$),用示波器的 Y/X 功能显示输入特性曲线,并根据实验结果计算端口 $GS$ 的输入电阻(理想情况下,压控型器件的控制端口应该是开路的).
            \item 转移特性: $U_{GS} – I_{DS}$ (其中 $U_{D}=5V$, $U_{GS}$ 变化范围 $0-5V$)(测量的不是同一个端口的电压电流,故称为转移特性),注意使得端口 $DS$ 之间导通的 $U_{GS}$ 的最小值(即阈值电压 $U_{T}$).用示波器的 Y/X 功能显示转移特性曲线.
            \item 输出端口特性: $U_{DS} – I_{DS}$ ,参考实验内容 (2) 选取合适的 $U_{GS}$,使得当 $U_{DS}$ 在 $0-5V$ 变化时, $U_{DS} – I_{DS}$ 端口特性能够完整地体现 MOSFET 的三个工作区,即截止区,电流源区和电阻区,用示波器的 Y/X 功能显示输出端口特性曲线.
        \end{enumerate}
    \section{原始数据}
    原始数据为由便携式实验设备的示波器 Y/X 功能记录的 \texttt{csv} 文件数据,电流值通过电流采样电阻上测得的电压值计算得到,经 \texttt{Excel} 绘制散点图展示如下.(完整数据及分析可从\href{https://cloud.tsinghua.edu.cn/d/72e4a2c87e694f9d9a4d/}{此处}下载)
        \begin{figure}[htbp]
            \centering
            \exinkfig{homework_02_figure_02}{
                \caption{输入端口特性 $U_{GS}-I_{GS}$ 图像}
            }
            \vspace{20pt}
            \exinkfig{homework_02_figure_03}{
                \caption{转移特性 $U_{GS}-I_{DS}$ 图像}
            }
        \end{figure}
        \begin{figure}[htbp]
            \centering
            \exinkfig{homework_02_figure_04}{
                \caption{多个 $U_{GS}$ 值对应的输出端口特性 $U_{DS}-I_{DS}$ 图像}
            }
        \end{figure}
    \section{数据分析}
        \begin{enumerate}
            \item 对于输入端口特性,由于 MOSFET 的特性, G 端与 S 端之间等效电阻极大,在 $0-5V$ 的电压下电流极小,因此在 $U-I$ 图像下显示为一条 $x$ 轴上 $0-5V$ 之间的线段.要求计算 $GS$ 的输入电阻,由于 $I_{GS}$ 的测量值受采样精度的影响误差较大,去除 $I_{GS}$ 采样为 0 的数据后求取平均电阻约为 $1584\Omega$ 仍然较大接近开路.
            \item 对于转移特性, MOSFET 在 $U_{GS}$ 上升到某个阈值时将 DS 导通,等效电阻从极大值下降至极小值.因此 $U_{GS}-I_{DS}$ 图像分为三段:
                \begin{enumerate}
                    \item 在导通之前, DS 间电阻极大,电流极小,在 $U_{GS}-I_{DS}$ 图像上显示为一条 $x$ 轴上的线段.在实际测试中,该段的 $U_{GS}$ 范围约为 $0-3V$.
                    \item\label{enumerate:01} 在达到导通阈值后, $DS$ 间的等效电阻在一个小区间内快速下降至极小值,因此 $I_{DS}=\frac{U_{D}}{R_{L}+R_{DS}}$ 快速上升,在图像上显示为一段陡峭的上升沿.在实际测试中,该段的 $U_{GS}$ 范围约为 $3-3.5V$, 描点图像因电流增大速度过快,采样频率不足而出现断续,完全导通的阈值电压约为 $U_{T}=3.5V$
                    \item 在 $U_{GS}$ 继续上升超过导通阈值电压 $U_{T}=3.5V$ 时, $R_{DS}$ 下降到相对于 $R_{L}$ 的极小值,此时 $I_{DS}=\frac{U_{D}}{R_{L}+R_{DS}}\approx \frac{U_{D}}{R_{L}}$ 接近于定值 $5mA$,在图像上显示为一段平行于 $x$ 轴的线段.在实际测试中,该段的 $U_{GS}$ 范围约为 $3.5-5V$,对应的电流值等于理论计算值 $5mA$.
                \end{enumerate}
            \item 对于输出端口特性,在上述转移特性的 \ref{enumerate:01} 段对应的电压 $U_{GS}$ 下,等效电阻 $R_{DS}$ 随 $U_{GS}$ 上升快速下降,对应的电流 $I_{DS}$ 上升,此时在不同的 $U_{GS}$ 下 $U_{DS}-I_{DS}$ 图像曲线应当有明显的上扬,即对于同一个 $U_{DS}$, $U_{GS}$ 更高时对应的 $I_{DS}$ 更高.在实际测试中,在 \ref{enumerate:01} 段内均匀选取五个 $U_{GS}$,测量得到的曲线随 $U_{GS}$ 上升由下至上排列,符合理论计算.其中曲线的平直段应当为 $U_{GS}$ 对应的 MOSFET 饱和电流,在导通过程中,饱和电流也随着 $U_{GS}$ 的上升而上升.在 $U_{GS}=3.5V$ 时, MOSFET 接近完全导通, $U_{DS}$ 接近 0,图像表现为贴近 $y$ 轴的线段.
        \end{enumerate}
    \section{思考题}
        \begin{itemize}
            \item 电阻 $R_{L}$ 的值可以任意设置(比如设为 $10k\Omega$ 或 $47k\Omega$)吗,为什么?

                不可以,如果 $R_{L}$ 设置得过低,可能使得 $I_{DS}$ 过大,烧坏实验设备,引发危险.如果 $R_{L}$ 设置得过高,可能使得 $I_{DS}$ 过小,受到噪声影响加重,影响实验结果的清晰度.
            \item 用实验室的信号发生器提供三角波型电源时,其频率应该如何选取,设置为 $100$kHz 可以吗?

                电源频率应当与采样频率相配合,如果电源频率设置为 $100$kHz,由于便携设备的默认采样频率仅为 $f=800$kHz,将导致一个电源周期内仅进行极少量的采样,影响图线的绘制精度和数据分析结果.同样的,如果电源频率过低,由于用于绘图的默认采样数仅为 $s=2048$,采样数据可能无法完全展示电源周期内的数据变化.综上所述,合理的电源频率应保证在设备的一个采样时间段内有合理数量的电源周期(5 个左右),使用设备的默认值进行计算约为 $1/\frac{s}{5f}\approx 2$kHz.实际实验时,采样数为 $s_1=4096$,采样频率 $f_1=80$kHz,与使用的电源频率 $100$Hz 配合良好,与理论频率 $1 / \frac{s_1}{5f_1}\approx 97$Hz 非常接近.
        \end{itemize}
    \section{结论}
        本实验通过测量 U-I 关系分析了某一特定 MOSFET 元件的输入端口特性,转移特性与输出端口特性,半定量地分析了该元件在导通阶段以及导通前后的各端口特性.其中对导通阶段的输出端口特性进行了更详细的采样分析,得出了输出端口在该段的等效电阻快速下降且存在饱和电流的性质.

        在数据分析和实验步骤分析中,对实验设计中电阻和电源频率的选取进行了分析,证明了其合理性,特别对电源频率的选取方式进行了推测,得到了一个基于采样数和采样频率的经验公式.
\end{document}
