\input{../../lectures_preamble.tex}
\usepackage{../../lectures_preamble}

\begin{document}
    \section{光学1}
    增透过程有 $2ne=\left( 2k-1 \right) \frac{\lambda}{2}$.

    相干长度 $L=\frac{\lambda^2}{\Delta \lambda}$.其中 $\Delta \lambda$ 为线宽,将相干长度视为波长计算出来的频率 $\Delta \nu= \frac{c}{L}$ 称为频宽,其对应的时间称为相干时间.迈克耳孙干涉镜可以测量的厚度范围是 $\frac{L}{2}$.

    双缝干涉 $d\Delta x=\lambda L$.

    双棱镜等效间距 $d=2\alpha a\left( n-1 \right) $,菲涅尔双镜 $d=2L_1\phi$,其中 $L_1$ 为光源到镜距离, $\phi$ 是镜间夹角.

    双缝光强 $I=4I_1\cos ^2{\pi \frac{x}{\Delta x}}$.注意 $I_1$ 为单缝产生的光强,且系数为 $\pi$.

    迈克耳孙干涉仪光路往返,放干涉膜的影响要乘二.

    牛顿环的暗环半径为 $r=\sqrt{kR\lambda}$,明环半径根号内加上 $\frac{R\lambda}{2}$.中心由于有半波损失形成一暗斑.具体要看中心是否有半波损失.

    等倾条纹(反射) 光程差 $\delta=2h\sqrt{n_2^2-n_1^2\sin^2 i}+\frac{\lambda}{2}$,透射/迈克耳孙干涉使用 $2n_2 h\cos r=k\lambda$ 计算 $k$ 级明纹 (中央不一定为 $0$ 级),其中 $i$ 为透镜中心到光屏位置的倾角, $r$ 为折射角.注意由此计算外侧圆环的 $k$ 较小,内侧圆环的 $k$ 较大.每移动一个条纹,薄膜厚度的变化为 $\frac{\lambda}{2n}$,厚度增大亮斑增加.注意如果薄膜下方不是空气而是折射率更大的材料则上下都有半波损失相消,光程差不带 $\frac{\lambda}{2}$ 修正.

    单缝弗朗霍夫衍射暗条纹中心 $a\sin\theta=\pm k\lambda$,明条纹中心 $a\sin\theta=\pm\left( 2k+1 \right) \frac{\lambda}{2}$.中央明条纹的半角宽度为 $\theta\approx \sin\theta=\frac{\lambda}{a}$.对应的线宽度为 $\Delta x=2f \frac{\lambda}{a}$,其中 $f$ 为透镜焦距.

    多缝衍射的缺级发生在 $k=\pm \frac{d}{a}k'$.
\end{document}
