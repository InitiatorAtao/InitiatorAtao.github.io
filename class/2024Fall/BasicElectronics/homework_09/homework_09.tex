\documentclass[12pt]{article}

\author{陶文华}

\usepackage{../../homeworks_preamble}
\title{电子学基础-第九次作业}

\begin{document}
    \maketitle
    \section{11-40} 题图 11-40 所示电路中,阻抗 $Z$ 为何值时其上获得最大功率,并求此最大功率值.
    \begin{figure}[htbp]
        \centering
        \includegraphics[width=0.7\linewidth]{./figures/homework_09_figure_01.png}
    \end{figure}
    
    解:即求等效源阻抗的共轭复数,先求等效源阻抗,电流源相当于开路,令 $Z$ 两端电压为 $\dot{U}$ 与 $\dot{I}$ 关联,则 $\dot{I}=\frac{\dot{U}}{2-j 3} $,流控电压源电压为 $\frac{9 \dot{U}}{2-j 3}$,其所处支路电流为 $\frac{1}{3}\left( {\dot{U}-\frac{9 \dot{U}}{2-j 3}}  \right)$,故可知等效源阻抗为 $0.121+j 2.16 \ \mathrm{\Omega}$,故 $Z=0.121-j 2.16 \ \mathrm{\Omega}$ 时获得最大功率,计算得最大功率 $P=6325.2 \ \mathrm{W}$.
    \section{11-46} 电路如题图 11-46 所示,已知 $U_{S 1}=U_{S 2}=100 \ \mathrm{V}$,$\dot{U}_{S 1}$ 领先 $\dot{U}_{S 2}$ $60^{\circ}$, $Z_1=\left( 1-j 1 \right) \ \mathrm{\Omega}$, $Z_2=\left( 2+j 3 \right) \ \mathrm{\Omega}$, $Z_3=\left( 3+j 6 \right) \ \mathrm{\Omega}$.求电流 $\dot{I}$ 及两个电源各自发出的复功率 $\overline{S}_{1}$ 和 $\overline{S}_{2}$.
    \begin{figure}[htbp]
        \centering
        \includegraphics[width=0.5\linewidth]{./figures/homework_09_figure_02.png}
    \end{figure}
    \newpage

    解:不妨设 $\dot{U}_{S1}=100\angle 60^{\circ},\dot{U}_{S2}=100\angle 0^{\circ}$,则总电压 $\dot{U}=\dot{U}_{S1}-\dot{U}_{S2}=100\angle 120^{\circ} \ \mathrm{V}$,又有总阻抗 $Z=Z_1+Z_2+Z_3=\left( 6+j 8 \right) \ \mathrm{\Omega}$,故 $\dot{I}=\frac{\dot{U}}{Z}=\left( 3.92+ j9.19\right) \ \mathrm{A}=10\angle 66.9^{\circ} \ \mathrm{A}$, $\overline{S}_{1}=\dot{U}_{S 1}\dot{I}\angle\left(\phi_{u}-\phi_{i} \right) \approx 992.8-j 120.1 \ \mathrm{V\cdot A}$,同理 $\overline{S}_{2}\approx 392.3-j 920.0 \ \mathrm{V\cdot A}$.
    
    \section{11-47} 电路如題图 11-47 所示,其中 $Z_1=\left( 8+j 10 \right) \ \mathrm{\Omega},I_1=15 \ \mathrm{A}$, $Z_2$ 吸收的有功功率 $P_2=500 \ \mathrm{W}$,功率因数 $\cos\phi_2=0.7$ (滞后).求电流 $\dot{I}$ 及电路总功率因数.
    \begin{figure}[htbp]
        \centering
        \includegraphics[width=0.5\linewidth]{./figures/homework_09_figure_03.png}
    \end{figure}

    解: $P_2=I_2^2| Z_2 |\cos\phi_2=500 \ \mathrm{W}$,由并联性质可知 $\dot{I}_1Z_1=\dot{I}_2Z_2$,令 $\dot{I}_{1}=\left( a+j b \right) \ \mathrm{A}$,有 $\sqrt{a^2+b ^2}=15 \ \mathrm{A}$, $\dot{U}=\dot{I}_{1}Z_1=\left( 8a-10b+j \left( 10a+8b \right)  \right) \ \mathrm{V}$,

    $P_2=UI_2\cos\phi_2=0.7 \sqrt{\left( 8a-10b \right) ^2 +\left( 10a+8b \right) ^2}I_2=500 \ \mathrm{W}$,联立解得 $\dot{I}\approx 18.7\angle 1.13^{\circ} \ \mathrm{A}$,功率因数 $\cos\phi=0.64$.
    \section{7-1} 试用阶跃函数和延迟阶跃函数表示题图 7-1 所示各波形.
    \begin{figure}[htbp]
        \centering
        \includegraphics[width=0.7\linewidth]{./figures/homework_09_figure_04.png}
    \end{figure}
    解:
    \begin{enumerate}
        \item[(a)] $f=t[\epsilon\left( t \right) -\epsilon\left( t-1 \right) ]+\left( 2-t \right)[\epsilon\left( t-1 \right) -\epsilon\left( t-2 \right) ]$.
        \item[(b)] $f=[\epsilon\left( t \right) -\epsilon\left( t-1 \right) ]+\left( 2-t \right)[\epsilon\left( t-1 \right) -\epsilon\left( t-2 \right) ]$.
    \end{enumerate}
    \section{7-2} 绘出下列各函数的波形.
    \begin{itemize}
        \item $e^{-t}\epsilon\left( t \right) $

            如图:
            \newpage
            \begin{figure}[htbp]
                \centering
                \includegraphics[width=0.7\linewidth]{./figures/homework_09_figure_05.pdf}
            \end{figure}
            
        \item $e^{-\left( t-1 \right) \epsilon\left( t-1 \right) }$

            如图:
            \begin{figure}[htbp]
                \centering
                \includegraphics[width=0.7\linewidth]{./figures/homework_09_figure_06.pdf}
            \end{figure}
            
        \item $\left( t-1 \right) [\epsilon\left( t-1 \right) -\epsilon\left( t-2 \right) ]$

            如图:
            \newpage
            \begin{figure}[htbp]
                \centering
                \includegraphics[width=0.7\linewidth]{./figures/homework_09_figure_07.pdf}
            \end{figure}
            
    \end{itemize}
    \section{7-3} 求下列各函数表示的值.
    \begin{itemize}
        \item $\int_{-\infty}^{\infty}e^{t}\delta\left( t-2 \right) \mathrm{d}t=e^{t}\big|_{t=2}=e^{2}$.
        \item $\int_{-\infty}^{\infty}\left( t+\sin t \right) \delta\left( t+\frac{\pi}{3} \right) \mathrm{d}t=\left( t+\sin t \right) \big|_{t=-\frac{\pi}{3}}\approx -1.91$.
        \item $\int_{-\infty}^{\infty}\delta\left( t-t_0 \right) \epsilon\left( t-2t_0 \right) \mathrm{d}t=\epsilon\left( t-2t_0 \right) \big|_{t=t_0}=\epsilon\left( -t_0 \right) =0$.
    \end{itemize}
\end{document}
