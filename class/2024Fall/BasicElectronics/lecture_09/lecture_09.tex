\documentclass[12pt]{article}

\author{陶文华}

\usepackage{../../lectures_preamble}

\begin{document}
    \subsection{正弦电路中的功率}
        \begin{definition}
            (平均功率) 
        \begin{align}
            P=&\frac{1}{T}\int_{0}^{T}p\mathrm{d}t\nonumber\\
            =&\frac{1}{T}\int_{0}^{T}[UI\cos \phi -UI\cos{(2\omega t-\phi)}]\mathrm{d}t\nonumber\\
            =&UI\cos \phi\nonumber\\
            =&|Z|I I\cos \phi\nonumber\\
            =&I^2|Z|\cos\phi\nonumber\\
            =&I^2R\nonumber
        \end{align}\mn{注意模长形式的欧姆定律定义 $U=I|Z|$ 由 $\dot{U}=\dot{I}Z$ 取模而来,两侧均为复数形式故可以在不考虑辐角的情况下取模}
        \end{definition}
        \begin{definition}
            (视在功率,表观功率) $S=UI$,单位为 $\ \mathrm{VA}$ (伏安),反映电气设备的容量.
        \end{definition}

    \subsection{复功率}
        \begin{align}
            \dot{U}=&U\angle \psi_{u}\nonumber\\
            \dot{I}=&I\angle\psi_{i}\nonumber\\
            P=&UI\cos{(\psi_{u}-\psi_{i})}\nonumber\\
            =&UI\Re[e^{j(\psi_{u}-\psi_{i})}]\nonumber\\
            =&\Re[Ue^{j\psi_{u}}Ie^{-j\psi_{i}}]\nonumber\\
            =&\Re[\dot{U}\dot{I}^{*}]\nonumber
        \end{align}
        
        \begin{definition}
            (复功率) 记 $\overline{S}=\dot{U}\dot{I}^{*}$ 为复功率,单位 $\ \mathrm{VA}$,有:
        
        \begin{align}
            \overline{S}=&\dot{U}\dot{I}\angle(\psi_{u}-\psi_{i})\nonumber\\
            =&UI\angle\phi\nonumber\\
            =&UI\cos\phi+jUI\sin\phi\nonumber\\
            =&P+jQ\nonumber
        \end{align}
        
        其中 $Q=UI\sin\phi$ 称无功功率,单位为乏 (var),反映电源与负载之间交换能量的速率.
        \end{definition}
        \begin{remark}
            (有功,无功,视在功率的关系) 
            \begin{align}
                \overline{S}=&\dot{U}\dot{I}=UI\angle\phi=S\angle\phi=P+jQ\nonumber
            \end{align}
            视在功率 $S$ 是复功率 $\overline{S}$ 的模长,可表为 $S=\sqrt{P^2+Q^2}$.
        \end{remark}
        复功率守恒 $\sum_{k=1}^{b}\overline{S}_{k}=0$. 可拆成实部和虚部即有/无功功率守恒进行计算.\sn{视在功率为模长不一定守恒,在电阻性的电路中才守恒}

        最大功率传输:对于有复内阻 $Z_{i}$ 的非理想交流电压源,外电路阻抗 $Z_{L}=\overline{Z}_{i}$ 时外电路有功功率达到最大值,条件 $Z_{L}=\overline{Z}_{i}$ 称阻抗匹配,此时最大功率 $P_{\mathrm{max}}=\frac{U_{s}^2}{4R_{i}}$ 与直流结论相同.
\section{动态电路时域分析}
    \subsection{动态电路概述}
        \begin{definition}
            (动态电路,dynamic circuit) 即含有电容电感等动态元件的电路,对其的分析集中于其由一个稳态过渡到另一个稳态需要经历的过程,也称过渡过程或瞬态 (transient state),用微分方程描述.
        \end{definition}
        
        过渡过程产生的原因:
        \begin{itemize}
            \item 电路内部含有储能元件 $L,C$ :能量储存和释放需要时间.
            \item 电路结构发生变化:支路接入或断开或电路参数变化,统称换路.
        \end{itemize}

        \begin{definition}
            (一阶电路) 由一个独立储能元件组成的电路,描述电路的方程是一阶分方程.
        \end{definition}
        \begin{definition}
            (单位阶跃函数) 
            \begin{align}
                \epsilon(t)=&
                \begin{cases}
                    0 & (t<0)\\
                    1 & (t>0)
                \end{cases}\nonumber
            \end{align}
        \end{definition}
        通常使用其延迟形式 $\epsilon(t-t_0)=
        \begin{cases}
            0 & (t<t_0)\\
            1 & (t>t_0)
        \end{cases}$
        \begin{definition}
            (单位冲激函数) 
            \begin{align}
                \delta(t)=&
                \begin{cases}
                    0 & (t<0)\\
                    1 & (t>0)
                \end{cases}\nonumber
            \end{align}
            且 $\int_{-\infty}^{\infty}\delta(t)\mathrm{d}t=1$.
        \end{definition}
        延迟 $\delta(t-t_0)=0\ (t\ne t_0)$ 且 $(-\infty,\infty)$ 上积分为 $1$.
        
        其筛分性为 $\int_{-\infty}^{\infty}f(t)\delta(t-t_0)\mathrm{d}t=f(t_0)$\sn{需要 $f(t)$ 在 $t_0$ 处连续}.

        单位阶跃函数是单位冲激函数的原函数.
\end{document}
