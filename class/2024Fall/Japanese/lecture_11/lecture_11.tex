\input{../../lectures_preamble.tex}
\usepackage{../../lectures_preamble}

\begin{document}
    \section{新しい単語}
    \begin{figure}[htbp]
        \centering
        \begin{tabular}{l|l|l}
            くうこう 〔空港〕 &  (名) & 机场\\\hline
            けってい 〔決定〕 &  (名、他サ) & 决定,确定\\\hline
            おめでとうございます & (词组) & 恭喜\\\hline
            らいにち 〔来日〕 & (名、自サ) & 来日本\\\hline
            とうちゃく 〔到着〕 &  (名、自サ) & 到达\\\hline
            なんじ〔何時〕 & (名) & 几点\\\hline
            よてい〔予定〕 &  (名、他サ) & 预定,计划\\\hline
            むかえ〔迎え〕 & (名) & 迎接\\\hline
            ロビー & (名) & 门厅,前厅\\\hline
            かえる〔帰る〕 &  (自五) & 回,回去\\\hline
            しゅうしよく〔就職〕 & (名、自サ) & 就业\\\hline
            ぼうえき〔貿易〕 & (名、自サ) & 贸易\\\hline
            っかう〔使う〕 & (他五) & 使用\\\hline
            ざんぎよう〔残業〕 & (名、自サ) & 加班\\\hline
            ひび〔日々〕 & (名) & 每天\\\hline
            なっかしい〔懐かしい〕 & (形) & 令人怀念\\\hline
            じゅうきょ〔住居〕 & (名) & 住所\\\hline
            けっして〔決して〕 & (副) & 决(不),绝对(不)\\\hline
            しんばいだ〔心配だ〕 & (形动、自他サ) & 担心\\\hline
            いる〔要る〕 & (自五) & 需要\\\hline
            やすい〔安い〕 & (形) & 便宜\\\hline
            アパート & (名) & 公寓\\\hline
            しようかい〔紹介〕 & (名、他サ) & 介绍\\\hline
            あんしんだ〔安心だ〕 & (形动、自サ) & 放心\\\hline
            では & (接) & 那么\\\hline
            げんきだ〔元気だ〕 & (名、形动) & 健康\\\hline
            \~{}さま「\~{}様〕 &  & (接人名后)表示敬意\\\hline
            レストラン & (名) & 西餐馆\\\hline
            のど 〔喉〕 & (名) & 嗓子\\\hline
            かわく〔渇く〕 &  (自五) & 渴\\\hline
            のむ〔飲む〕 & (他五) & 喝\\\hline
            ふじさん〔富士山〕 & (名) & 富士山\\\hline
            すばらしい〔素晴らしい〕 & (形) & 极美,绝好\\\hline
            わらう〔笑う〕 &  (自五) & 笑\\\hline
            をの & (感) & 喂,那个\\\hline
            おてあらい〔お手洗い〕 & (名) & 洗手间\\\hline
            まっ〔待っ〕 &  (他五) & 等待
        \end{tabular}
    \end{figure}
    \section{本文}
    銭軍さん:

    東京大学大学院への留学決定、おめでとうございます。来らいにちは25日ですね。到着は何時の予定ですか。教えてくたさい。その日は土曜日なので、空港まで迎えに行きます。くうこうの到着ロビーでっています。

    わたしは日本に帰ってから、ある会社に就職しました。いま、中国との貿易の仕事をしています。中国語も使います。毎日、残業しています。北京での日々がとても懐かしいです。住ゅ居については決して心配いりません。安くていいアパートを紹介します。安心してください。

    では、お元気で。ご家族の皆様にもよろしく。

    3月8日

    渡
    辺
    早
    百
    合
    \section{会話}
    すず木:範さん、レストランがありますね。あそこに行って、ちょと休みましようか。わたし、のどが渇きました。何か飲みまよう。

    範:わたしもです。富士山は素晴らしいですね。すみません、すず木さん、写真を撮ってください。

    鈴木:いいですよ。じゃ、笑ってください。はい、撮ります。

    範:どうも。

    鈴木:いいえ。あの、範さん、ちょとお手洗いに行きたいので待っていくたさい。

    範:いいですよ。そこの店で待ています。
\end{document}
