\documentclass[12pt]{article}

\author{陶文华}

\usepackage{../../lectures_preamble}

\begin{document}
    \section{第五课-よろしくお願いします}
    \begin{definition}
        (名词句) 相当于汉语的 "...是/不是..."
        \begin{itemize}
            \item ...は...です(でした)\mn{括号内的假名和汉字相互对应,表示可用的替换或增添部分,表示时态或肯否定} : ...是...(过去式)
            \item ...は...ではありません(+でした) : ...不是...(过去式)
            \item ...は...で... : ...是...,也是...(以最后的时态和肯否定为准)
            \item ...は...でしょう : ...应该是...(表示推量)
        \end{itemize}
    \end{definition}
    \begin{definition}
        (动词句,ます型) 主要由主语+は(+宾语+を)+动词的变形构成,其中接宾语的动词称他动词,否则称自动词,动词在句中变为表示礼貌的ます型,可能的变形如下:
        \begin{itemize}
            \item 五段动词(以う段假名结尾):将结尾假名变为い段+ます
            \item 一段动词(以い,え段假名+る结尾):将结尾る去掉+ます
            \item さ变动词(仅する):变为します
            \item か变动词(仅くる):变为きます
        \end{itemize}
        ます部分的变形可能如下:
        \begin{itemize}
            \item ...ます(でした)(+か) : 肯定形式(过去式)(询问)
            \item ...ました(ませんでした) : 否定形式(过去式)
        \end{itemize}
    \end{definition}
    \begin{definition}
        (指示代词,こそあど 系词语) こ,そ,あ,ど 修饰代词的位置,分别表示离说话人近,离听话人近,离两人都远和未知的位置.

        可能的后缀和代表的含义如下:
        \begin{itemize}
            \item れ 代指事物
            \item こ 代指场所
            \item ちら 代指方向,或人的敬称
            \item の 代指事物,连体词\mn{连体词是后面必须接名词的修饰语,与后接名词共同构成名词短语}
            \item んな 代指某种性质或状态,连体词
            \item んなに 代指状态,副词\mn{修饰形容词或动词,修饰动词的用法尚未学习}
        \end{itemize}
    \end{definition}
    \subsection{助词一览}
    \begin{itemize}
        \item の 连接名词,表示 "的"
        \item は 表示主语或强调
        \item か 在句尾表示疑问
        \item を 表示宾语
    \end{itemize}
\end{document}
