\documentclass[12pt]{article}

\author{陶文华}

\usepackage{../../lectures_preamble}

\begin{document}
    \section{第七课-日本の四季}
    \begin{definition}
        (形容词句) 使用形容词修饰对象,句中的く表示形容词的く型,即将い改为く:
        \begin{itemize}
            \item ...は...です(かったです) : ...有...的特点(过去式)
            \item ...は...く+ないです(ありません) : ...没有...的特点(另一种表达)
            \item ...は...く+なかったです(ありませんでした) : ...过去没有...的特点(另一种表达)
        \end{itemize}
    \end{definition}
    \begin{definition}
        (形容词活用型) 在不同位置使用的形容词的不同词尾:
        \begin{itemize}
            \item い 基本型,连体型,后接修饰的名词
            \item く 连用型,接否定句式
            \item くて 中顿型,继续接形容词
            \item ければ 假定型,表示假定条件
            \item かろう 推量型,表示推测\mn{较少使用,一般直接在句尾使用でしょう表示推测}
        \end{itemize}
    \end{definition}
    \begin{definition}
        (比较句) 表示两者相比或最高性质:
        \begin{itemize}
            \item ...は...より...です : ...比...更...
            \item ...より...のほうが...です : ...不如...更...
            \item ...は...ほど...く+ないです(ありません) : ...不如...更...
            \item ...ほど...はありません : ...是最...\mn{这里后面要用形容词修饰的名词,表示具有该性质的物品范围}的
        \end{itemize}
    \end{definition}
    \subsection{助词一览}
        \begin{itemize}
            \item より 表示比较的对象
            \item に 表示具体时间\sn{每/明天/周/年等以及询问时不用加に,但是星期要加}
            \item に 表示说明事物性质或状态时比较的基准或对象\sn{如修饰"适合..."中的对象,或者表示"对...怎么样"中的对象}
            \item で 表示范围,相当于"在..."
            \item まで 表示终点,相当于"到...(为止)"
        \end{itemize}
\end{document}
