\documentclass[10pt]{homework}
% \usepackage{import}
% \usepackage{xifthen}
% \usepackage{pdfpages}
% \usepackage{transparent}
% \newcommand{\incfig}[1]{%
    % \def\svgwidth{\columnwidth}
    % \import{./figures/}{#1.pdf_tex}
% }

\name{Tao Wenhua} % Replace (Student Name) with your name.
\id{2023010782}
\term{2024 Autumn}
\course{Formal Languages and Automata}
\hwnum{3}

%\hwname{(Name)}          % Uncomment and replace (Name) with the type of the
                          % homework (e.g, Assignment, Problem Set, etc.) if you
                          % don't want the document to be labeled as "Homework."
%\problemname{(Name)}     % Uncomment and replace (Name) with the desired label
                          % for problems created with the problem environment.
%\solutionname{(Name)}    % Uncomment and replace (Name) with the desired label
                          % for solutions created with the solution environment.

% Load any other packages you need here.

\begin{document}

\begin{problem} (Exercise 2.3.4 (c)) Convert regular expression $00{(0+1)}^{*}$
  to an NFA with \(\epsilon\)-transitions using Thompson's method.
\end{problem}

\begin{solution}
    Answer like:
  \begin{table}[h]
    \centering
    \begin{tabular}{r||c|c|c} % chktex 44
      & $\epsilon$ & $0$ & $1$\\\hline\hline % chktex 44
        $\rightarrow 1$ & $\emptyset$ & $\{2\}$ & $\emptyset$\\
        $2$ & $\{3\}$ & $\emptyset$ & $\emptyset$\\
        $3$ & $\emptyset$ & $\{4\}$ & $\emptyset$\\
        $4$ & $\{5\}$ & $\emptyset$ & $\emptyset$\\
        $5$ & $\{6\}$ & $\emptyset$ & $\emptyset$\\
        $6$ & $\{7,8\}$ & $\emptyset$ & $\emptyset$\\
        $7$ & $\emptyset$ & $\{9\}$ & $\emptyset$\\
        $8$ & $\emptyset$ & $\emptyset$ & $\{10\}$\\
        $9$ & $\{11\}$ & $\emptyset$ & $\emptyset$\\
        $10$ & $\{11\}$ & $\emptyset$ & $\emptyset$\\
        $11$ & $\{6,12\}$ & $\emptyset$ & $\emptyset$\\
        $*12$ & $\emptyset$ & $\emptyset$ & $\emptyset$\\
    \end{tabular}
  \end{table}
\end{solution}

\begin{problem} (Exercise 2.3.6)
  Let $A = (Q, \Sigma, \delta, q_{0}, \{q_{f}\})$ be an $\epsilon$-NFA such that
  there are no transitions into $q_{0}$ and no transitions out of $q_{f}$.
  Describe the language accepted by each of the following modifications of $A$,
  in terms of $L = L(A)$:
  \begin{parts}
    \part\label{2.a} The automaton constructed from $A$ by adding an
    $\epsilon$-transition from $q_{f}$ to $q_{0}$.
    \part\label{2.b} The automaton constructed from $A$ by adding an $\epsilon$-transition
    from $q_{0}$ to every state reachable from $q_{0}$ (along a path whose
    labels may include symbols of $\Sigma$ as well as $\epsilon$).
    \part\label{2.c} The automaton constructed from $A$ by adding an $\epsilon$-transition
    to $q_{f}$ from every state that can reach $q_{f}$ along some path.
    \part\label{2.d} The automaton constructed from $A$ by doing both (b) and (c).
  \end{parts}
\end{problem}

\begin{solution}
    \begin{parts}
        \part $L^{+}$, since now can loop by adding any suffix $s\in L(A)$
        \part The new language includes all suffixes of all strings in $L$. (suffixes may be the whole string)
        \part The new language includes all prefixes of all strings in $L$. (prefixes may be the whole string)
        \part The new language includes all substrings of all strings in $L$. (maybe the whole string)
    \end{parts}
\end{solution}

\begin{problem} (Exercise 3.2.3)
  Convert the following DFA to a regular expression, using the state-elimination
  technique of Section 3.2.2.
  \begin{table}[h]
    \centering
    \begin{tabular}{r||c|c} % chktex 44
      & $0$ & $1$\\\hline\hline % chktex 44
      $\rightarrow * p$ & $s$ & $p$\\
      $q$ & $p$ & $s$\\
      $r$ & $r$ & $q$\\
      $s$ & $q$ & $r$
    \end{tabular}
  \end{table}
\end{problem}

\begin{solution}
    First eliminate $r$, which turns into $10^{*}1$ from $s$ to $q$. Then eliminate $p,s$ into self loop $0(0+10^{*}1)(1(0+10^{*}1))^{*}0$ in $q$. So for $q$ the final RE is $(1+0(0+10^{*}1)(1(0+10^{*}1))^{*})^{*}$
\end{solution}

\begin{problem}
  Prove or disprove each of the following statements about regular expressions.
  \begin{parts}
    \part\ (Exercise 3.4.1 (g)) ${(\epsilon + R)}^{*} = R^{*}$.
    \part\ (Exercise 3.4.2 (b)) ${(RS + R)}^{*}R = R{(RS + R)}^{*}$.
    \part\ (Exercise 3.4.2 (d)) ${(R + S)}^{*}S = {(R^{*}S)}^{*}$.
  \end{parts}
\end{problem}

\begin{solution}
    \begin{parts}
        \part True. $(\epsilon+R)^{*}=(\epsilon^{*}R^{*})^{*}=(R^{*})^{*}=R^{*}$.
        \part False. $RRS$ can be accepted by the right side but not the left side. (If this is abstract, use Theorm 3.13 to turn to concrete)
        \part False. $\epsilon$ can be accepted by the right side but not the left side.
    \end{parts}
\end{solution}

\begin{problem}
  Construct a DFA for each of the regular languages below:
  \begin{parts}
    \part\ All strings over $\{a, b\}$ that have and only have odd number of
    substrings $ab$.
    \part\ All strings over $\{a, b\}$ that contain $ab$ but not $bb$ as substring.
  \end{parts}
\end{problem}

\begin{solution}
    \begin{parts}
        \part Answer like:
  \begin{table}[h]
    \centering
    \begin{tabular}{r||c|c} % chktex 44
      & $a$ & $b$\\\hline\hline % chktex 44
      $\rightarrow s$ & $a$ & $s$\\
      $a$ & $a$ & $ab$\\
      $*ab$ & $aba$ & $ab$\\
      $aba$ & $aba$ & $s$
    \end{tabular}
  \end{table}
        \newpage
        \part Answer like:
  \begin{table}[h]
    \centering
    \begin{tabular}{r||c|c} % chktex 44
      & $a$ & $b$\\\hline\hline % chktex 44
      $\rightarrow s$ & $a$ & $b$\\
      $a$ & $a$ & $ab$\\
      $b$ & $a$ & $f$\\
      $*ab$ & $aba$ & $f$\\
      $*aba$ & $aba$ & $ab$\\
      $f$ & $f$ & $f$
    \end{tabular}
  \end{table}
    \end{parts}
\end{solution}

\begin{problem}
  Give a regular expression for each of the regular language below:
  \begin{parts}
    \part\ $\{xwx^{R} \mid x, w \in {(a+b)}^{+}\}$.
    \part\ $\bigl\{w \mid w \in {\{0,1\}}^{*}, w \text{ has at least three 1's and
      the third from the last}\linebreak \text{position is a 1} \bigr\}$.
  \end{parts}
\end{problem}

\begin{solution}
    \begin{parts}
        \part $a(a+b)^{+}a+b(a+b)^{+}b$
        \part $(0+1)^{*}1(0+1)^{*}1(0+1)^{*}100\\+(0+1)^{*}1(0+1)^{*}1(01+10)\\+(0+1)^{*}111$
    \end{parts}
\end{solution}

\end{document}
