\documentclass[10pt]{homework}

\name{Tao Wenhua} % Replace (Student Name) with your name.
\id{2023010782}
\term{2024 Autumn}
\course{Formal Languages and Automata}
\hwnum{8}

%\hwname{(Name)}          % Uncomment and replace (Name) with the type of the
                          % homework (e.g, Assignment, Problem Set, etc.) if you
                          % don't want the document to be labeled as "Homework."
%\problemname{(Name)}     % Uncomment and replace (Name) with the desired label
                          % for problems created with the problem environment.
%\solutionname{(Name)}    % Uncomment and replace (Name) with the desired label
                          % for solutions created with the solution environment.

% Load any other packages you need here.

\begin{document}

\begin{problem} (Exercise 6.3.2)
  Convert the grammar
  \begin{align*}
    S & \rightarrow aAA\\
    A & \rightarrow aS \,|\, bS \,|\, a
  \end{align*}
  to a PDA that accepts the same language by empty stack.
\end{problem}

\begin{solution}
    Answer PDA like:
    \begin{align}
        P=&\left( \{q\},\{a,b\},\{S,A\},\delta,q,S,\emptyset \right) \nonumber
    \end{align}
    where:
    \begin{enumerate}
        \item $\delta\left( q,a,S \right) =\{\left( q,AA \right) \}$ 
        \item $\delta\left( q,a,A \right) =\{\left( q,S \right) ,\left( q,\epsilon \right) \}$ 
        \item $\delta\left( q,b,A \right) =\{\left( q,S \right) \}$
    \end{enumerate}
\end{solution}

\begin{problem} (Exercise 6.3.4)
  Consider the PDA
  \begin{equation*}
    P = \bigl( \{q, p\}, \{0, 1\}, \{Z_{0}, X\}, \delta, q, Z_{0}, \{p\} \bigr)
  \end{equation*}
  having the following transition function:
  \begin{enumerate}
    \item $\delta(q, 0, Z_{0}) = \{(q, XZ_{0})\}$.
    \item $\delta(q, 0, X) = \{(q, XX)\}$.
    \item $\delta(q, 1, X) = \{(q, X)\}$.
    \item $\delta(q, \epsilon, X) = \{(p, \epsilon)\}$.
    \item $\delta(p, \epsilon, X) = \{(p, \epsilon)\}$.
    \item $\delta(p, 1, X) = \{(p, XX)\}$.
    \item $\delta(p, 1, Z_{0}) = \{(p, \epsilon)\}$.
  \end{enumerate}
  Convert the PDA to a context-free grammar.
\end{problem}

\begin{solution}
    Answer CFG like:
    \begin{align}
        G=&\left( \{P,Q\},\{0,1\},A,P \right) \nonumber
    \end{align}
    where:
    \begin{align}
        A:\{P\rightarrow& 0Q\nonumber\\
        Q\rightarrow &1Q\ |\ 0Q\ |\ \epsilon\}\nonumber
    \end{align}
\end{solution}

\begin{problem} (Exercise 6.3.5 (c)) Devise a PDA that accepts the language
  $\{0^{n}1^{m} \mid n \le m \le 2n \}$ by empty stack.
  You may, if you wish, first construct a grammar for the language, and then
  convert to a PDA\@.
\end{problem}

\begin{solution}
    CFG have transition like $P\rightarrow 0P 1\ |\ 0P 11\ |\ \epsilon$. So the answer PDA like:
    \begin{align}
        P=&\left( \{q,p\},\{0,1\},\{Z_0,X\},\delta,q,Z_0,\emptyset \right) \nonumber
    \end{align}
    where:
    \begin{enumerate}
        \item $\delta\left( q,0,Z_0 \right) =\{\left( q,XZ_0 \right),\left( q,XXZ_0 \right)  \}$
        \item $\delta\left( q,0,X \right) =\{\left( q,XX \right) ,\left( q,XXX \right) \}$ 
        \item $\delta\left( q,\epsilon,X \right) =\{\left( p,X \right) \}$ 
        \item $\delta\left( q,\epsilon,Z_0 \right) =\{\left( p,Z_0 \right) \}$
        \item $\delta\left( p,1,X \right) =\{\left( p,\epsilon \right) \}$ 
        \item $\delta\left( p,\epsilon,Z_0 \right) =\{\left( p,\epsilon \right) \}$
    \end{enumerate}
\end{solution}

\begin{problem} (Exercise 6.4.2 (c))
  Give a deterministic pushdown automaton to accept the following language
  \begin{equation*}
    \{0^{n}1^{m}0^{n} \mid n \text{ and } m \text{ are arbitrary}\}.
  \end{equation*}
\end{problem}

\begin{solution}
    Answer DPDA like:
    \begin{align}
        P=&\left( \{q,m,p,a\},\{0,1\},\{Z_0,X\},\delta,q,Z_0,\{a\} \right) \nonumber
    \end{align}
    where:
    \begin{enumerate}
        \item $\delta\left( q,0,Z_0 \right) =\left( q,XZ_0 \right) $ 
        \item $\delta\left( q,0,X \right) =\left( q,XX \right) $ 
        \item $\delta\left( q,1,Z_0 \right) =\left( a,Z_0 \right) $
        \item $\delta\left( q,1,X \right) =\left( m,X \right) $
        \item $\delta\left( m,1,X \right) =\left( m,X \right) $ 
        \item $\delta\left( m,0,X \right)=\left( p,\epsilon \right)  $
        \item $\delta\left( p,1,X \right) =\left( p,\epsilon \right) $
        \item $\delta\left( p,\epsilon,Z_0 \right) =\left( a,\epsilon \right) $
        \item $\delta\left( a,1,Z_0 \right) =\left( a,Z_0 \right) $
    \end{enumerate}
\end{solution}

\begin{problem} (Exercise 6.4.3)
  We can prove Theorem 6.19 in three parts:
  \begin{parts}
    \part\ Show that if $L = N(P)$ for some DPDA $P$, then $L$ has the prefix
    property.
    \part\ Show that if $L = N(P)$ for some DPDA $P$, then there exists a DPDA
    $P'$ such that $L = L(P')$.
    \part\ Show that if $L$ has the prefix property and is $L(P')$ for some DPDA
    $P'$, then there exists a DPDA $P$ such that $L = N(P)$.
  \end{parts}
\end{problem}

\begin{solution}
    \begin{parts}
        \part Since state with empty stack cannot move again, if $w\in L$, in other word, $w$ reach empty stack in $P$, all strings $s$ with real prefix $w$ reach empty stack in prefix $w$. Then no more moves are accepted, so $s\notin L$.
        \part Construct $P'$ by making all origin final states no more accepted, then add new start state $s$, final state $e$ and initial stack symbol $S$. Then add transition $\delta\left( s,\epsilon,S \right)=\left( q,Z_0S \right)$ and $\delta\left( *,\epsilon,S \right) =\left( e,\epsilon \right) $. Where $Z_0$ is the origin initial stack symbol, $q$ is the origin start state, $*$ refers to any state in $P$.

        Then we proof $L=L\left( P' \right) $, or $\forall w,w\in N\left( P \right) \Leftrightarrow w\in L\left( P' \right) $.
        \begin{enumerate}
            \item $\Rightarrow $ : In $P$ we have $\left( q,w,Z_0 \right) \vdash^{*} \left( a,\epsilon,\epsilon \right) $ for some middle state $a$. Then for the same transition in $P'$ we have $\left( q,w,Z_0S \right) \vdash^{*} \left( a,\epsilon,S \right) $ and $\left( a,\epsilon,S \right) \vdash \left( e,\epsilon,\epsilon \right) $ where $e$ is accepted. So $w\in L\left( P' \right) $.
            \item $\Leftarrow$: In $P'$ we have $\left( s,w,S \right) \vdash \left( q,w,Z_0S \right) \vdash^{*} \left( a,\epsilon,S \right) \vdash \left( e,\epsilon,\epsilon \right) $ for some $a$. Since transitions in $P$ cannot remove bottom symbol $S$, we can savely remove $S$ to get $\left( q,w,Z_0 \right) \vdash^{*}\left( a,\epsilon,\epsilon \right) $ in $P$, or $w\in N\left( P \right) $.
        \end{enumerate}
        So the $P'$ above has $L=L\left( P' \right) $.
        \part Prefix property ensures final states in $P'$ cannot transfer to other final states. So to construct $P$, we can remove all transitions out of final states. Then add new start state $s$ ,final state $e$ and initial stack symbol $S$. Then add transition $\delta\left( s,\epsilon,S \right) =\left( q,Z_0S \right) $ the same and $\delta\left( a,\epsilon,* \right) =\delta\left( e,\epsilon,* \right) =\left( e,\epsilon \right) $ where $a$ is the origin final state, $*$ refers to any stack symbols.

        Then we proof $L=N\left( P \right) $, or $\forall w,w\in L\left( P' \right) \Leftrightarrow w\in N\left( P \right) $.
        \begin{enumerate}
            \item $\Rightarrow $: In $P'$ we have $\left( q,w,Z_0 \right) \vdash^{*}\left( a,\epsilon,* \right) $ where $a$ is final state. So in $P$ we have $\left( s,w,S \right) \vdash \left( q,w,Z_0S \right) \vdash^{*}\left( a,\epsilon,*S \right) \vdash \left( e,\epsilon,*S \right) \vdash^{*} \left( e,\epsilon,\epsilon \right) $, so $w\in N\left( P \right) $.
            \item $\Leftarrow$: In $P$ we have $\left( s,w,S \right) \vdash\left( q,w,Z_0S \right) \vdash^{*}\left( a,\epsilon,*S \right) \vdash \left( e,\epsilon,*S \right) \vdash^{*}\left( e,\epsilon,\epsilon\right) $. Symbol $S$ is keeped because transition in $P'$ cannot remove it. So $\left( q,w,Z_0 \right) \vdash \left( a,\epsilon,* \right) $, $w\in L\left( P' \right) $.
        \end{enumerate}
        So the $P$ above has $L=N\left( P \right) $.
    \end{parts}
\end{solution}

\end{document}
