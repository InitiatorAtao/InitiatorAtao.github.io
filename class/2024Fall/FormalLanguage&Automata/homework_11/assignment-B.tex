\documentclass[10pt]{homework}

\name{Tao Wenhua} % Replace (Student Name) with your name.
\id{2023010782}
\term{2024 Autumn}
\course{Formal Languages and Automata}
\hwnum{11}

%\hwname{(Name)}          % Uncomment and replace (Name) with the type of the
                          % homework (e.g, Assignment, Problem Set, etc.) if you
                          % don't want the document to be labeled as "Homework."
%\problemname{(Name)}     % Uncomment and replace (Name) with the desired label
                          % for problems created with the problem environment.
%\solutionname{(Name)}    % Uncomment and replace (Name) with the desired label
                          % for solutions created with the solution environment.

% Load any other packages you need here.

\begin{document}

\begin{problem} (Exercise 8.4.4) Consider the nondeterministic Turing machine
  \(M = (\{q_0, q_1, q_2, q_f\}, \{0, 1\}, \{0, 1, B\}, \delta, q_0, B, \{q_f\})\).
  Informally but clearly describe the language \(L(M)\) if \(M\) consists of the
  following sets of rules:
  \begin{align*}
    & \delta(q_{0}, 0) = \{(q_{0}, 1, R), (q_{1}, 1, R)\}\\
    & \delta(q_{1}, 1) = \{(q_{2}, 0, L)\},\\
    & \delta(q_{2}, 1) = \{(q_{0}, 1, R)\},\\
    & \delta(q_{1}, B) = \{(q_{f}, B, R)\}.
  \end{align*}
\end{problem}

\begin{solution}
    $L\left( M \right) =L\left( 0\left( 0+1 \right)^{*} \right) $ expressed by regular expression. Because $q_0$ must read a $0$ at the beginning, then all $0$s and $1$s are turned to $1$s by the TM.
\end{solution}

\begin{problem}
  A Turing machine with left-most move is similar to a semi-infinite-tape Turing
  machine, but the transition function $\delta$ takes values in
  $Q \times \Gamma \times \{\text{R}, \text{L-Most}\}$.
  If $\delta(q, a) = (p, b, \text{L-Most})$, the machine's head moves to the
  left-most cell of the tape after it writes $b$ on the tape and enters state
  $p$.
  Note that these machines do not have the usual ability to move the head one
  symbol left.
  Show that Turing machines with left-most moves recognize all languages in
  \class{RE}.
  (Hint: You may need to use the multi-track method and two markers.)
\end{problem}

\begin{solution}
    For all languages in RE, we can build a TM with left-most move from their semi-infinite-tape TM ($L$ in RE must have a normal TM, then semi-infinite-tape TM can simulate that TM).

    To simulate semi-infinite-tape TM $M$ from left-most move TM $M'$, use multi-track method (multi-track is only simulated by items in $\Gamma$, so the changes in $\delta$ does not influent it). One track records normal symbols in $M$, another record marks when head need a left-move. The head leave a mark (named $x$) in the second track then move to left-most. After that it use two different marks (named $y,z$) at the left-most and push it forward until $x$, that is, use extra state $q_0,q_1,q_2,q_3$ and start from left-most:
    \begin{align}
        \delta\left( q_0,\left( a,B \right)  \right) =&\left( q_1,\left( a,y \right) ,\text{L-Most} \right) \nonumber\\
        \delta\left( q_1,\left( a,B \right)  \right) =&\left( q_1,\left( a,B \right) ,\text{R} \right) \nonumber\\
        \delta\left( q_1,\left( a,z \right)  \right) =&\left( q_1,\left( a,B \right) ,\text{R} \right) \nonumber\\
        \delta\left( q_1,\left( a,y \right)  \right) =&\left( q_2,\left( a,z \right) ,\text{R} \right) \nonumber\\
        \delta\left( q_2,\left( a,B \right)  \right) =&\left( q_1,\left( a,y \right) ,\text{L-Most} \right) \nonumber\\
        \delta\left( q_2,\left( a,x \right)  \right) =&\left( q_3,\left( a,B \right) ,\text{R} \right) \nonumber\\
        \delta\left( q_3,\left( a,B \right)  \right) =&\left( q_3,\left( a,B \right) ,\text{R} \right) \nonumber\\
        \delta\left( q_3,\left( a,z \right)  \right) =&\left( p,\left( a,B \right) ,\text{R} \right) \nonumber
    \end{align}
    Where $p$ is the next state in origin $M$. note that $q_0$ will reject mark $x$ which means you do a left move in the left-most.
\end{solution}

\end{document}
